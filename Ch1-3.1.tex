
% A LaTeX template for MSc Thesis submissions to 
% Politecnico di Milano (PoliMi) - School of Industrial and Information Engineering
%
% S. Bonetti, A. Gruttadauria, G. Mescolini, A. Zingaro
% e-mail: template-tesi-ingind@polimi.it
%
% Last Revision: October 2021
%
% Copyright 2021 Politecnico di Milano, Italy. NC-BY

\documentclass{Configuration_Files/PoliMi3i_thesis}

%------------------------------------------------------------------------------
%	REQUIRED PACKAGES AND  CONFIGURATIONS
%------------------------------------------------------------------------------

% CONFIGURATIONS
\usepackage{parskip} % For paragraph layout
\usepackage{setspace} % For using single or double spacing
\usepackage{emptypage} % To insert empty pages
\usepackage{multicol} % To write in multiple columns (executive summary)
\setlength\columnsep{15pt} % Column separation in executive summary
\setlength\parindent{0pt} % Indentation
\raggedbottom  

% PACKAGES FOR TITLES
\usepackage{titlesec}
% \titlespacing{\section}{left spacing}{before spacing}{after spacing}
\titlespacing{\section}{0pt}{3.3ex}{2ex}
\titlespacing{\subsection}{0pt}{3.3ex}{1.65ex}
\titlespacing{\subsubsection}{0pt}{3.3ex}{1ex}
\usepackage{color}

% PACKAGES FOR LANGUAGE AND FONT
\usepackage[english]{babel} % The document is in English  
\usepackage[utf8]{inputenc} % UTF8 encoding
\usepackage[T1]{fontenc} % Font encoding
\usepackage[10pt]{moresize} % Big fonts

% PACKAGES FOR IMAGES
\usepackage{graphicx}
\usepackage{transparent} % Enables transparent images
\usepackage{eso-pic} % For the background picture on the title page
\usepackage{subfig} % Numbered and caption subfigures using \subfloat.
\usepackage{tikz} % A package for high-quality hand-made figures.
\usetikzlibrary{}
\graphicspath{{./Images/}} % Directory of the images
\usepackage{caption} % Coloured captions
\usepackage{xcolor} % Coloured captions
\usepackage{amsthm,thmtools,xcolor} % Coloured "Theorem"
\usepackage{float}
\usepackage{tkz-euclide}

% STANDARD MATH PACKAGES
\usepackage{amsmath}
\usepackage{amsthm}
\usepackage{amssymb}
\usepackage{amsfonts}
\usepackage{bm}
\usepackage[overload]{empheq} % For braced-style systems of equations.
\usepackage{fix-cm} % To override original LaTeX restrictions on sizes

% PACKAGES FOR TABLES
\usepackage{tabularx}
\usepackage{longtable} % Tables that can span several pages
\usepackage{colortbl}

% PACKAGES FOR ALGORITHMS (PSEUDO-CODE)
\usepackage{algorithm}
\usepackage{algorithmic}

% PACKAGES FOR REFERENCES & BIBLIOGRAPHY
\usepackage[colorlinks=true,linkcolor=black,anchorcolor=black,citecolor=black,filecolor=black,menucolor=black,runcolor=black,urlcolor=black]{hyperref} % Adds clickable links at references
\usepackage{cleveref}
\usepackage[square, numbers, sort&compress]{natbib} % Square brackets, citing references with numbers, citations sorted by appearance in the text and compressed
\bibliographystyle{abbrvnat} % You may use a different style adapted to your field

% OTHER PACKAGES
\usepackage{pdfpages} % To include a pdf file
\usepackage{afterpage}
\usepackage{lipsum} % DUMMY PACKAGE
\usepackage{fancyhdr} % For the headers
\fancyhf{}

% Input of configuration file. Do not change config.tex file unless you really know what you are doing. 
% Define blue color typical of polimi
\definecolor{bluepoli}{cmyk}{0.4,0.1,0,0.4}

% Custom theorem environments
\declaretheoremstyle[
  headfont=\color{bluepoli}\normalfont\bfseries,
  bodyfont=\color{black}\normalfont\itshape,
]{colored}

% Set-up caption colors
\captionsetup[figure]{labelfont={color=bluepoli}} % Set colour of the captions
\captionsetup[table]{labelfont={color=bluepoli}} % Set colour of the captions
\captionsetup[algorithm]{labelfont={color=bluepoli}} % Set colour of the captions

\theoremstyle{colored}
\newtheorem{theorem}{Theorem}[chapter]
\newtheorem{proposition}{Proposition}[chapter]

% Enhances the features of the standard "table" and "tabular" environments.
\newcommand\T{\rule{0pt}{2.6ex}}
\newcommand\B{\rule[-1.2ex]{0pt}{0pt}}

% Pseudo-code algorithm descriptions.
\newcounter{algsubstate}
\renewcommand{\thealgsubstate}{\alph{algsubstate}}
\newenvironment{algsubstates}
  {\setcounter{algsubstate}{0}%
   \renewcommand{\STATE}{%
     \stepcounter{algsubstate}%
     \Statex {\small\thealgsubstate:}\space}}
  {}

% New font size
\newcommand\numfontsize{\@setfontsize\Huge{200}{60}}

% Title format: chapter
\titleformat{\chapter}[hang]{
\fontsize{50}{20}\selectfont\bfseries\filright}{\textcolor{bluepoli} \thechapter\hsp\hspace{2mm}\textcolor{bluepoli}{|   }\hsp}{0pt}{\huge\bfseries \textcolor{bluepoli}
}

% Title format: section
\titleformat{\section}
{\color{bluepoli}\normalfont\Large\bfseries}
{\color{bluepoli}\thesection.}{1em}{}

% Title format: subsection
\titleformat{\subsection}
{\color{bluepoli}\normalfont\large\bfseries}
{\color{bluepoli}\thesubsection.}{1em}{}

% Title format: subsubsection
\titleformat{\subsubsection}
{\color{bluepoli}\normalfont\large\bfseries}
{\color{bluepoli}\thesubsubsection.}{1em}{}

% Shortening for setting no horizontal-spacing
\newcommand{\hsp}{\hspace{0pt}}

\makeatletter
% Renewcommand: cleardoublepage including the background pic
\renewcommand*\cleardoublepage{%
  \clearpage\if@twoside\ifodd\c@page\else
  \null
  \AddToShipoutPicture*{\BackgroundPic}
  \thispagestyle{empty}%
  \newpage
  \if@twocolumn\hbox{}\newpage\fi\fi\fi}
\makeatother

%For correctly numbering algorithms
\numberwithin{algorithm}{chapter}

%----------------------------------------------------------------------------
%	NEW COMMANDS DEFINED
%----------------------------------------------------------------------------

% EXAMPLES OF NEW COMMANDS
\newcommand{\bea}{\begin{eqnarray}} % Shortcut for equation arrays
\newcommand{\eea}{\end{eqnarray}}
\newcommand{\e}[1]{\times 10^{#1}}  % Powers of 10 notation

%----------------------------------------------------------------------------
%	ADD YOUR PACKAGES (be careful of package interaction)
%----------------------------------------------------------------------------
\usepackage{pgfplots}
\usepackage{mathrsfs}
\usepackage{cancel}

%----------------------------------------------------------------------------
%	ADD YOUR DEFINITIONS AND COMMANDS (be careful of existing commands)
%----------------------------------------------------------------------------
\newtheorem{definition}{Definition}[chapter]
%\newtheorem{proposition}{Definition}[section]
\DeclareMathOperator*{\subsetminus}{\mathrel{\dot{\subset}}}
%----------------------------------------------------------------------------
%	BEGIN OF YOUR DOCUMENT
%----------------------------------------------------------------------------

\begin{document}

\fancypagestyle{plain}{%
\fancyhf{} % Clear all header and footer fields
\fancyhead[RO,RE]{\thepage} %RO=right odd, RE=right even
\renewcommand{\headrulewidth}{0pt}
\renewcommand{\footrulewidth}{0pt}}

%----------------------------------------------------------------------------
%	TITLE PAGE
%----------------------------------------------------------------------------

\pagestyle{empty} % No page numbers
\frontmatter % Use roman page numbering style (i, ii, iii, iv...) for the preamble pages

\puttitle{
	title=QUANTUM COMPUTATION ON THE TORIC CODE, % Title of the thesis
	name=Alessia Conca Roncari, % Author Name and Surname
	course=Computer Science and Engineering - Ingegneria Informatica, % Study Programme (in Italian)
	ID  = 996809,  % Student ID number (numero di matricola)
	advisor= Prof. Michele Correggi, % Supervisor name
	coadvisor={Massimo Moscolari}, % Co-Supervisor name, remove this line if there is none
	academicyear={2023-24},  % Academic Year
} % These info will be put into your Title page 

%----------------------------------------------------------------------------
%	PREAMBLE PAGES: ABSTRACT (inglese e italiano), EXECUTIVE SUMMARY
%----------------------------------------------------------------------------
\startpreamble
\setcounter{page}{1} % Set page counter to 1

% ABSTRACT IN ENGLISH
\chapter*{Abstract} 
Here goes the Abstract in English of your thesis followed by a list of keywords.
The Abstract is a concise summary of the content of the thesis (single page of text)
and a guide to the most important contributions included in your thesis.
The Abstract is the very last thing you write.
It should be a self-contained text and should be clear to someone who hasn't (yet) read the whole manuscript.
The Abstract should contain the answers to the main scientific questions that have been addressed in your thesis.
It needs to summarize the adopted motivations and the adopted methodological approach as well as the findings of your work and their relevance and impact.
The Abstract is the part appearing in the record of your thesis inside POLITesi,
the Digital Archive of PhD and Master Theses (Laurea Magistrale) of Politecnico di Milano.
The Abstract will be followed by a list of four to six keywords.
Keywords are a tool to help indexers and search engines to find relevant documents.
To be relevant and effective, keywords must be chosen carefully.
They should represent the content of your work and be specific to your field or sub-field.
Keywords may be a single word or two to four words. 
\\
\\
\textbf{Keywords:} here, the keywords, of your thesis % Keywords

% ABSTRACT IN ITALIAN
\chapter*{Abstract in lingua italiana}
Qui va l'Abstract in lingua italiana della tesi seguito dalla lista di parole chiave.
\\
\\
\textbf{Parole chiave:} qui, vanno, le parole chiave, della tesi % Keywords (italian)

%----------------------------------------------------------------------------
%	LIST OF CONTENTS/FIGURES/TABLES/SYMBOLS
%----------------------------------------------------------------------------

% TABLE OF CONTENTS
\thispagestyle{empty}
\tableofcontents % Table of contents 
\thispagestyle{empty}
\cleardoublepage

%-------------------------------------------------------------------------
%	THESIS MAIN TEXT
%-------------------------------------------------------------------------
% In the main text of your thesis you can write the chapters in two different ways:
%
%(1) As presented in this template you can write:
%    \chapter{Title of the chapter}
%    *body of the chapter*
%
%(2) You can write your chapter in a separated .tex file and then include it in the main file with the following command:
%    \chapter{Title of the chapter}
%    \input{chapter_file.tex}
%
% Especially for long thesis, we recommend you the second option.

\addtocontents{toc}{\vspace{2em}} % Add a gap in the Contents, for aesthetics
\mainmatter % Begin numeric (1,2,3...) page numbering

% --------------------------------------------------------------------------
% NUMBERED CHAPTERS % Regular chapters following
% --------------------------------------------------------------------------
\chapter*{Introduction}

This document is intended to be both an example of the Polimi \LaTeX{} template for Master Theses,
as well as a short introduction to its use. It is not intended to be a general introduction to \LaTeX{} itself,
and the reader is assumed to be familiar with the basics of creating and compiling \LaTeX{} documents. 
\\
The cover page of the thesis must contain all the relevant information:
title of the thesis, name of the Study Programme and School, name of the author,
student ID number, name of the supervisor, name(s) of the co-supervisor(s) (if any), academic year.
The above information are provided by filling all the entries in the command \verb|\puttitle{}|
in the title page section of this template.
\\
Be sure to select a title that is meaningful.
It should contain important keywords to be identified by indexer.
Keep the title as concise as possible and comprehensible even to people who are not experts in your field.
The title has to be chosen at the end of your work so that it accurately captures the main subject of the manuscript. 
\\
Since a thesis might be a substantial document, it is convenient to break it into chapters.
You can create a new chapter as done in this template by simply using the following command
\begin{verbatim}
\chapter{Title of the chapter}
\end{verbatim}
followed by the body text.
\\
Especially for long manuscripts, it is recommended to give each chapter its own file.
In this case, you write your chapter in a separated \verb|chapter_n.tex| file
and then include it in the main file with the following command
\begin{verbatim}
\input{chapter_n.tex}
\end{verbatim}
It is recommended to give a label to each chapter by using the command
\begin{verbatim}
\label{ch:chapter_name}%
\end{verbatim}
where the argument is just a text string that you'll use to reference that part
as follows: \textit{Chapter~\ref{ch:chapter_one} contains \sc{an introduction to}  \dots}.\\
If necessary, an unnumbered chapter can be created by
\begin{verbatim}
\chapter*{Title of the unnumbered chapter}
\end{verbatim}






























\chapter{Toric code}
\label{ch:chapter_one}%
% The \label{...}% enables to remove the small indentation that is generated, always leave the % symbol.


\section{Spin Observables in Quantum Mechanics}
\label{sec:Observables}

In the context of quantum mechanics, an observable is a quantity that can be observed in a quantum system. Some examples could be: position, momentum or energy of a system. The latter is represented by the Hamiltonian, which not only describes the energy but also encodes the rules that govern the dynamic of the system. 
Formally, an observable is over an Hilbert space $\mathscr{H}$ is represented by an Hermitian operator over $\mathscr{H}$.

\begin{definition} (Hilbert space)
	An Hilbert space $\mathscr{H}$ over $\mathbb{C}$ is a vector space on $\mathbb{C}$:  
	\begin{enumerate}
		\item over which is defined the scalar product $\langle \cdot \vert \cdot \rangle $: $\mathbb{C} \times \mathbb{C} \rightarrow \mathbb{C}$;
		\item which is complete with respect to the norm induced by the scalar product.
	\end{enumerate}
\end{definition}

In particular, a 'spin' observable is a physical quantity associated with the intrinsic angular momentum of elementary particles, such as electrons. The spin is a fundamental property of particles and is characterized by a quantum number typically denoted as $S$, representing the spin quantum number.
In quantum mechanics, spin observables represent the possible measurements that can be made on the spin of a particle. Moreover, the spin is described using spin operators, which are, as anticipated above, hermitian matrices representing the components of the angular momentum of the spin along different directions. 



\subsection{Hermiticity and unitarity}

The hermiticity of an operator can be stated as follows:

\begin{definition} (Hermitian operator) Let $\mathscr{H}$ be an Hilbert space and  $A$: $\mathscr{H} \rightarrow \mathscr{H}$ be a bounded linear operator. An operator $A$ is called Hermitian if for all $\psi,\phi \in \mathscr{H}$ we have that
\end{definition}

\begin{center}
	$\langle A\psi|\phi \rangle = \langle \psi|A^{\dagger}\phi \rangle = \langle \phi|A\psi \rangle^{*}$.
\end{center}

The daga symbol $\dagger$ represents the conjugate transpose and the $\langle \ \rangle$ the Hermitian inner product. A square matrix $A$ of complex numbers, representing a linear operator, is called hermitian if $A^{\dagger} = (A^*)^T = A$. One important consequences of hermiticity is that it ensures that the spectrum of the operator is real. Notice also that, for real matrices, it is sufficient to compute only the transpose of the matrix to verify hermiticity. \newline
%Furthermore, as mentioned above, vertex and plaquette operators satisfy the involutory property, which is derived from the unitarity of the matrices representing the vertex and plaquette operators, i.e. $X$ and $Z$ Pauli matrices. \newline 

In order to understand the dynamics of observables in time we introduce anotehr important property: unitarity. 
As previously said, observables such as position, momentum and energy are represented by hermitian operators, which ensure the existence of real eigenvalues and orthogonal eigenvectors. Though, unitarity governs the evolution in time of quantum states and operators. In particular, it ensures that the total probability of all possible outcomes of a measurement remains conserved over time, preserving the fundamental principles of quantum mechanics. 

\begin{definition} (Unitary operator) Let $\mathscr{H}$ be an Hilbert space and let $U$: $\mathscr{H} \rightarrow U$ be a bounded linear operator. We define $U$ to be complex unitary if for all $\psi,\phi \in \mathscr{H}$ we have that
\end{definition}

\begin{center}
	$\langle U\psi|U\phi \rangle = \langle \psi|\phi \rangle$.
\end{center}

In terms of the conjugate transpose, we can define a complex matrix $U$ to be unitary if $ (U^*)^T=U^{-1}$. We can also define that a real matrix $U$, representing a linear operator, is unitary if $(U)^T = U^{-1}$ or, equivalently, if $(U)^T U=I$. 
%In our case this directly translates to $U^2=1$ beacause for $X$ and $Z$ Pauli matrices we have that $U^T=U$. Note also that one important property of  unitary operators is that their eigenvalues have modulus equal to one.\newline

Overall, hermiticity and unitarity provide a comprehensive framework for understanding the behavior of observables in quantum mechanics.


\subsection{The spectrum}

In the framework of quantum mechanics, the concept of the spectrum emerges as a natural consequence of the unitarity and hermiticity of observables. The spectrum of an observable refers to the set of possible values that can be obtained when measuring the physical quantity represented by the observable. 


%The trivial spectrum of vertices and plaquette operators will be of fundamental importance to perform fault-tolerant quantum compuations, in particular for measuring whether any error has occurred during computation (see Chapter 3). 


\subsection{Commutation and anticommutation}

Given two observables defined as bounded linear and hermitian operators $A$ and $B$, we define the commutator as follows

\begin{center}
	$[A, B] = AB - BA$.
\end{center}

The two observables commute if

\begin{center}
	$AB = BA$.
\end{center}

On the otehr hand, they anticommute if 

\begin{center}
	$AB = - BA$. 
\end{center}


\subsection{Spin-$\frac{1}{2}$ particles example}

In order to exemplify the above properties of observables, we take spin-$\frac{1}{2}$ particles {as discussed in \cite{Cor23}}, i.e. electrons, and define the components of the spin in three dimensions. In this case the Hilbert space is bidimensional and the observables can be written in terms of Pauli matrices in the basis $\{ | \frac{1}{2}, \pm \frac{1}{2} \rangle \}$ (spin-up and spin-down) for $\hbar=1$:


\[
\begin{array}{ccc}
	\text{$S_3$} = 
	\begin{pmatrix}
		1 & 0 \\
		0 & -1
	\end{pmatrix} &
	\text{$S_1$} = 
	\begin{pmatrix}
		0 & 1 \\
		1 & 0
	\end{pmatrix} &
	\text{$S_2$} = 
	\begin{pmatrix}
		0 & -i \\
		i & 0
	\end{pmatrix}
\end{array}
\]

Pauli matrices satisfy hermiticity and unitarity Moreover they satisfy some specific commuation and anticommutation properties.

Commutation properties:

\begin{center}
	$[\sigma_i, \sigma_j] = \sigma_i \sigma_j - \sigma_j \sigma_i 
	= 2i\epsilon_{ijk} \sigma_k$
\end{center}

where $\epsilon_{ijk}$ is the Levi-Civita symbol. \newline

Anticommutation properties:

\begin{center}
	$\{\sigma_i, \sigma_j\} = \sigma_i \sigma_j + \sigma_j \sigma_i 
	= 2 \delta_{ij}\mathbb{I}$
\end{center}

where $ \delta_{ij} $ is the Kronecker delta and $\mathbb{I}$ is the identity matrix.


\subsection{Tensor product of Hilbert spaces}

Given two observables $A$ and $B$ that belong to a bidimensional Hilbert space we compute their tensor product $A \otimes B$ over $\mathbb{C}^2 \otimes \mathbb{C}^2$ given their respective basis.


\[
\begin{array}{ccc}
	\begin{pmatrix}
		u_1  \\
		u_2  
	\end{pmatrix} 
	\otimes
	\begin{pmatrix}
		w_1  \\
		w_2 
	\end{pmatrix} &
\end{array}
\]

Overall, we would obtain four combinations:

\[
\begin{array}{ccc}
	\begin{pmatrix}
		1 \\
		0  
	\end{pmatrix} 
	\otimes
	\begin{pmatrix}
		1  \\
		0 
	\end{pmatrix} ,
	
	\begin{pmatrix}
		1 \\
		0  
	\end{pmatrix} 
	\otimes
	\begin{pmatrix}
		0  \\
		1 
	\end{pmatrix} ,
	
	\begin{pmatrix}
		0 \\
		1  
	\end{pmatrix} 
	\otimes
	\begin{pmatrix}
		0  \\
		1 
	\end{pmatrix} ,
	
	\begin{pmatrix}
		0 \\
		1  
	\end{pmatrix} 
	\otimes
	\begin{pmatrix}
		1  \\
		0 
	\end{pmatrix} &
\end{array}
\]


Though, considering that $A$ acts only an the first Hilbert space $A \otimes \mathbb{I}$ and that $B$ acts only on the second $\mathbb{I} \otimes B$, we can rename:

\begin{center}
	$A \otimes \mathbb{I}$=$A_1$ \\
	$\mathbb{I} \otimes B$=$B_2$
\end{center}

and write the tensor product as a simple product of $A_1$ and $B_2$:

\begin{center}
	$A \otimes B$=$A_1$$B_2$.
\end{center}

This is because the operators acts locally, so we can ignore the identity. \newline
Similarly if we compute the tensor product explicitly, it becomes a simple product of the scalar products between the components of the basis:

\begin{center}
	$(u_1 \otimes w_1) \dot (u_2 \otimes w_2) = \langle u_1 | w_1 \rangle \langle u_2 | w_2 \rangle$
\end{center}







%------------------------------------

\newpage
\section{Description of the model}
\label{sec:Model}

The toric code model is defined on a square lattice with periodic boundary conditions in both directions. These latter characteristics are typical of what is known as a torus topology or simply a torus, after which the model takes name.\newline
A square lattice, here labelled as $\mathcal{L}$, is a particular lattice defined in a two dimensional space. It is denoted as $\mathbb{Z}^{2}$ such that each lattice point is identified with a pair of integers. Though, the above mentioned boundary conditions also specify that for any site $(i, j)$ in the lattice, the neighboring sites are going to be: $((i+1)\mathrm{mod}N, j)$, $((i-1)\mathrm{mod} N, j)$, $(i, (j+1)\mathrm{mod}N)$ and $(i, (j-1)\mathrm{mod}N)$, where $N$ is the dimension of the torus.  \newline
Then, we are also going to consider a dual lattice, which will be labelled as $\mathcal{L}'$, and that will be positioned as represented in figure 1.1, where the continuous line represents the main lattice $\mathcal{L}$, while the dashed line represents its dual $\mathcal{L'}$.  	\newline

On each edge is located a spin-$\frac{1}{2}$ particle, i.e. a Fermion, represented in the image below with an empty circle. Circles shaded in gray represent the boundary conditions. 
For each cell on $\mathcal{L}$ we are going to consider two spins, therefore the total number of spins will correspond to $2N^2$, where $N^2$ represents both the total number of cells in $\mathcal{L}$ and $N$ the dimension of the lattice. 
Since all of these spins exhibit the same characteristics they are identified with identical particles, which will be useful to know in the next sections in order to study the behaviour of the system.

%\afterpage{
\begin{figure}[b]
	\begin{center}
		\begin{tikzpicture}
			% Draw dashed lines
			\foreach \i in {-3,-2.5,...,3}
			{
				\draw[dashed] (\i,-3) -- (\i,3);
			}
			\foreach \j in {-3,-2.5,...,3}
			{
				\draw[dashed] (-3,\j) -- (3,\j);
			}
			
			% Draw solid grid and nodes with circles in the middle of each side
			\draw[step=1cm] (-3,-3) grid (3,3);
			\foreach \i in {-2.5,...,2.5}
			{
				\foreach \j in {-2.5,...,2.5}
				{
					\begin{scope}[transform canvas={xshift=\i cm,yshift=\j cm}]
						\node[right,xshift=0.2cm,yshift=0.4cm] {};
						% Convert \j and \i to integers
						\pgfmathtruncatemacro{\intj}{\j}
						\pgfmathtruncatemacro{\inti}{\i}
						
						% Draw circles at the midpoints of each side
						\ifnum\intj=2
						\draw node[draw,circle,fill=gray] at (0,0.5) {};
						\else
						\draw node[draw,circle,fill=white] at (0,0.5) {};
						\fi
						
						\ifnum\inti=2
						\draw node[draw,circle,fill=gray] at (0.5,0) {};
						\else
						\draw node[draw,circle,fill=white] at (0.5,0) {};
						\fi
						
						\draw node[draw,circle,fill=white] at (0,-0.5) {};
						\draw node[draw,circle,fill=white] at (-0.5,0) {};
					\end{scope}
				}
			}
		\end{tikzpicture}
	\end{center}
	
	\caption{Square lattice with boundary conditions.}
	\label{fig:lattice}
\end{figure}
%}

\newpage
The key part of the Toric Code are the so called \textit{vertex} and \textit{plaquette} operators that are going to be placed, respectively, on the vertices $v$ and cells $p$ of $\mathcal{L}$. Such operators can be defined formally as in definition 1.6 and implemented by means of Pauli matrices as exemplified in figure 1.2. 


\begin{figure}
	\begin{center}
		\begin{tikzpicture}
			% Draw dashed lines
			\foreach \i in {-3,-2.5,...,3}
			{
				\draw[dashed] (\i,-3) -- (\i,3);
			}
			\foreach \j in {-3,-2.5,...,3}
			{
				\draw[dashed] (-3,\j) -- (3,\j);
			}
			
			
			
			% Draw solid grid and nodes with circles in the middle of each side
			\draw[step=1cm] (-3,-3) grid (3,3);
			\foreach \i in {-2.5,...,2.5}
			{
				\foreach \j in {-2.5,...,2.5}
				{
					
					
					\begin{scope}[transform canvas={xshift=\i cm,yshift=\j cm}]
						\node[right,xshift=0.2cm,yshift=0.4cm] {};
						% Convert \j and \i to integers
						\pgfmathtruncatemacro{\intj}{\j}
						\pgfmathtruncatemacro{\inti}{\i}
						
						% Draw circles at the midpoints of each side
						\ifnum\intj=2
						\draw node[draw,circle,fill=gray] at (0,0.5) {};
						\else
						\draw node[draw,circle,fill=white] at (0,0.5) {};
						\fi
						
						\ifnum\inti=2
						\draw node[draw,circle,fill=gray] at (0.5,0) {};
						\else
						\draw node[draw,circle,fill=white] at (0.5,0) {};
						\fi
						
						\draw node[draw,circle,fill=white] at (0,-0.5) {};
						\draw node[draw,circle,fill=white] at (-0.5,0) {};
					\end{scope}
				}
			}
			
			\foreach \i in {-1,...,-1} %column
			{
				
				\draw[blue!50, line width=1.5mm] (\i,0.5) -- (\i,1.5);
				\node[draw, circle, fill=blue!50,label=center:\textbf{Z}] at (\i,0.5) {};
				\node[draw, circle, fill=blue!50,label=center:\textbf{Z}] at (\i,1.5) {};
				
			}
			\foreach \j in {1,...,1}
			{
				
				\draw[blue!50, line width=1.5mm] (-1.5, \j) -- (-0.5, \j);
				\draw node[draw,circle,fill=blue!50,label=center:\textbf{Z},,label=\textbf{Av}] at (-1.5,\j) {};
				\draw node[draw,circle,fill=blue!50,label=center:\textbf{Z}] at (-0.5,\j) {};
				
			}
			
			
			\foreach \i in {2,...,2}
			{
				
				\draw[red!70, line width=1.5mm] (\i,-1) -- (\i,0);
				\node[draw, circle, fill=red!70,label=center:\textbf{X}] at (\i,-0.5) {};
				
			}
			\foreach \j in {-1,...,-1}
			{
				
				\draw[red!70, line width=1.5mm] (2, \j) -- (1, \j);
				\draw node[draw,circle,fill=red!70,label=center:\textbf{X}] at (1.5,\j) {};
				
			}
			\foreach \i in {1,...,1}
			{
				
				\draw[red!70, line width=1.5mm] (\i,-1) -- (\i,0);
				\node[draw, circle, fill=red!70,label=center:\textbf{X},label=left:\textbf{Bp}] at (\i,-0.5) {};
				
			}
			\foreach \j in {0,...,0}
			{
				
				\draw[red!70, line width=1.5mm] (2, \j) -- (1, \j);
				\draw node[draw,circle,fill=red!70,label=center:\textbf{X}] at (1.5,\j) {};
				
			}
			
		\end{tikzpicture}
	\end{center}
	
	\vspace*{1cm}
	
	\begin{center}
		\begin{tikzpicture}
			% Draw dashed lines
			\foreach \i in {-3,-2.5,...,3}
			{
				\draw[dashed] (\i,-3) -- (\i,3);
			}
			\foreach \j in {-3,-2.5,...,3}
			{
				\draw[dashed] (-3,\j) -- (3,\j);
			}
			
			
			
			% Draw solid grid and nodes with circles in the middle of each side
			\draw[step=1cm] (-3,-3) grid (3,3);
			\foreach \i in {-2.5,...,2.5}
			{
				\foreach \j in {-2.5,...,2.5}
				{
					
					
					\begin{scope}[transform canvas={xshift=\i cm,yshift=\j cm}]
						\node[right,xshift=0.2cm,yshift=0.4cm] {};
						% Convert \j and \i to integers
						\pgfmathtruncatemacro{\intj}{\j}
						\pgfmathtruncatemacro{\inti}{\i}
						
						% Draw circles at the midpoints of each side
						\ifnum\intj=2
						\draw node[draw,circle,fill=gray] at (0,0.5) {};
						\else
						\draw node[draw,circle,fill=white] at (0,0.5) {};
						\fi
						
						\ifnum\inti=2
						\draw node[draw,circle,fill=gray] at (0.5,0) {};
						\else
						\draw node[draw,circle,fill=white] at (0.5,0) {};
						\fi
						
						\draw node[draw,circle,fill=white] at (0,-0.5) {};
						\draw node[draw,circle,fill=white] at (-0.5,0) {};
					\end{scope}
				}
			}
			
			\foreach \i in {-1,...,-1} %column
			{
				
				\draw[blue!50, line width=1.5mm] (\i,0.5) -- (\i,1.5);
				\node[draw, circle, fill=blue!50,label=center:\textbf{Z}] at (\i,0.5) {};
				\node[draw, circle, fill=blue!50,label=center:\textbf{Z}] at (\i,1.5) {};
				
			}
			\foreach \j in {1,...,1}
			{
				
				\draw[blue!50, line width=1.5mm] (-1.5, \j) -- (-0.5, \j);
				\draw node[draw,circle,fill=blue!50,label=center:\textbf{Z},label=\textbf{Av}] at (-1.5,\j) {};
				\draw node[draw,circle,fill=blue!50,label=center:\textbf{Z} ] at (-0.5,\j) {};
				
			}
			
			
			
			
			\foreach \i in {1.5,...,1.5}
			{
				
				\draw[red!70, line width=1.5mm] (\i,-1) -- (\i,0);
				\node[draw, circle, fill=red!70,label=center:\textbf{X},label=left:\textbf{Bp}] at (\i,0) {};
				\node[draw, circle,fill=red!70,label=center:\textbf{X}] at (\i,-1) {};
				
			}
			\foreach \j in {-0.5,...,-0.5}
			{
				
				\draw[red!70, line width=1.5mm] (2, \j) -- (1, \j);
				\draw node[draw,circle,fill=red!70,label=center:\textbf{X}] at (1,\j) {};
				\draw node[draw,circle,fill=red!70,label=center:\textbf{X} ] at (2,\j) {};
				
			}
			
			
			
		\end{tikzpicture}
	\end{center}
	
	\caption{In the first picture: vertex (blue) and plaquette (red) operators applied, respectively, on a vertex $v$ and cell $p$ of $\mathcal{L}$. In the second picture: vertex (blue) and plaquette (red) operators applied, respectively, on a vertex $v$ of $\mathcal{L}$ and a vertex $v'$ of $\mathcal{L'}$.}
	\label{fig:operators}
\end{figure}

\begin{definition} (Groups of neighbouring spins of $\mathcal{L}$) We define the quartets of spins sorrounding a vertex $v$ and a plaquette $p$ of $\mathcal{L}$, respectively, as the following groups of indeces: $j \in \mathrm{star(v)}$ and $j \in \mathrm{bdy(p)}$.
\end{definition}

%\begin{definition} (Groups of neighbouring spins of $\mathcal{L'}$) We define the quartets of spins sorrounding a vertex $v$ and $v'$ of $\mathcal{L'}$, respectively, as the following groups of indeces: $j \in \mathrm{star(v)}$ and $j \in \mathrm{star(v')}$.
%\end{definition}

\begin{definition} (Vertex and Plaquette operators) Given a vertex $v$ and a plaquette $p$ we can define, respectively, vertex and plaquette operators as products over Pauli operators, each acting locally on four neighbouring spins of $\mathcal{L}$
	
\begin{center}
	$ A_v = \prod_{j \in star(v)} Z_j $ 
	
	$ B_p = \prod_{j \in bdy(v)} X_j $.
\end{center}

We can equivalently define $B_p$ on a vertex $v'$ pf the dual lattice $\mathcal{L'}$ as shown in figure 1.2

\begin{center}
	$ B_p = \prod_{j \in star(v')} X_j $.
\end{center}

\end{definition}

Where we have defined each $Z_i$ and $X_i$ to be equal to the following tensor proucts:

\begin{center}
	
	$Z_i = \mathbb{I}_{\mathcal{L} \setminus  e_i} \otimes \sigma^z_{e_i}$ 
	
	$X_i = \mathbb{I}_{\mathcal{L} \setminus  e_i} \otimes \sigma^x_{e_i} $ .
	
\end{center}


For the sake of simplicity, we can omit the tensor product because there is no risk of ambiguity in the interpretation. In fact, since the vertex and plaquette operators act locally, we can omit writing down the tensor product with the identity for sites that are not part of the support of the operator. 
%Thus, we can simply rewrite them by means of the product of the Pauli matrices that act on the quartets of spins indexed as $j \in \mathrm{star(v)}$ and $j \in \mathrm{bdy(p)}$. 
 



All the elements that constitute the model of the toric code are brought together by its Hamiltonian. Definition 1.2 describes the Hamiltonian of the toric code 
{according to \cite{Kit02}}. For recent reviews on the subject see \cite{Her20}.

\begin{definition} (Toric Code Hamiltonian) Given vertex and plaquette operators each acting, respectively, on verteces and cells of the torus, we can define the following Hamiltonian for the system:
\end{definition}

\begin{center}
	
	$H = -\sum_{v} 
	A_v - \sum_{p} B_p $
	
\end{center}

Each one of the $A_v$ and $B_p$ operators that appear in the Hamiltonian share some important properties. Firstly, notice that $A_v$ and $B_p$ operators are Hermitian and square to the identity. Given the considerations made in definition 1.2 and 1.3 for real matrices, we can easily prove the first following proposition:

\begin{proposition} (Hermiticity and involutory property of $Av$ and $Bp$ operators)
$Av$ and $Bp$ operators are Hermitian and satisfy the involutory property.
\end{proposition}
	
\textit{Proof.}
We know that the operator $ A_v = \prod_{j \in star(v)} Z_j $ and that $ B_p = \prod_{j \in bdy(v)} X_j $ .\newline
Firstly, recall the form of the $\sigma_x$ and $\sigma_z$ matrices representing the Pauli gates $X$ and $Z$:



\[
\text{Z = $\sigma_z$} =
\begin{pmatrix}
	1 & 0 \\
	0 & -1
\end{pmatrix}
\]


\[
\text{X = $\sigma_x$} =
\begin{pmatrix}
	0 & 1 \\
	1 & 0
\end{pmatrix}
\]


It can be proved that they are Hermitian:\newline

\[
\text{$( \sigma_z )^{\dagger}$} = 
\begin{pmatrix}
	1 & 0 \\
	0 & -1
\end{pmatrix} ^\dagger =
\begin{pmatrix}
	1 & 0 \\
	0 & -1
\end{pmatrix}^T =
\begin{pmatrix}
	1 & 0 \\
	0 & -1
\end{pmatrix}
= \text{$ \sigma_z $}
\]


\[
\text{$( \sigma_x )^{\dagger}$} = 
\begin{pmatrix}
	0 & 1 \\
	1 & 0
\end{pmatrix} ^\dagger =
\begin{pmatrix}
	0 & 1 \\
	1 & 0
\end{pmatrix}^T =
\begin{pmatrix}
	0 & 1 \\
	1 & 0
\end{pmatrix}
= \text{$ \sigma_x $}
\]\newline

since Pauli matrices are real matrices and thus $A^{\dagger}=A^T$.
%Then, knowing that $[\sigma_i,\sigma_j]=0 \ for \ i=j$ we can write:

\begin{center}
	$(A_v)^{\dagger} = (\sigma_{x_1} \sigma_{x_2} \sigma_{x_3} \sigma_{x_4})^{\dagger} = \sigma_{x_1}^{\dagger} \sigma_{x_2}^{\dagger} \sigma_{x_3}^{\dagger} \sigma_{x_4}^{\dagger} = \sigma_{x_1} \sigma_{x_2} \sigma_{x_3} \sigma_{x_4} = Av$ \newline
	
	$(B_p)^{\dagger} = (\sigma_{z_1} \sigma_{z_2} \sigma_{z_3} \sigma_{z_4})^{\dagger} = \sigma_{z_1}^{\dagger} \sigma_{z_2}^{\dagger} \sigma_{z_3}^{\dagger} \sigma_{z_4}^{\dagger} = \sigma_{z_1} \sigma_{z_2} \sigma_{z_3} \sigma_{z_4} = Bp$ \newline
\end{center}

It can also be proved that they satisfy the involutory property:\newline

\[
\text{$( \sigma_z )^{2}$} = 
\begin{pmatrix}
	1 & 0 \\
	0 & -1
\end{pmatrix} \cdot
\begin{pmatrix}
	1 & 0 \\
	0 & -1
\end{pmatrix} =
\begin{pmatrix}
	1 & 0 \\
	0 & 1
\end{pmatrix}
= \text{$I$}
\]


\[
\text{$( \sigma_x )^{2}$} = 
\begin{pmatrix}
	0 & 1 \\
	1 & 0
\end{pmatrix} \cdot
\begin{pmatrix}
	0 & 1 \\
	1 & 0
\end{pmatrix} =
\begin{pmatrix}
	1 & 0 \\
	0 & 1
\end{pmatrix}
= \text{$I$}
\]\newline


Then, again we can write \newline

\begin{center}
	$(A_v)^{2} = (\sigma_{x_1} \sigma_{x_2} \sigma_{x_3} \sigma_{x_4})^{2} = \sigma_{x_1}^2 \sigma_{x_2}^2 \sigma_{x_3}^2 \sigma_{x_4}^2 = I$ 
	
	$(B_p)^{2} = (\sigma_{z_1} \sigma_{z_2} \sigma_{z_3} \sigma_{z_4})^{2} = \sigma_{z_1}^2 \sigma_{z_2}^2 \sigma_{z_3}^2 \sigma_{z_4}^2 = I$
\end{center}

\hfill $\square$

Given proposition 1.1, we can know further characterize the opeartors in terms of their eigenvalues and define their spectrum.

\begin{proposition} (Eigenvalues of $A_v$ and $B_p$ operators) Given Hermitian operators $A_v$ and $B_p$, which satisfy the involutory property, their possible eigenvalues are $\pm 1$.
\end{proposition}


\textit{Proof.}\newline
Given the Hermiticity and involutory proof in 1.1, we firstly prove that vertex and plaquette operators have real eigenvalues: 

write the expression for the eigenvalues $A_v |\xi \rangle = \lambda |\xi \rangle$ and take as hypothesis that $|\xi \rangle \neq 0$. Then, by means of the scalar product:

\begin{center}
	$\langle \xi|A_v|\xi \rangle = \lambda \langle \xi | \xi \rangle$
	
	$\lambda = \frac {\langle \xi|A_v|\xi \rangle}{\langle \xi |\xi \rangle}$ = $\frac {\langle \xi|A_v|\xi \rangle}{||\xi||^2}$ = $\frac {\langle \xi|A_v^{\dagger}|\xi \rangle}{||\xi||^2}$ = $\frac {\langle \xi|A_v|\xi\rangle^*}{||\xi||^2}$ = $\lambda^*$ 
\end{center}

Notice that we have applied antilinearity of the adjoint in the penultimate equality: 

\begin{center}
	$(\langle \xi|A_v^{\dagger}|\xi \rangle)^{\dagger}$ = $|\xi\rangle^{\dagger} (A_v^{\dagger})^{\dagger} \langle\xi|^{\dagger}$ = $\langle \xi|A_v|\xi \rangle^*$.  
\end{center}

Therefore, we have derived that the eigenvalues of $A_v$ are $\lambda = \lambda^*$, which implies that they must be real.

%Using hermiticity with the fact that $(Av)^2=I$ we can derive the unitarity of $Av$ and state that $Av Av^{\dagger} = Av^{\dagger} Av = (Av)^2 = I$. \newline
%Given the property of unitarity, which states that the corresponding eigenvalues of the operator have modulus equal to one,:\newline 

Then we exploit unitarity while taking as hypothesis that $|\xi \rangle \neq 0$. 
We write the expression $A_v |\xi \rangle = \lambda |\xi \rangle$ and its self-adjoint $\langle \xi| A_v^{\dagger} = \lambda^* \langle \xi|$. Recall that $A_v^{\dagger}=A_v^{-1}$, thus by means of the scalar product we obtain 

\begin{center}
	$\langle \xi|A_v A_v^{\dagger}|\xi\rangle = \lambda \lambda^* \langle \xi |\xi \rangle$
	
	$\langle \xi|I|\xi \rangle = \lambda \lambda^* \langle \xi |\xi \rangle$
	
	$\langle \xi|\xi \rangle = \lambda \lambda^* \langle \xi |\xi \rangle $
	
	$ \lambda \lambda^* = 1 $
	
	$ |\lambda|^2 = 1 $
\end{center}

Which means that the eigenvalues of $A_v$ have modulus equal to one, as expected. Though, this can happen for two real choices of $\lambda$ which are $+1$ and $-1$.
Therefore, putting together hermiticity and unitarity, 

\begin{center}
$\begin{cases}
	 \lambda = \lambda^* \\
	 |\lambda|^2 = 1 
\end{cases}$
\end{center}

we obtain that the only two remaining possibilities for the eigenvalues of $A_v$ are $\pm 1$. \newline

The same reasoning can be carried out for the plaquette operators.

\hfill $\square$ 

Given the result of proposition 1.2, is easy to find the spectrum of the operators: 

\begin{proposition} (Spectrum of $A_v$ and $B_p$ operators) The spectrum of $A_v$ and $B_p$ operators is equal to $\{-1,+1\}$. 
\end{proposition}

In particular, if we compute the corresponding eigenvalue of $A_v$ (or $B_p$), this can either assume value $-1$ or $+1$ according to the fact that the Pauli matrices that make up the operator anticommute, respectively, with an odd or even number of edges.
In fact we can prove that following relationships between vertex and plaquette operators holds:

\begin{proposition} (Commutation of $A_v$ and $B_p$ operators) The operator $A_v$ commutes with the operator $B_p$ for an even number of overlapping edges.
\end{proposition}


\begin{proposition} (Anticommutation of $A_v$ and $B_p$ operators) The operator $A_v$ anticommutes with the operator $B_p$ for an odd number of overlapping edges.
\end{proposition}


\textit{Proof.}\newline 
%Av commutes with itself.\newline
%Bp commutes with itself.\newline
%Av commutes with Bp for an even number of edges.\newline
Fix the origin of the coordinate system in the bottom left corner of the lattice as shown in figure 1.3. 

\begin{figure}
\begin{center}
	\begin{tikzpicture}
		% Draw dashed lines
		\foreach \i in {-3,-2.5,...,3}
		{
			\draw[dashed] (\i,-3) -- (\i,3);
		}
		\foreach \j in {-3,-2.5,...,3}
		{
			\draw[dashed] (-3,\j) -- (3,\j);
		}
		
		
		
		% Draw solid grid and nodes with circles in the middle of each side
		\draw[step=1cm] (-3,-3) grid (3,3);
		\foreach \i in {-2.5,...,2.5}
		{
			\foreach \j in {-2.5,...,2.5}
			{
				
				
				\begin{scope}[transform canvas={xshift=\i cm,yshift=\j cm}]
					\node[right,xshift=0.2cm,yshift=0.4cm] {};
					% Convert \j and \i to integers
					\pgfmathtruncatemacro{\intj}{\j}
					\pgfmathtruncatemacro{\inti}{\i}
					
					% Draw circles at the midpoints of each side
					\ifnum\intj=2
					\draw node[draw,circle,fill=gray] at (0,0.5) {};
					\else
					\draw node[draw,circle,fill=white] at (0,0.5) {};
					\fi
					
					\ifnum\inti=2
					\draw node[draw,circle,fill=gray] at (0.5,0) {};
					\else
					\draw node[draw,circle,fill=white] at (0.5,0) {};
					\fi
					
					\draw node[draw,circle,fill=white] at (0,-0.5) {};
					\draw node[draw,circle,fill=white] at (-0.5,0) {};
				\end{scope}
			}
		}
		
		\foreach \i in {-1,...,-1} %column
		{
			
			\draw[blue!50, line width=1.5mm] (\i,0.5) -- (\i,1.5);
			\node[draw, circle, fill=blue!50,label=center:\textbf{Z}] at (\i,0.5) {};
			\node[draw, circle, fill=blue!50,label=center:\textbf{Z}] at (\i,1.5) {};
			
		}
		\foreach \j in {1,...,1}
		{
			
			\draw[blue!50, line width=1.5mm] (-1.5, \j) -- (-0.5, \j);
			\draw node[draw,circle,fill=blue!50,label=center:\textbf{Z},label=\textbf{Av}] at (-1.5,\j) {};
			\draw node[draw,circle,fill=blue!50,label=center:\textbf{Z} ] at (-0.5,\j) {};
			
		}
		
		
		
		
		\foreach \i in {1.5,...,1.5}
		{
			
			\draw[red!70, line width=1.5mm] (\i,-1) -- (\i,0);
			\node[draw, circle, fill=red!70,label=center:\textbf{X},label=left:\textbf{Bp}] at (\i,0) {};
			\node[draw, circle,fill=red!70,label=center:\textbf{X}] at (\i,-1) {};
			
		}
		\foreach \j in {-0.5,...,-0.5}
		{
			
			\draw[red!70, line width=1.5mm] (2, \j) -- (1, \j);
			\draw node[draw,circle,fill=red!70,label=center:\textbf{X}] at (1,\j) {};
			\draw node[draw,circle,fill=red!70,label=center:\textbf{X} ] at (2,\j) {};
			
		}
		
		% Define vertices
		\coordinate (A) at (-3,-3);
		\coordinate (B) at (-1,1);
		
		% Draw vector
		\draw[->, line width=1.1pt] (A) -- (B) node[right] {$\vec{v}$};
		
		
		% Define vertices
		\coordinate (A) at (-3,-3);
		\coordinate (B) at (1.5,-0.5);
		
		% Draw vector
		\draw[->, line width=1.1pt] (A) -- (B) node[right] {$\vec{p}$};
		
		
		
		
		
		
		% Define vertices
		\coordinate (A) at (-3,-3);
		\coordinate (B) at (-3,-2);
		
		% Draw vector
		\draw[->, line width=1.1pt] (A) -- (B) node[midway, left] {$\vec{e2}$};
		
		
		% Define vertices
		\coordinate (A) at (-3,-3);
		\coordinate (B) at (-2,-3);
		
		% Draw vector
		\draw[->, line width=1.1pt] (A) -- (B)  node[midway, below] {$\vec{e1}$};
		
		
		
	\end{tikzpicture}
\end{center}

\caption{Vertex and plaquette operators according to the coordinate system with origin in (0,0). Here the plaquette operator is implemented on $\mathcal{L}'$ in order to coherently define its edges from the point of view of $\mathcal{L}$ using vector $\vec{v}$.}
\label{fig:coordinate_system}
\end{figure}


Then, we can define the two vectors representing the site of application of the vertex and plaquette operator, respectively over the lattice $\mathcal{L}$ and dual lattice $\mathcal{L}'$ \newline

\begin{center}
	$\vec{v}$= $n\hat{e_1} + m\hat{e_2}$, where $n,m \in \mathbb{Z}$ 
	
	$\vec{p}$= $(n + \frac{1}{2}) \hat{e_1} + (m + \frac{1}{2}) \hat{e_2}$, where $n,m \in \mathbb{Z}$.
\end{center}

Rewrite the operators as follows: \newline

\begin{center}
	
	$A_{v} = \sigma^z_{\vec{v}+\frac{1}{2}\hat{e_1}} \sigma^z_{\vec{v}+\frac{1}{2}\hat{e_2}} \sigma^z_{\vec{v}-\frac{1}{2}\hat{e_1}} \sigma^z_{\vec{v}-\frac{1}{2}\hat{e_2}}$ 
	
	$B_{p} = \sigma^x_{\vec{p}+\frac{1}{2}\hat{e_1}} \sigma^x_{\vec{p}+\frac{1}{2}\hat{e_2}} \sigma^x_{\vec{p}-\frac{1}{2}\hat{e_1}} \sigma^x_{\vec{p}-\frac{1}{2}\hat{e_2}}$.
	
\end{center}

In order to simplify the calculations we rewrite $B_p$ on the lattice $\mathcal{L}$ by rewriting the indeces in terms of vector $\vec{v}$ \newline

\begin{center}
	$(n\hat{e_1} + m\hat{e_2}) + \frac{1}{2}\hat{e_2}= n\hat{e_1} + (m+\frac{1}{2}\hat{e_2})$
	
	$(n\hat{e_1} + m\hat{e_2}) + \frac{1}{2}\hat{e_1}= (n+ \frac{1}{2})\hat{e_1} + m\hat{e_2}$
	
	$(n\hat{e_1} + m\hat{e_2}) + \frac{1}{2}\hat{e_1}+\hat{e_2}= (n+ \frac{1}{2})\hat{e_1} + (m + 1)\hat{e_2}$
	
	$(n\hat{e_1} + m\hat{e_2}) + \frac{1}{2}\hat{e_2}+\hat{e_1}= (n+ 1)\hat{e_1} + (m + \frac{1}{2})\hat{e_2}$.
\end{center}


Then the $B_{p}$ operator becomes: \newline

\begin{center}
	
	$B_{p} = \sigma^x_{n\hat{e_1} + (m+\frac{1}{2}\hat{e_2})} \sigma^x_{(n+ \frac{1}{2})\hat{e_1} + m\hat{e_2}} \sigma^x_{(n+ \frac{1}{2})\hat{e_1} + (m + 1)\hat{e_2}} \sigma^x_{(n+ 1)\hat{e_1} + (m + \frac{1}{2})\hat{e_2}}$ \newline
	
\end{center}

and the Hamiltonian can be written by grouping the indices: \newline

\begin{center}
	
	$H = - \sum_{m,n \in \mathbb{Z}} \{ 
	\sigma^z_{(n+\frac{1}{2})\hat{e_1} + m\hat{e_2}} \sigma^z_{n\hat{e_1}+(m+\frac{1}{2})\hat{e_2}} \sigma^z_{(n-\frac{1}{2})\hat{e_1} + m\hat{e_2}} \sigma^z_{n\hat{e_1}+(m-\frac{1}{2})\hat{e_2}} +
	\sigma^x_{n\hat{e_1} + (m+\frac{1}{2}\hat{e_2})} \sigma^x_{(n+ \frac{1}{2})\hat{e_1} + m\hat{e_2}} \sigma^x_{(n+ \frac{1}{2})\hat{e_1} + (m + 1)\hat{e_2}} \sigma^x_{(n+ 1)\hat{e_1} + (m + \frac{1}{2})\hat{e_2}} \} $.\newline
	
\end{center}

Now calcuate the commutator $[A_{\vec{v}},B_{\vec{v}}] = A_{\vec{v}}B_{\vec{v}} - B_{p}A_{v}$ by focusing on the first term: \newline

\begin{center}
	
	$ A_{v}B_{p} =
	\sigma^z_{(n+\frac{1}{2})\hat{e_1} + m\hat{e_2}} \sigma^z_{n\hat{e_1}+(m+\frac{1}{2})\hat{e_2}} \sigma^z_{(n-\frac{1}{2})\hat{e_1} + m\hat{e_2}} \sigma^z_{n\hat{e_1}+(m-\frac{1}{2})\hat{e_2}} $ $\cdot$
	
	$\sigma^x_{n\hat{e_1} + (m+\frac{1}{2}\hat{e_2})} \sigma^x_{(n+ \frac{1}{2})\hat{e_1} + m\hat{e_2}} \sigma^x_{(n+ \frac{1}{2})\hat{e_1} + (m + 1)\hat{e_2}} \sigma^x_{(n+ 1)\hat{e_1} + (m + \frac{1}{2})\hat{e_2}}$. \newline
	
\end{center}


Matrices do not always commute but for Pauli matrices we have the following commutation relationship :
\newline

\begin{center}
	$\sigma^x_{\vec{v}}\sigma^z_{\vec{v}'} = \sigma^z_{\vec{v}'} \sigma^x_{\vec{v}} + 2 \sigma^x_{\vec{v}}\sigma^z_{\vec{v}'} \delta_{\vec{v} \vec{v}'}$\newline
\end{center}

where $\hspace{1cm} \delta_{\vec{v} \vec{v}'} =$
$\begin{cases}
	1, \hspace{1cm} if \hspace{1cm}  \vec{v} = \vec{v}'\\
	0, \hspace{1cm} if \hspace{1cm} \vec{v} \neq \vec{v}'
\end{cases}$\newline

which states that 	$\sigma^x_{\vec{v}}\sigma^z_{\vec{v}'}$ commutes for $\vec{v} \neq \vec{v}'$   but anticommutes for $\vec{v} = \vec{v}'$. This is known from the anticommutation relationship of Pauli matrices $\sigma^x \sigma^z = - \sigma^x \sigma^z$. Thus, for an even numer of overlapping edges, in our case 2 or 4, the commutator becomes:\newline

\begin{center}
	
	$[A_{v},B_{p}] = 2
	\sigma^x_{n\hat{e_1} + (m+\frac{1}{2}\hat{e_2})} \sigma^x_{(n+ \frac{1}{2})\hat{e_1} + m\hat{e_2}} \sigma^x_{(n+ \frac{1}{2})\hat{e_1} + (m + 1)\hat{e_2}} \sigma^x_{(n+ 1)\hat{e_1} + (m + \frac{1}{2})\hat{e_2}} \cdot
	\sigma^z_{(n+\frac{1}{2})\hat{e_1} + m\hat{e_2}} \sigma^z_{n\hat{e_1}+(m+\frac{1}{2})\hat{e_2}} \sigma^z_{(n-\frac{1}{2})\hat{e_1} + m\hat{e_2}} \sigma^z_{n\hat{e_1}+(m-\frac{1}{2})\hat{e_2}} $. \newline

\end{center}
Instead, for an odd number of edges the commutator vanishes: $[A_{v},B_{p}]=0$.

These calculations conclude that $A_{v},B_{p}$ commute for an even numer of edges but anticommute for an odd number of edges.\newline

\hfill $\square$\newline

In the same way, we notice that two vertex/plaquette operators applied on different sites always commute; this is because they either do not share any edge or they share only one.\newline

Now that we have described the toric code model and its operators, we proceed to compute its ground state, which is at the heart of its fault-tolerant behaviour. The ground state arises from the interaction of vertex and plaquette operators acting on the system's lattice. Through the characterization of these operators and the understanding of their role in enforcing some additional constraints, we gain insight into the structure of the ground state manifold. \newline
In the following section, we derive the unique features of the ground state, including its topological order. \newline 

%--------------



















\newpage
\section{Ground states}
\label{sec:GS}

The ground state of the toric code represents the system's lowest energy configuration. What makes this ground state particularly intriguing is its highly degenerate nature, meaning that there are numerous distinct configurations that all share the same minimum energy. 
The ground state exhibit topological order, characterized by non-local correlations between spins. This means that the model is robust  even when the system undergoes local perturbations.  In fact, quantum entanglement between spins is not confined to neighboring pairs, instead, it extends over long distances and involves collective behaviours of spins that cannot be understood by considering them individually.
Not only, as we will see, these features are intimately related to the underlying topology of the torus. This topological protection ensures the robustness of quantum information against local perturbations, giving the toric code the potential to be feasibly applied in quantum information processing (see Chapter 3).\newline

We start by explicitly defining the ground state by making use of the properties of the vertex and plaquette operators that we have derived in the previous section.

\begin{proposition} (Ground state of the toric code Hamiltonian) The ground states of the toric code Hamiltonian are the simultaneous $+1$ eigenstates of all the $A_v$ and $B_p$ operators. 
\end{proposition}

\textit{Proof.}\newline 
Recall the form of the Hamiltonianin, having $N^2$ lattice sites, in the following way 

\begin{center}
	
	$H = -\sum_{i=1}^{N}
	Av_i - \sum_{j=1}^{N} Bp_j $
	
\end{center}

we already know that commuting hermitian matrices are simultaneously diagonalizable, moreover $Av$ and $Bp$ operators commute for an even number of edges, thus there exists a common base of eigenvectors with their respective eigenvalues. We also know from proposition 1.2 that such eigenvalues can assume two values $\{-1,+1\}$ due to the specific properties of our operators.

To determine the ground state of the Hamiltonian we have to determine the minimum energy of the system. Therefore we compute all the possible combinations of the eigenvalues associated to the vertex and palquette operators to obtain the following range of values:

\begin{center}
	$\sigma( H) =$
	$\begin{cases}
		2N, \hspace{1cm} if \ all\ Av,Bp \ have \ \lambda_{H}= -1\\
		2N-1,\\
		.\\
		intermediate \ energies,\\
		.\\
		-2N+1\\
		-2N, \hspace{1cm} if \ all \ Av,Bp \ have \ \lambda_{H}= +1
	\end{cases}$
	
\end{center}

Taking the minimum of this spectrum means considering $\sigma(H)=-2N$, thus considering only eigenstates for the Hamiltonian associated to $+1$ eigenvalues.
Those are the ones that will form a basis for the ground state manifold of the system.\newline

\hfill $\square$


In order to determine which are the admissible configurations of eigenstates that we can use to form the ground state, we treat the $Av$ and $Bp$ operators as constraint equations over the torus, i.e. the lattice and dual lattice.
As a consequence, all of our configurations will need to respect the two following equations:

\begin{center}
	(1)	$Av|\psi\rangle$ = $+1|\psi\rangle$
\end{center}

\begin{center}
	(2)	$Bp|\psi\rangle$ = $+1|\psi\rangle$
\end{center}

This means that configuration $|\psi\rangle$ needs to be be an eigenvector for all the veretces and plaquette operators. \newline
In order to satisfy equation (1), we look for loop configurations such that if we apply $Av$ to that state, the result would still yield an overall positive value.\newline
Graphically, we identify the loop configurations through strings of 'occupied' edges (shaded in black in figure 1.4 and 1.5), each identified as a qubit in configuration $|1\rangle$.

\begin{definition}(Occupied edges)
	We define as occupied an edge on $mathcal{L}$ whose spin-$\frac{1}{2}$ particle is in superstate $|1\rangle$.
\end{definition}

Then, if we apply $Av$ at one of the open ends of such strings, we have two possibilities: leaving the string of occupied edges open (figure 1.4) or closing the string over the $Av$ operator (figure 1.5).
In the first case, what we do is computing $\sigma^{z} |1\rangle$ for one endpoint of the string, which implies obtaining $-1|1\rangle$ eigenvalue for that endpoint; as a result we violate constraint (1). In the second case, instead, we end up with positive eigenvalues for each edge, this is becuse the negative signs obtained at the endpoints cancel themselves out. Overall, in order to respect contraint (1) we are interested only in closed loops on $Av$. Notice that such loops will always have an even length and will always involve a maximum of two extremities of the $Av$ operator.\textit{\cite{Her20}}

%drawings closed/open loop with Av
\begin{figure}
	
	\begin{center}
		
		\begin{tikzpicture}
			% Draw dashed lines
			\foreach \i in {-3,-2.5,...,3}
			{
				\draw[dashed] (\i,-3) -- (\i,3);
			}
			\foreach \j in {-3,-2.5,...,3}
			{
				\draw[dashed] (-3,\j) -- (3,\j);
			}
			
			
			
			% Draw solid grid and nodes with circles in the middle of each side
			\draw[step=1cm] (-3,-3) grid (3,3);
			\foreach \i in {-2.5,...,2.5}
			{
				\foreach \j in {-2.5,...,2.5}
				{
					
					
					\begin{scope}[transform canvas={xshift=\i cm,yshift=\j cm}]
						\node[right,xshift=0.2cm,yshift=0.4cm] {};
						% Convert \j and \i to integers
						\pgfmathtruncatemacro{\intj}{\j}
						\pgfmathtruncatemacro{\inti}{\i}
						
						% Draw circles at the midpoints of each side
						\ifnum\intj=2
						\draw node[draw,circle,fill=gray] at (0,0.5) {};
						\else
						\draw node[draw,circle,fill=white] at (0,0.5) {};
						\fi
						
						\ifnum\inti=2
						\draw node[draw,circle,fill=gray] at (0.5,0) {};
						\else
						\draw node[draw,circle,fill=white] at (0.5,0) {};
						\fi
						
						\draw node[draw,circle,fill=white] at (0,-0.5) {};
						\draw node[draw,circle,fill=white] at (-0.5,0) {};
					\end{scope}
				}
			}
			
			\foreach \i in {-2,...,-2}
			{
				\draw[black, line width=1.5mm] (\i,-0.5) -- (\i,1.5);
				\node[draw, circle, fill=black] at (\i,-0.5) {};
				\node[draw, circle, fill=black] at (\i,0.5) {};
				
				\draw[blue!50, line width=1.5mm] (\i,1.5) -- (\i,2.5);
				\node[draw, circle, fill=blue!50,label=center:\textbf{Z}] at (\i,1.5) {};
				\node[draw, circle, fill=blue!50,label=center:\textbf{Z}] at (\i,2.5) {};
				
				
			}
			
			\foreach \j in {2,...,2}
			{
				
				\draw[blue!50, line width=1.5mm] (-2.5, \j) -- (-1.5, \j);
				
				\draw node[draw,circle,fill=blue!50,label=center:\textbf{Z},label=north:\textbf{Av}] at (-2.5,\j) {};
				\draw node[draw,circle,fill=blue!50,label=center:\textbf{Z}] at (-1.5,\j) {};
				
			}
			
			
		\end{tikzpicture}
	\end{center}
	
	\caption{Operator $Av$ applied on an open string of occupied edges.}
	\label{fig:Aveigen2}
\end{figure}




\begin{figure}
	
\begin{center}
\begin{tikzpicture}
	% Draw dashed lines
	\foreach \i in {-3,-2.5,...,3}
	{
		\draw[dashed] (\i,-3) -- (\i,3);
	}
	\foreach \j in {-3,-2.5,...,3}
	{
		\draw[dashed] (-3,\j) -- (3,\j);
	}
	
	
	
	% Draw solid grid and nodes with circles in the middle of each side
	\draw[step=1cm] (-3,-3) grid (3,3);
	\foreach \i in {-2.5,...,2.5}
	{
		\foreach \j in {-2.5,...,2.5}
		{
			
			
			\begin{scope}[transform canvas={xshift=\i cm,yshift=\j cm}]
				\node[right,xshift=0.2cm,yshift=0.4cm] {};
				% Convert \j and \i to integers
				\pgfmathtruncatemacro{\intj}{\j}
				\pgfmathtruncatemacro{\inti}{\i}
				
				% Draw circles at the midpoints of each side
				\ifnum\intj=2
				\draw node[draw,circle,fill=gray] at (0,0.5) {};
				\else
				\draw node[draw,circle,fill=white] at (0,0.5) {};
				\fi
				
				\ifnum\inti=2
				\draw node[draw,circle,fill=gray] at (0.5,0) {};
				\else
				\draw node[draw,circle,fill=white] at (0.5,0) {};
				\fi
				
				\draw node[draw,circle,fill=white] at (0,-0.5) {};
				\draw node[draw,circle,fill=white] at (-0.5,0) {};
			\end{scope}
		}
	}
	
	\foreach \i in {-2,...,-2}
	{
		\draw[black, line width=1.5mm] (\i,-1) -- (\i,1.5);
		\node[draw, circle, fill=black] at (\i,-0.5) {};
		\node[draw, circle, fill=black] at (\i,0.5) {};
		
		\draw[blue!50, line width=1.5mm] (\i,1.5) -- (\i,2.5);
		\node[draw, circle, fill=blue!50,label=center:\textbf{Z}] at (\i,1.5) {};
		\node[draw, circle, fill=blue!50,label=center:\textbf{Z}] at (\i,2.5) {};
		
		
	}
	
	\foreach \j in {2,...,2}
	{
		
		\draw[blue!50, line width=1.5mm] (-2.5, \j) -- (-1.5, \j);
		
		
		\draw[black, line width=1.5mm] (-1.5, \j) -- (1, \j);
		\draw node[draw,circle,fill=black] at (-0.5,\j) {};
		\draw node[draw,circle,fill=black] at (0.5,\j) {};
		\draw node[draw,circle,fill=blue!50,label=center:\textbf{Z},label=north:\textbf{Av}] at (-2.5,\j) {};
		\draw node[draw,circle,fill=blue!50,label=center:\textbf{Z}] at (-1.5,\j) {};
		
	}
	
	
	\foreach \i in {1,...,1}
	{
		\draw[black, line width=1.5mm] (\i,-1) -- (\i,2);
		\node[draw, circle, fill=black] at (\i,-0.5) {};
		\node[draw, circle, fill=black] at (\i,0.5) {};
		\draw node[draw,circle,fill=black] at (\i,1.5) {};
		
	}
	
	\foreach \j in {-1,...,-1}
	{
		\draw[black, line width=1.5mm] (-2, \j) -- (1, \j);
		\draw node[draw,circle,fill=black] at (0.5,\j) {};
		\draw node[draw,circle,fill=black] at (-0.5,\j) {};
		\draw node[draw,circle,fill=black] at (-1.5,\j) {};
	}
	
	
\end{tikzpicture}
\end{center}

\caption{String of occupied edges closed on an operator $Av$.}
\label{fig:Aveigen}
\end{figure}
		
In a more formal notation what we are stating is that:

\begin{center}
	$\prod_{i=1}^{4} \sigma_{i}^{z} |\psi\rangle = +1 |\psi\rangle$. 
\end{center}

If we consider only closed loops, we can identify different configurations having such characteristic. The illustrations in figure 1.6 show some examples of them.
Each of these loops, having such characteristics, is an eigenstate for $Av$ since, no matter how I locally apply $Av$, I will always preserve the sign of the state. 

%drawings 4 types of loops


\begin{figure}
	
	\begin{center}
		
		\begin{tikzpicture}
			% Draw dashed lines
			\foreach \i in {-2,-1.5,...,2}
			{
				\draw[dashed] (\i,-2) -- (\i,2);
			}
			\foreach \j in {-2,-1.5,...,2}
			{
				\draw[dashed] (-2,\j) -- (2,\j);
			}
			
			% Draw solid grid and nodes with circles in the middle of each side
			\draw[step=1cm] (-2,-2) grid (2,2);
			\foreach \i in {-1.5,...,1.5}
			{
				\foreach \j in {-1.5,...,1.5}
				{
					\begin{scope}[transform canvas={xshift=\i cm,yshift=\j cm}]
						\node[right,xshift=0.2cm,yshift=0.4cm] {};
						% Convert \j and \i to integers
						\pgfmathtruncatemacro{\intj}{\j}
						\pgfmathtruncatemacro{\inti}{\i}
						
						% Draw circles at the midpoints of each side
						\ifnum\intj=1
						\draw node[draw,circle,fill=gray] at (0,0.5) {};
						\else
						\draw node[draw,circle,fill=white] at (0,0.5) {};
						\fi
						
						\ifnum\inti=1
						\draw node[draw,circle,fill=gray] at (0.5,0) {};
						\else
						\draw node[draw,circle,fill=white] at (0.5,0) {};
						\fi
						
						\draw node[draw,circle,fill=white] at (0,-0.5) {};
						\draw node[draw,circle,fill=white] at (-0.5,0) {};
					\end{scope}
				}
			}
			
			\foreach \i in {-1,...,-1}
			{
				\draw[black, line width=1.5mm] (\i,-1) -- (\i,1);
				\node[draw, circle, fill=black] at (\i,-0.5) {};
				\node[draw, circle, fill=black] at (\i,0.5) {};	
				
			}
			
			\foreach \i in {1,...,1}
			{
				\draw[black, line width=1.5mm] (\i,-1) -- (\i,1);
				\node[draw, circle, fill=black] at (\i,-0.5) {};
				\node[draw, circle, fill=black] at (\i,0.5) {};	
				
			}
			
			\foreach \j in {1,...,1}
			{
				\draw[black, line width=1.5mm] (-1,\j) -- (1,\j);
				\node[draw, circle, fill=black] at (-0.5,\j) {};
				\node[draw, circle, fill=black] at (0.5,\j) {};		
				
			}
			
			\foreach \j in {-1,...,-1}
			{
				\draw[black, line width=1.5mm] (-1,\j) -- (1,\j);
				\node[draw, circle, fill=black] at (-0.5,\j) {};
				\node[draw, circle, fill=black] at (0.5,\j) {};		
				
			}
			
			
		\end{tikzpicture}
		
	\end{center}
	
	\vspace{1cm} 
	
	\begin{center}
		\begin{tikzpicture}
			% Draw dashed lines
			\foreach \i in {-2,-1.5,...,2}
			{
				\draw[dashed] (\i,-2) -- (\i,2);
			}
			\foreach \j in {-2,-1.5,...,2}
			{
				\draw[dashed] (-2,\j) -- (2,\j);
			}
			
			% Draw solid grid and nodes with circles in the middle of each side
			\draw[step=1cm] (-2,-2) grid (2,2);
			\foreach \i in {-1.5,...,1.5}
			{
				\foreach \j in {-1.5,...,1.5}
				{
					\begin{scope}[transform canvas={xshift=\i cm,yshift=\j cm}]
						\node[right,xshift=0.2cm,yshift=0.4cm] {};
						% Convert \j and \i to integers
						\pgfmathtruncatemacro{\intj}{\j}
						\pgfmathtruncatemacro{\inti}{\i}
						
						% Draw circles at the midpoints of each side
						\ifnum\intj=1
						\draw node[draw,circle,fill=gray] at (0,0.5) {};
						\else
						\draw node[draw,circle,fill=white] at (0,0.5) {};
						\fi
						
						\ifnum\inti=1
						\draw node[draw,circle,fill=gray] at (0.5,0) {};
						\else
						\draw node[draw,circle,fill=white] at (0.5,0) {};
						\fi
						
						\draw node[draw,circle,fill=white] at (0,-0.5) {};
						\draw node[draw,circle,fill=white] at (-0.5,0) {};
					\end{scope}
				}
			}
			
			\foreach \i in {0,...,0}
			{
				\draw[black, line width=1.5mm] (\i,-2) -- (\i,2);
				\node[draw, circle, fill=black] at (\i,-0.5) {};
				\node[draw, circle, fill=black] at (\i,0.5) {};		
				\node[draw, circle, fill=black] at (\i,-1.5) {};
				\node[draw, circle, fill=black] at (\i,1.5) {};		
				
			}
			
			
		\end{tikzpicture}		
	\end{center}
	
	\vspace{1cm} 
	
	\begin{center}
		\begin{tikzpicture}
			% Draw dashed lines
			\foreach \i in {-2,-1.5,...,2}
			{
				\draw[dashed] (\i,-2) -- (\i,2);
			}
			\foreach \j in {-2,-1.5,...,2}
			{
				\draw[dashed] (-2,\j) -- (2,\j);
			}
			
			% Draw solid grid and nodes with circles in the middle of each side
			\draw[step=1cm] (-2,-2) grid (2,2);
			\foreach \i in {-1.5,...,1.5}
			{
				\foreach \j in {-1.5,...,1.5}
				{
					\begin{scope}[transform canvas={xshift=\i cm,yshift=\j cm}]
						\node[right,xshift=0.2cm,yshift=0.4cm] {};
						% Convert \j and \i to integers
						\pgfmathtruncatemacro{\intj}{\j}
						\pgfmathtruncatemacro{\inti}{\i}
						
						% Draw circles at the midpoints of each side
						\ifnum\intj=1
						\draw node[draw,circle,fill=gray] at (0,0.5) {};
						\else
						\draw node[draw,circle,fill=white] at (0,0.5) {};
						\fi
						
						\ifnum\inti=1
						\draw node[draw,circle,fill=gray] at (0.5,0) {};
						\else
						\draw node[draw,circle,fill=white] at (0.5,0) {};
						\fi
						
						\draw node[draw,circle,fill=white] at (0,-0.5) {};
						\draw node[draw,circle,fill=white] at (-0.5,0) {};
					\end{scope}
				}
			}
			
			\foreach \j in {0,...,0}
			{
				\draw[black, line width=1.5mm] (-2,\j) -- (2,\j);
				\node[draw, circle, fill=black] at (-0.5,\j) {};
				\node[draw, circle, fill=black] at (0.5,\j) {};		
				\node[draw, circle, fill=black] at (-1.5,\j) {};
				\node[draw, circle, fill=black] at (1.5,\j) {};		
				
			}
			
			
		\end{tikzpicture}
	\end{center}
	
	\vspace{1cm} 
	
	\begin{center}
		\begin{tikzpicture}
			% Draw dashed lines
			\foreach \i in {-2,-1.5,...,2}
			{
				\draw[dashed] (\i,-2) -- (\i,2);
			}
			\foreach \j in {-2,-1.5,...,2}
			{
				\draw[dashed] (-2,\j) -- (2,\j);
			}
			
			% Draw solid grid and nodes with circles in the middle of each side
			\draw[step=1cm] (-2,-2) grid (2,2);
			\foreach \i in {-1.5,...,1.5}
			{
				\foreach \j in {-1.5,...,1.5}
				{
					\begin{scope}[transform canvas={xshift=\i cm,yshift=\j cm}]
						\node[right,xshift=0.2cm,yshift=0.4cm] {};
						% Convert \j and \i to integers
						\pgfmathtruncatemacro{\intj}{\j}
						\pgfmathtruncatemacro{\inti}{\i}
						
						% Draw circles at the midpoints of each side
						\ifnum\intj=1
						\draw node[draw,circle,fill=gray] at (0,0.5) {};
						\else
						\draw node[draw,circle,fill=white] at (0,0.5) {};
						\fi
						
						\ifnum\inti=1
						\draw node[draw,circle,fill=gray] at (0.5,0) {};
						\else
						\draw node[draw,circle,fill=white] at (0.5,0) {};
						\fi
						
						\draw node[draw,circle,fill=white] at (0,-0.5) {};
						\draw node[draw,circle,fill=white] at (-0.5,0) {};
					\end{scope}
				}
			}
			
			
			\foreach \i in {0,...,0}
			{
				\draw[black, line width=1.5mm] (\i,-2) -- (\i,2);
				\node[draw, circle, fill=black] at (\i,-0.5) {};
				\node[draw, circle, fill=black] at (\i,0.5) {};		
				\node[draw, circle, fill=black] at (\i,-1.5) {};
				\node[draw, circle, fill=black] at (\i,1.5) {};		
				
			}
			
			\foreach \j in {0,...,0}
			{
				\draw[black, line width=1.5mm] (-2,\j) -- (2,\j);
				\node[draw, circle, fill=black] at (-0.5,\j) {};
				\node[draw, circle, fill=black] at (0.5,\j) {};		
				\node[draw, circle, fill=black] at (-1.5,\j) {};
				\node[draw, circle, fill=black] at (1.5,\j) {};		
				
			}
			
			
		\end{tikzpicture}		
	\end{center}
	
	\caption{Examples of closed loops yielding $+1$ eigenvalue for $Av$}
	\label{fig:Loops}
\end{figure}


Now we focus on constraint (2). We want to apply the plaquette operator $Bp$ only to eigenstates of $Av$, which we have determined above. The illustration in figure 1.7 shows an example of how to do it in practice with one of the eigenstates of $Av$.
Notice that, after the transformation, we do always end up in a valid eigenstate of $Av$ but the new state $|\psi'\rangle$ is not an eigenstate of $Bp$ by itself.

%drawings Bp applied to loops
\begin{figure}
	
	\begin{center}
		
		\begin{tikzpicture}
			% Draw dashed lines
			\foreach \i in {-2,-1.5,...,2}
			{
				\draw[dashed] (\i,-2) -- (\i,2);
			}
			\foreach \j in {-2,-1.5,...,2}
			{
				\draw[dashed] (-2,\j) -- (2,\j);
			}
			
			% Draw solid grid and nodes with circles in the middle of each side
			\draw[step=1cm] (-2,-2) grid (2,2);
			\foreach \i in {-1.5,...,1.5}
			{
				\foreach \j in {-1.5,...,1.5}
				{
					\begin{scope}[transform canvas={xshift=\i cm,yshift=\j cm}]
						\node[right,xshift=0.2cm,yshift=0.4cm] {};
						% Convert \j and \i to integers
						\pgfmathtruncatemacro{\intj}{\j}
						\pgfmathtruncatemacro{\inti}{\i}
						
						% Draw circles at the midpoints of each side
						\ifnum\intj=1
						\draw node[draw,circle,fill=gray] at (0,0.5) {};
						\else
						\draw node[draw,circle,fill=white] at (0,0.5) {};
						\fi
						
						\ifnum\inti=1
						\draw node[draw,circle,fill=gray] at (0.5,0) {};
						\else
						\draw node[draw,circle,fill=white] at (0.5,0) {};
						\fi
						
						\draw node[draw,circle,fill=white] at (0,-0.5) {};
						\draw node[draw,circle,fill=white] at (-0.5,0) {};
					\end{scope}
				}
			}
			
			\foreach \i in {0,...,0}
			{
				\draw[black, line width=1.5mm] (\i,-2) -- (\i,0);
				\draw[red!80, line width=1.5mm] (\i,0) -- (\i,1);
				\draw[black, line width=1.5mm] (\i,1) -- (\i,2);
				\node[draw, circle, fill=black] at (\i,-0.5) {};
				\node[draw, circle, fill=red!80,label = center:\textbf{X}] at (\i,0.5) {};
				\node[draw, circle, fill=black] at (\i,-1.5) {};
				\node[draw, circle, fill=black] at (\i,1.5) {};		
				
			}
			
			\foreach \i in {-1,...,-1}
			{
				
				\draw[red!80, line width=1.5mm] (\i,0) -- (\i,1);
				\node[draw, circle, fill=red!80,label = center:\textbf{X}] at (\i,0.5) {};
				
			}
			
			\foreach \j in {0,...,0}
			{
				\draw[red!80, line width=1.5mm] (-1,\j) -- (0,\j);
				\node[draw, circle, fill=red!80,label = center:\textbf{X}] at (-0.5,\j) {};		
				
			}
			
			\foreach \j in {1,...,1}
			{
				\draw[red!80, line width=1.5mm] (-1,\j) -- (0,\j);
				\node[draw, circle, fill=red!80,label = center:\textbf{X}] at (-0.5,\j) {};	
				
			}
			
			
		\end{tikzpicture}
		
		
		
		\vspace{1cm} 
		
		
		
		\begin{tikzpicture}
			% Draw dashed lines
			\foreach \i in {-2,-1.5,...,2}
			{
				\draw[dashed] (\i,-2) -- (\i,2);
			}
			\foreach \j in {-2,-1.5,...,2}
			{
				\draw[dashed] (-2,\j) -- (2,\j);
			}
			
			% Draw solid grid and nodes with circles in the middle of each side
			\draw[step=1cm] (-2,-2) grid (2,2);
			\foreach \i in {-1.5,...,1.5}
			{
				\foreach \j in {-1.5,...,1.5}
				{
					\begin{scope}[transform canvas={xshift=\i cm,yshift=\j cm}]
						\node[right,xshift=0.2cm,yshift=0.4cm] {};
						% Convert \j and \i to integers
						\pgfmathtruncatemacro{\intj}{\j}
						\pgfmathtruncatemacro{\inti}{\i}
						
						% Draw circles at the midpoints of each side
						\ifnum\intj=1
						\draw node[draw,circle,fill=gray] at (0,0.5) {};
						\else
						\draw node[draw,circle,fill=white] at (0,0.5) {};
						\fi
						
						\ifnum\inti=1
						\draw node[draw,circle,fill=gray] at (0.5,0) {};
						\else
						\draw node[draw,circle,fill=white] at (0.5,0) {};
						\fi
						
						\draw node[draw,circle,fill=white] at (0,-0.5) {};
						\draw node[draw,circle,fill=white] at (-0.5,0) {};
					\end{scope}
				}
			}
			
			\foreach \i in {0,...,0}
			{
				\draw[black, line width=1.5mm] (\i,-2) -- (\i,0);
				\draw[black, line width=1.5mm] (\i,1) -- (\i,2);
				\node[draw, circle, fill=black] at (\i,-0.5) {};
				
				\node[draw, circle, fill=black] at (\i,-1.5) {};
				\node[draw, circle, fill=black] at (\i,1.5) {};		
				
			}
			
			\foreach \i in {-1,...,-1}
			{
				
				\draw[black, line width=1.5mm] (\i,0) -- (\i,1);
				\node[draw, circle, fill=black] at (\i,0.5) {};
				
			}
			
			\foreach \j in {0,...,0}
			{
				\draw[black, line width=1.5mm] (-1,\j) -- (0,\j);
				\node[draw, circle, fill=black] at (-0.5,\j) {};		
				
			}
			
			\foreach \j in {1,...,1}
			{
				\draw[black, line width=1.5mm] (-1,\j) -- (0,\j);
				\node[draw, circle, fill=black] at (-0.5,\j) {};
				\node[draw, circle, black] at (-0.5,\j) {};	
				
			}
			
			
		\end{tikzpicture}	

	\end{center}

\caption{Examples of $Bp$ to a closed loop.}
\label{fig:applyBp}
\end{figure}


Furthermore, any new configuration that we obtain through the application of the plaquette operator to an eigenstate of $Av$ simply yields one of the possible permutations of the edges of the initial state, provided that the topological characteristics of the loop are preserved.

This means that if we firstly partition the eigenstates of $Av$ in the following fours classes: 

\begin{enumerate}
	\item class 0 : contains all closed loops and thus the null state,
	this is because all of the closed loops can be continuously deformed into a null state;
	
	\item class 1 : contains loops that wind all the way around the horizontal dimension of the torus and their permutations;
	
	\item class 2 : contains loops that wind all the way around the vertical dimension of the torus and their permutations;
	
	\item class 3 : contains loops that wind all the way around both dimension of the torus and their permutations. Notice that the vertical loop must be taken on the dual lattice to yield a valid configuration.
	
\end{enumerate}

Then, applying a plaquette operator to any of the eigenstates belonging to one of the classes above must yield an eigenstate that lies in the same class. This is formally expressed by stating that the class is invariant under the action of the operator $Bp$. 

\begin{proposition} (Invariance under $Bp$) The four classes of eigenstates of $Av$ are invariant under the action of the $Bp$ operator. 
\end{proposition}

We can prove that the above definition holds by means of counterexamples. \newline



%drawing loop chiuso ma unito da Bp
\begin{figure}
	
	\begin{center}	
		\begin{tikzpicture}
			% Draw dashed lines
			\foreach \i in {-2,-1.5,...,2}
			{
				\draw[dashed] (\i,-2) -- (\i,2);
			}
			\foreach \j in {-2,-1.5,...,2}
			{
				\draw[dashed] (-2,\j) -- (2,\j);
			}
			
			% Draw solid grid and nodes with circles in the middle of each side
			\draw[step=1cm] (-2,-2) grid (2,2);
			\foreach \i in {-1.5,...,1.5}
			{
				\foreach \j in {-1.5,...,1.5}
				{
					\begin{scope}[transform canvas={xshift=\i cm,yshift=\j cm}]
						\node[right,xshift=0.2cm,yshift=0.4cm] {};
						% Convert \j and \i to integers
						\pgfmathtruncatemacro{\intj}{\j}
						\pgfmathtruncatemacro{\inti}{\i}
						
						% Draw circles at the midpoints of each side
						\ifnum\intj=1
						\draw node[draw,circle,fill=gray] at (0,0.5) {};
						\else
						\draw node[draw,circle,fill=white] at (0,0.5) {};
						\fi
						
						\ifnum\inti=1
						\draw node[draw,circle,fill=gray] at (0.5,0) {};
						\else
						\draw node[draw,circle,fill=white] at (0.5,0) {};
						\fi
						
						\draw node[draw,circle,fill=white] at (0,-0.5) {};
						\draw node[draw,circle,fill=white] at (-0.5,0) {};
					\end{scope}
				}
			}
			
			\foreach \i in {0,...,0}
			{
				\draw[black, line width=1.5mm] (\i,-2) -- (\i,0);
				\draw[black, line width=1.5mm] (\i,1) -- (\i,2);
				\node[draw, circle, fill=black] at (\i,-0.5) {};
				
				\node[draw, circle, fill=black] at (\i,-1.5) {};
				\node[draw, circle, fill=black] at (\i,1.5) {};		
				
			}
			
			\foreach \i in {-1,...,-1}
			{
				\draw[black, line width=1.5mm] (\i,-2) -- (\i,0);
				\draw[black, line width=1.5mm] (\i,1) -- (\i,2);
				\node[draw, circle, fill=black] at (\i,-0.5) {};
				
				\node[draw, circle, fill=black] at (\i,-1.5) {};
				\node[draw, circle, fill=black] at (\i,1.5) {};		
				
			}
			
			\foreach \i in {-1,...,-1}
			{
				
				\draw[black, line width=1.5mm] (\i,1) -- (\i,2);
				\node[draw, circle, fill=black] at (\i,1.5) {};
				
			}
			
			\foreach \j in {0,...,0}
			{
				\draw[black, line width=1.5mm] (-1,\j) -- (0,\j);
				\node[draw, circle, fill=black] at (-0.5,\j) {};		
				
			}
			
			\foreach \j in {1,...,1}
			{
				\draw[black, line width=1.5mm] (-1,\j) -- (0,\j);
				\node[draw, circle, fill=black] at (-0.5,\j) {};
				\node[draw, circle, black] at (-0.5,\j) {};	
				
			}
			
			
		\end{tikzpicture}	
		
		\vspace{1cm} 
		
		\begin{tikzpicture}
			% Draw dashed lines
			\foreach \i in {-2,-1.5,...,2}
			{
				\draw[dashed] (\i,-2) -- (\i,2);
			}
			\foreach \j in {-2,-1.5,...,2}
			{
				\draw[dashed] (-2,\j) -- (2,\j);
			}
			
			% Draw solid grid and nodes with circles in the middle of each side
			\draw[step=1cm] (-2,-2) grid (2,2);
			\foreach \i in {-1.5,...,1.5}
			{
				\foreach \j in {-1.5,...,1.5}
				{
					\begin{scope}[transform canvas={xshift=\i cm,yshift=\j cm}]
						\node[right,xshift=0.2cm,yshift=0.4cm] {};
						% Convert \j and \i to integers
						\pgfmathtruncatemacro{\intj}{\j}
						\pgfmathtruncatemacro{\inti}{\i}
						
						% Draw circles at the midpoints of each side
						\ifnum\intj=1
						\draw node[draw,circle,fill=gray] at (0,0.5) {};
						\else
						\draw node[draw,circle,fill=white] at (0,0.5) {};
						\fi
						
						\ifnum\inti=1
						\draw node[draw,circle,fill=gray] at (0.5,0) {};
						\else
						\draw node[draw,circle,fill=white] at (0.5,0) {};
						\fi
						
						\draw node[draw,circle,fill=white] at (0,-0.5) {};
						\draw node[draw,circle,fill=white] at (-0.5,0) {};
					\end{scope}
				}
			}
			
			\foreach \i in {0,...,0}
			{
				\draw[black, line width=1.5mm] (\i,-2) -- (\i,0);
				\draw[black, line width=1.5mm] (\i,1) -- (\i,2);
				\node[draw, circle, fill=black] at (\i,-0.5) {};
				
				\node[draw, circle, fill=black] at (\i,-1.5) {};
				\node[draw, circle, fill=black] at (\i,1.5) {};		
				
				\draw[red!80, line width=1.5mm] (\i,0) -- (\i,1);
				\node[draw, circle, fill=red!80, label = center:\textbf{X}] at (\i,0.5) {};		
				
			}
			
			\foreach \i in {-1,...,-1}
			{
				\draw[black, line width=1.5mm] (\i,-2) -- (\i,0);
				\draw[black, line width=1.5mm] (\i,1) -- (\i,2);
				\node[draw, circle, fill=black] at (\i,-0.5) {};
				
				\node[draw, circle, fill=black] at (\i,-1.5) {};
				\node[draw, circle, fill=black] at (\i,1.5) {};		
				
				
				
			}
			
			\foreach \i in {-1,...,-1}
			{
				
				\draw[black, line width=1.5mm] (\i,1) -- (\i,2);
				\draw[red!80, line width=1.5mm] (\i,0) -- (\i,1);
				\node[draw, circle, fill=black] at (\i,1.5) {};
				\node[draw, circle, fill=red!80,label = center:\textbf{X}] at (\i,0.5) {};
				
			}
			
			\foreach \j in {0,...,0}
			{
				\draw[red!80, line width=1.5mm] (-1,\j) -- (0,\j);
				\node[draw, circle, fill=red!80,label = center:\textbf{X}] at (-0.5,\j) {};		
				
			}
			
			\foreach \j in {1,...,1}
			{
				\draw[red!80, line width=1.5mm] (-1,\j) -- (0,\j);
				\node[draw, circle, fill=red!80,label = center:\textbf{X}] at (-0.5,\j) {};
				
				
			}
			
			
		\end{tikzpicture}	
		
		\vspace{1cm} 
		
		\begin{tikzpicture}
			% Draw dashed lines
			\foreach \i in {-2,-1.5,...,2}
			{
				\draw[dashed] (\i,-2) -- (\i,2);
			}
			\foreach \j in {-2,-1.5,...,2}
			{
				\draw[dashed] (-2,\j) -- (2,\j);
			}
			
			% Draw solid grid and nodes with circles in the middle of each side
			\draw[step=1cm] (-2,-2) grid (2,2);
			\foreach \i in {-1.5,...,1.5}
			{
				\foreach \j in {-1.5,...,1.5}
				{
					\begin{scope}[transform canvas={xshift=\i cm,yshift=\j cm}]
						\node[right,xshift=0.2cm,yshift=0.4cm] {};
						% Convert \j and \i to integers
						\pgfmathtruncatemacro{\intj}{\j}
						\pgfmathtruncatemacro{\inti}{\i}
						
						% Draw circles at the midpoints of each side
						\ifnum\intj=1
						\draw node[draw,circle,fill=gray] at (0,0.5) {};
						\else
						\draw node[draw,circle,fill=white] at (0,0.5) {};
						\fi
						
						\ifnum\inti=1
						\draw node[draw,circle,fill=gray] at (0.5,0) {};
						\else
						\draw node[draw,circle,fill=white] at (0.5,0) {};
						\fi
						
						\draw node[draw,circle,fill=white] at (0,-0.5) {};
						\draw node[draw,circle,fill=white] at (-0.5,0) {};
					\end{scope}
				}
			}
			
			\foreach \i in {0,...,0}
			{
				\draw[black, line width=1.5mm] (\i,-2) -- (\i,0);
				\draw[black, line width=1.5mm] (\i,1) -- (\i,2);
				\node[draw, circle, fill=black] at (\i,-0.5) {};
				
				\node[draw, circle, fill=black] at (\i,-1.5) {};
				\node[draw, circle, fill=black] at (\i,1.5) {};		
				
				\draw[black, line width=1.5mm] (\i,0) -- (\i,1);
				\node[draw, circle, fill=black] at (\i,0.5) {};		
				
			}
			
			\foreach \i in {-1,...,-1}
			{
				\draw[black, line width=1.5mm] (\i,-2) -- (\i,0);
				\draw[black, line width=1.5mm] (\i,1) -- (\i,2);
				\node[draw, circle, fill=black] at (\i,-0.5) {};
				
				\node[draw, circle, fill=black] at (\i,-1.5) {};
				\node[draw, circle, fill=black] at (\i,1.5) {};		
				
				
				
			}
			
			\foreach \i in {-1,...,-1}
			{
				
				\draw[black, line width=1.5mm] (\i,1) -- (\i,2);
				\draw[black, line width=1.5mm] (\i,0) -- (\i,1);
				\node[draw, circle, fill=black] at (\i,1.5) {};
				\node[draw, circle, fill=black] at (\i,0.5) {};
				
			}
			
			
			
		\end{tikzpicture}	
		

	\end{center}

\caption{Apply $Bp$ to a closed loop.}
\label{fig:Bpstrano}
\end{figure}

If we take the loop illustrated in figure 1.8 and close it by means of a plaquette operator as illustrated, we would be brought to believe that there indeed exist a way to apply the operator $Bp$ such that we exit class 0 and land in class 1. Though, this is not possible. \newline
In fact, we can show the impossibility of the above action by defining two topological indeces to label the four categories of eigenstates. These indeces can assume values in $\mathbb{Z}_2=\{\overline{0},\overline{1} \}$, whose elements respectively represent the 'sets' of even and odd numbers. In our context such numerosities identify the number of vertical and horizontal loops of a configuration intersecting the vertical and horizontal dimensions of the torus (figure 1.9).

The above indeces can be expressed more clearly as follows: 

\begin{center}
	$n_x= (number \ of \ vertical \ intersections)mod2 = 
	\begin{cases} 
		0mod2 \\
		1mod2  
	\end{cases}$ 
\end{center}
\begin{center}
	$n_y= (number \ of \ horizontal \ intersections)mod2 =\begin{cases} 
		0mod2 \\
		1mod2  
	\end{cases}$ 
\end{center}


in order to create a correspondence with the actual definition of the torus over $(\mathbb{Z}$ x $\mathbb{Z})$. 
%We only change the representatives of the elements $\{\overline{0},\overline{1} \}$ in the intermediate step. \newline

In total our classes are going to be wider than expected; we can label them as : $(\overline{0},\overline{0} )$, $(\overline{0},\overline{1} )$, $(\overline{1},\overline{0})$, $(\overline{1},\overline{1})$.\newline

For example, if we take again the configuration in figure 1.9, we have 1 vertical intersection and 1 horizontal intersection, therefore we are in the class three indexed by $n_x=1mod2$ and $n_y=1mod2$, which is labelled as $(\overline{1},\overline{1})$. This would have been true for any odd number of vertical and horizintal intersections, since they all fall in the set of numbers given by $1mod2$.


%drawing 1 vertical 1 hor
\begin{figure}
	\begin{center}
	\begin{tikzpicture}
		% Draw dashed lines
		\foreach \i in {-2,-1.5,...,2}
		{
			\draw[dashed] (\i,-2) -- (\i,2);
		}
		\foreach \j in {-2,-1.5,...,2}
		{
			\draw[dashed] (-2,\j) -- (2,\j);
		}
		
		% Draw solid grid and nodes with circles in the middle of each side
		\draw[step=1cm] (-2,-2) grid (2,2);
		\foreach \i in {-1.5,...,1.5}
		{
			\foreach \j in {-1.5,...,1.5}
			{
				\begin{scope}[transform canvas={xshift=\i cm,yshift=\j cm}]
					\node[right,xshift=0.2cm,yshift=0.4cm] {};
					% Convert \j and \i to integers
					\pgfmathtruncatemacro{\intj}{\j}
					\pgfmathtruncatemacro{\inti}{\i}
					
					% Draw circles at the midpoints of each side
					\ifnum\intj=1
					\draw node[draw,circle,fill=gray] at (0,0.5) {};
					\else
					\draw node[draw,circle,fill=white] at (0,0.5) {};
					\fi
					
					\ifnum\inti=1
					\draw node[draw,circle,fill=gray] at (0.5,0) {};
					\else
					\draw node[draw,circle,fill=white] at (0.5,0) {};
					\fi
					
					\draw node[draw,circle,fill=white] at (0,-0.5) {};
					\draw node[draw,circle,fill=white] at (-0.5,0) {};
				\end{scope}
			}
		}
		
		
		\foreach \i in {0,...,0}
		{
			\draw[black, line width=1.5mm] (\i,-2) -- (\i,2);
			\node[draw, circle, fill=black] at (\i,-0.5) {};
			\node[draw, circle, fill=black] at (\i,0.5) {};		
			\node[draw, circle, fill=black] at (\i,-1.5) {};
			\node[draw, circle, fill=black] at (\i,1.5) {};		
			
		}
		
		\foreach \j in {0,...,0}
		{
			\draw[black, line width=1.5mm] (-2,\j) -- (2,\j);
			\node[draw, circle, fill=black] at (-0.5,\j) {};
			\node[draw, circle, fill=black] at (0.5,\j) {};		
			\node[draw, circle, fill=black] at (-1.5,\j) {};
			\node[draw, circle, fill=black] at (1.5,\j) {};		
			
		}
		
		
	\end{tikzpicture}		
\end{center}

\caption{Notice that the above mentioned intersections are meant to be computed by fixing two circles (figure 1.9): one horizontal circle and one vertical circle passing passing through the spins of the main lattice.}
\label{fig:thirdkind}
\end{figure}


Similarly, if we go back to the configuration in figure 1.8, this would mean that, after having applied $Bp$ we do not land in class 1, as thought, but in class 0; this is because we have an even number of horizontal intersections and an even number of vertical intersections, i.e.   $(\overline{0}, \overline{0})$.

What follows naturally is that all the classes are $Bp$-$invariant$ due to the topological characteristics of the torus. More formally we can write that, given a set of eigenstates of $Av$ named $|\psi_1\rangle,...,|\psi_i\rangle,...,|\psi_n\rangle$ all belonging to the same class and forming state $|\psi\rangle$. Given that we know that $Bp|\psi_i\rangle=+|\psi_j\rangle$, then if we apply $Bp$ to the normalized state $|\psi\rangle$ we get:

\begin{center}
	$|\Xi\rangle= Bp \frac{1}{\sqrt{n}} \sum_{j=1}^{n} |\psi_i\rangle = \frac{1}{\sqrt{n}} \sum_{j=1}^{n} Bp |\psi_i\rangle = \frac{1}{\sqrt{n}} \sum_{j=1}^{n} |\psi_J\rangle =|\Xi\rangle$
\end{center}

because, as anticipated in the previous pages, what $Bp$ does is only permuting the $|\psi_i\rangle$, thus we get back our initial state. This result leads to the following proposition:

\begin{proposition}(Eigenstates of $Bp$) A completely symmetric superposition of a set of eigenstates of $Av$ is an eigenstate of $Bp$.
\end{proposition}

From Proposition 1.6 and Proposition 1.7, also naturally follows the degeneracy (and dimension) of the ground state: \newline

\begin{proposition} (Degeneracy of the ground state) The degeneracy of the ground state manifold is $4$. 
\end{proposition}

Because the four classes of eigenstates are $Bp$-$invariant$, we can construct a valid ground state through the configurations belonging to one the four classes, as these configurations respect both contraint (1) and (2).\newline 

Now that we have derived the ground state configurations of the toric code, we focus on studying its excited states. These states arise from perturbations or interactions within the system and offer valuable insights into the dynamic behavior of the toric code beyond its ground state equilibrium. While the ground state embodies the system's lowest energy configuration and serves as the foundation of its stability, the excited states provide an insight into the system's response to external influences. 
Through the systematic study of these excitations we discover some intrinsic characteristics of the toric code that will serve us later to explain its fault-tolerant nature.


%-------------------------





















\newpage
\section{Excited states}
\label{sec:ES}

%We shall now examine the case of a square lattice with boundary conditions, i.e. the toric code. \newline

Taking into account the results yielded by the examination of the physical system in the previous chapters we will now proceed to describe, firstly, what are the excitations on the ground state and what kind of excitations we can obtain on it; secondly, we will study their exchange statistics. In general, the low-energy excitations of a quantum system, often referred to as quasiparticles, represent deviations from the system's ground state. These excitations are associated with elementary quantum particles that can be created or annihilated with relatively low energy. The properties of these quasiparticles, including their statistics, are crucial for gaining insights into the system's ground state degeneracy, topological properties and response to external perturbations. \newline

In the particular case of the toric code, such low-energy excitations can be created through the application of a $S^x$ or $S^z$ operator (figure 1.10).

\begin{definition}($S^x$ operator)
	A $S^x$ operator corresponds to an open-ended string on the GS made up of $\sigma^x$ applied on $\mathcal{L}$.
\end{definition}

\begin{definition}($S^z$ operator)
	A $S^z$ operator corresponds to an open-ended string on the GS made up of $\sigma^z$ applied on $\mathcal{L'}$.
\end{definition}

At the extremities of the above mentioned operators particles are created. In particular we can get (figure 1.10):

\begin{definition}(Electric charges)
	The pairs of particles obtained through the application of operator $S^x$ are called electric charges $e$.
\end{definition}

\begin{definition}(Magnetic vorteces)
	The pairs of particles obtained through the application of operator $S^z$ are called magnetic vorteces $m$.
\end{definition}

These excitations can be transported on the GS by means of Pauli operators and interact with each other and with each component of the system in several ways, influencing at the same time the low energy of ground state.\newline
In order to understand how the string operators influence the ground state, we need to study the relationships between string operators and vertex/plaquette operators as we have defined them in section 1.2.\newline

%picture
\begin{figure}
	
	\begin{center}
		
		\begin{tikzpicture}
			% Draw dashed lines
			\foreach \i in {-3,-2.5,...,3}
			{
				\draw[dashed] (\i,-3) -- (\i,3);
			}
			\foreach \j in {-3,-2.5,...,3}
			{
				\draw[dashed] (-3,\j) -- (3,\j);
			}
			
			
			
			% Draw solid grid and nodes with circles in the middle of each side
			\draw[step=1cm] (-3,-3) grid (3,3);
			\foreach \i in {-2.5,...,2.5}
			{
				\foreach \j in {-2.5,...,2.5}
				{
					
					
					\begin{scope}[transform canvas={xshift=\i cm,yshift=\j cm}]
						\node[right,xshift=0.2cm,yshift=0.4cm] {};
						% Convert \j and \i to integers
						\pgfmathtruncatemacro{\intj}{\j}
						\pgfmathtruncatemacro{\inti}{\i}
						
						% Draw circles at the midpoints of each side
						\ifnum\intj=2
						\draw node[draw,circle,fill=gray] at (0,0.5) {};
						\else
						\draw node[draw,circle,fill=white] at (0,0.5) {};
						\fi
						
						\ifnum\inti=2
						\draw node[draw,circle,fill=gray] at (0.5,0) {};
						\else
						\draw node[draw,circle,fill=white] at (0.5,0) {};
						\fi
						
						\draw node[draw,circle,fill=white] at (0,-0.5) {};
						\draw node[draw,circle,fill=white] at (-0.5,0) {};
					\end{scope}
				}
			}
			
			\foreach \i in {-2,...,-2}
			{
				\draw[red!30, line width=1.5mm] (\i,-1) -- (\i,-0.5);
				\draw[red!50, line width=1.5mm] (\i,-0.5) -- (\i,1.5);
				\draw[red!30, line width=1.5mm] (\i,1.5) -- (\i,2);
				\node[draw, circle, fill=red!50,label=center:\textbf{$\sigma^x$},label=south:\textbf{\large \bf $e$}] at (\i,-0.5) {};
				\node[draw, circle, fill=red!50,label=center:\textbf{$\sigma^x$}] at (\i,0.5) {};
				\node[draw, circle, fill=red!50,label=center:\textbf{$\sigma^x$},label=north:\textbf{\bf \large $e$}] at (\i,1.5) {};
				
			}
			
			\foreach \j in {-1.5,...,-1.5}
			{
				\draw[blue!30, line width=1.5mm] (0.5, \j) -- (1, \j);
				\draw[blue!50, line width=1.5mm] (1, \j) -- (2, \j);
				\draw[blue!30, line width=1.5mm] (2, \j) -- (2.5, \j);
				
				\draw node[draw,circle,fill=blue!50,label=center:\textbf{$\sigma^z$},label=left:\textbf{\large \bf $m$}] at (1,\j) {};
				\draw node[draw,circle,fill=blue!50,label=center:\textbf{$\sigma^z$},label=right:\textbf{\large \bf $m$}] at (2,\j) {};
				
			}
			
			
		\end{tikzpicture}
	\end{center}
	
	\caption{$e$ and $m$ excitations on the ground state.}
	\label{fig:excitations}
\end{figure}





\newpage
We can prove that a $S^x$ operator anticommutes with one $A_v$ for each one of its endpoints. 

\begin{proposition}(Anticommutation of $S^x$)
	The $S^x$ operators anticommute with two $A_v$ each.
\end{proposition}

\textit{Proof.}\newline
We want to show that $A_v S^x + S^x A_v=0$. Knowing that 

\begin{center}
	$A_v = \prod_{i=1}^{4} \sigma_i^z$ \\ 
	$S^x = \prod_{j=1}^{N} \sigma_j^x$
\end{center}

by substitution we obtain the following expression: 

\begin{center}
	$(1)$ $\space$ $\prod_{i=1}^{4} \sigma_i^z \prod_{j=1}^{N} \sigma_j^x + \prod_{j=1}^{N} \sigma_j^x \prod_{i=1}^{4} \sigma_i^z = 0$  
\end{center}

remeber that only one $\sigma_i^z$ will overlap with the extremity of the string $S^x$; thus, only one $\sigma_i^z$ will anticommute with the extremity of the string $S^x$. Knowing that for Pauli matrices acting on the same edge we have anticommutation

\begin{center}
	$\sigma_{v'}^x \sigma_{v'}^z = - \sigma_{v'}^z \sigma_v^x$ for $v=v'$ \\
	$\sigma_v^x \sigma_{v'}^z =  \sigma_{v'}^z \sigma_v^x$ for $v \neq v'$ 
\end{center}

if for simplicity we fix $N=2$ we can easily see that :

\begin{center}
	$(\sigma_1^z \sigma_2^z \sigma_3^z \sigma_4^z)(\sigma_1^x \sigma_2^x)  = - (\sigma_1^x \sigma_2^x)(\sigma_1^z \sigma_2^z \sigma_3^z \sigma_4^z) $ 
\end{center}

having supposed that, for example, the extremity $\sigma_1^x$ overlaps with the $\sigma_3^z$ edge part of the $A_v$ operator (figure 1.11). This can then be easily generalized for $N$ Pauli operators. Finally, if we substitute such expression in $(1)$ we obtain that the overall equation is in fact equal to zero. \newline
This procedure should be iterated also for the other extremity of the string $S^x$ to show that the operator commutes with two $A_v$ operators, one for each endpoint.\newline

\hfill $\square$ 


\begin{figure}
	
	\begin{center}
		
		\begin{tikzpicture}
			% Draw dashed lines
			\foreach \i in {-3,-2.5,...,3}
			{
				\draw[dashed] (\i,-3) -- (\i,3);
			}
			\foreach \j in {-3,-2.5,...,3}
			{
				\draw[dashed] (-3,\j) -- (3,\j);
			}
			
			
			
			% Draw solid grid and nodes with circles in the middle of each side
			\draw[step=1cm] (-3,-3) grid (3,3);
			\foreach \i in {-2.5,...,2.5}
			{
				\foreach \j in {-2.5,...,2.5}
				{
					
					
					\begin{scope}[transform canvas={xshift=\i cm,yshift=\j cm}]
						\node[right,xshift=0.2cm,yshift=0.4cm] {};
						% Convert \j and \i to integers
						\pgfmathtruncatemacro{\intj}{\j}
						\pgfmathtruncatemacro{\inti}{\i}
						
						% Draw circles at the midpoints of each side
						\ifnum\intj=2
						\draw node[draw,circle,fill=gray] at (0,0.5) {};
						\else
						\draw node[draw,circle,fill=white] at (0,0.5) {};
						\fi
						
						\ifnum\inti=2
						\draw node[draw,circle,fill=gray] at (0.5,0) {};
						\else
						\draw node[draw,circle,fill=white] at (0.5,0) {};
						\fi
						
						\draw node[draw,circle,fill=white] at (0,-0.5) {};
						\draw node[draw,circle,fill=white] at (-0.5,0) {};
					\end{scope}
				}
			}
			
			\foreach \i in {-2,...,-2}
			{
				\draw[red!50, line width=1.5mm] (\i,-0.5) -- (\i,1.5);
				\node[draw, circle, fill=red!50,label=center:\textbf{$\sigma^x_3$}] at (\i,-0.5) {};
				\node[draw, circle, fill=red!50,label=center:\textbf{$\sigma^x_2$}] at (\i,0.5) {};
				
				\draw[blue!50, line width=1.5mm] (\i,1.5) -- (\i,2.5);
				\node[draw, circle, fill=blue!50,label=center:\textbf{$\sigma^z_3$}] at (\i,1.5) {};
				\node[draw, circle, fill=blue!50,label=center:\textbf{$\sigma^z_1$}] at (\i,2.5) {};
				
				
			}
			
			\foreach \j in {2,...,2}
			{
				
				\draw[blue!50, line width=1.5mm] (-2.5, \j) -- (-1.5, \j);
				
				\draw node[draw,circle,fill=blue!50,label=center:\textbf{$\sigma^z_1$},label=north:\textbf{Av}] at (-2.5,\j) {};
				\draw node[draw,circle,fill=blue!50,label=center:\textbf{$\sigma^z_2$}] at (-1.5,\j) {};
				
			}
			
			\foreach \j in {-1,...,-1}
			{
				
				\draw[blue!50, line width=1.5mm] (-2.5, \j) -- (-1.5, \j);
				
			    \node[draw,circle,fill=blue!50,label=center:\textbf{$\sigma^z$},label=north:\textbf{Av}] at (-2.5,\j) {};
				\node[draw,circle,fill=blue!50,label=center:\textbf{$\sigma^z$}] at (-1.5,\j) {};
				
			}
			
			\foreach \i in {-2,...,-2}
			{
				
				\draw[blue!50, line width=1.5mm] (\i,-0.5) -- (\i,-1.5);
				\node[draw, circle, fill=blue!50, label=center:\textbf{$\sigma^z$}] at (\i,-1.5) {};
				\node[draw, circle, fill=blue!50, label=center:\textbf{$\sigma^z$}] at (\i,-0.5) {};
				
				
			}
			
			
		\end{tikzpicture}
	\end{center}
	
	\caption{String operator anticommutes with two $A_v$ each.}
	\label{fig:Av&string}
\end{figure}


In the same way, we could prove that, if we take a string $S^z$ it anticommutes with one $B_p$ for each one of its endpoints.

\begin{proposition}(Anticommutation of $S^z$)
	The $S^z$ operators anticommute with two $B_p$ each.
\end{proposition}

The effect of putting such strings on the ground states correspond to raising the associated energy of the GS by 2. In fact, coherently with what said in {\cite{Her20}}, it is impossible to create unitary excitations on the ground states.

\begin{proposition}(Magnitude of the excitations)
	Placing a string operator on the ground states corresponds to raising the associated energy of the GS by 2.
\end{proposition}

\textit{Proof.} \newline
Recall that for a string operator, here we will choose $S^x$, both equations hold:

\begin{center}
	$\begin{cases} 
		A_{v1} S^x + S^x A_{v1} =0 \\
		A_{v2} S^x + S^x A_{v2} =0
	\end{cases}$ 
\end{center}

If we apply the string operator to the Ground State, taking into account anticommutativity stated above, we obtain that:

\begin{center}
	$\begin{cases}
		A_{v1} S^x |GS\rangle = - S^x A_{v1} |GS\rangle = - S^x |GS\rangle \\
		
		A_{v2} S^x |GS\rangle = - S^x A_{v2} |GS\rangle = - S^x |GS\rangle
	\end{cases}$ 
\end{center}

since $A_{v1}|GS\rangle = +1|GS\rangle$ and $A_{v2}|GS\rangle = +1|GS\rangle$.
Summing up term by term :

\begin{center}
	$\begin{cases}
		A_{v1} S^x |GS\rangle + A_{v2} S^x |GS\rangle = - 2 S^x A_{v1} |GS\rangle
	\end{cases}$ 
\end{center}

Thus, if the energy of the GS, as we have defined it in the previous sections, is $-2N$ by acting with a string operator on  it we obtain a final energy equal to $-2N+2$.

\hfill $\square$ 

%Firstly, we study the individual statistics of electric charges and magnetic votices, then we argue their mutual statistics.\newline






\subsection{Fermionic and bosonic exchange statistics in 3D and 2D}
We shall give a brief description of the statistics of fermions and bosons in both cases of three and two dimensions.
Let us first consider the exchange statistics of two particles in three dimensions. We can describe a system of two particles moving from point $A$ at $t_1$ to point $B$ at $t_2$ by means of the following path integral formulation discussed by \cite{Rao16}:

\begin{center}
	$A$ = $\sum_{paths} e^{iS}$
\end{center}

where $S$ represents the integral of the lagrangian $\mathcal{L}$ over time.
% which provides a measure of the total \textit{action} along a particular trajectory or path taken by the system. 

\begin{center}
	$S = \int_{t_1}^{t_2}dt \textit{$\mathcal{L}$} [ r_1(t),r_2(t) ] $.
\end{center}

In quantum mechanics, the path integral formulation involves considering all possible paths between the initial and final states and assigning a phase factor $e^{iS}$ to each. Then, the probability amplitude for a system to evolve from one point to another is given by the sum over all paths of these phase factors \cite{Wil91}. Since we are talking about a system of indistinguishable particles, we will only have two classes of paths in three dimensions: the direct paths and paths with exchanges. This is because the final configuration at time $t_2$ will always be the same due to indistinguishability. Though, notice that even though two paths may lead to the same final configuration they can exhibit different behaviours. To visualize this, consider the case where particles move in space but their paths do not intersect. In such a scenario, paths are constrained to move on a sphere. Therefore, closed paths emerge when particles return to their original positions (no exchange) or reach the antipodal point (exchange). In fact, we will have three distinct types of paths (figure 1.12): 

%picture


\begin{enumerate}
	\item[(a)]: closed loops without exchange; they can disappear if reduced to a point, implying that the wave-function cannot acquire any phase; 
	
	\item[(b)] : involve a single exchange; they connect two fixed points on the sphere, making it impossible to erase them. Consequently, this types of exchanges introduce a phase in the wave-function;
	
	\item[(c)] : closed loops with two exchanges, as in case 1; as a result, also in this case the wavefunction does not acquire any additional phase.
\end{enumerate}

\begin{figure}
	\centering
	\includegraphics[width=0.8\linewidth]{"../Thesis/Immagine 2024-02-22 113658"}
	\caption{\textit{ Possible exchanges of two particles in 3D. From: \cite{Rao16}.}}
	\label{fig:immagine-2024-02-22-113658}
\end{figure}

Overall, we only have two classes of paths: those that do not involve any exchange as in class a and c, and those that are characterized by an exchange as in class b. { If we let $\eta$ represent the phase acquired by the wave function under a single exchange then, since two exchanges are equivalent to no exchange $\eta^2=1$, it follows that $\eta = \pm 1 $. We can conclude that the only possible statistics in three dimensions are indeed the fermionic and bosonic ones.  \cite{Rao16}}.\newline

In terms of the path integral formulation that we have provided above, what we will get is an overall amplitude equal to:

\begin{center}
	$\sum_{direct \ paths} e^{iS}$ + $e^{i \phi} \sum_{exchange \ paths} e^{iS}$.
\end{center}

Since exchanging the particles twice leads again to a direct path, we will have $e^{2i \phi} = 1$. This means that $\phi$ can either be 0 or $\pi$ giving rise, respectively, to bosonic and fermionic statistics. 

\begin{enumerate}
	\item $\phi = 0$: this case leads to the constructive interference of the exchange paths. The resulting particles are known as bosons, and their behavior is characterized by this specific phase condition;
	
	\item $\phi = \pi$: this case results in destructive interference between the exchange paths. The particles corresponding to this scenario are referred to as fermions, and their behavior is governed by this particular phase condition.
\end{enumerate}

In contrast to what stated above, in the two dimensional case the topology of the space configuartion changes. In fact, in this case we end up moving on a circle. Though, this time several possible paths are possible, in particular closed paths. In fact, if we classify again the obtainable paths as we did above, this time we obtain these three categories (figure 1.13):

\begin{enumerate}
	\item[(a)] : those that can be shrunk into a point;

	\item[(b)] : those that cannot be contracted into a point as the endpoints remain fixed;

	\item[(c)] : also this kind of paths cannot be shrunk into a point since they wind all  the way around the circle. Notice that in three dimensions the path that forms under two exchanges can be erased but this is not possible if the exchange happens in two dimensions. This will be useful to define a third type of quasiparticles that exists only in 2D (section 1.5).
\end{enumerate}

\begin{figure}[H]
	\centering
	\includegraphics[width=0.8\textwidth]{"../Thesis/Immagine 2024-02-22 112906"}
	\caption{ \textit{ Possible exchanges of two particles in 2D. From: \cite{Rao16}.}}
	\label{fig:immagine-2024-02-22-112906}
\end{figure}

Bacause of the third class of paths, { if we let $\eta$ represent the phase associated with a single exchange, $\eta^2$ will denote the phase under two exchanges and $\eta^3$ the phase under three exchanges and so forth. The crucial observation is that, given that the wavefunction's modulus remains constant during exchanges, we can express $\eta$ as a phase factor $e^{i\theta}$. \cite{Rao16}}. \newline

Again, in terms of path integral formulation, what we get as probability amplitude is:

\begin{center}
	$\sum_{direct \ paths} e^{iS}$ + $e^{i \phi} \sum_{ one \ exchange} e^{iS}$  + $e^{2i \phi} \sum_{ two \ exchanges} e^{iS}$  + ...
\end{center}

where for $ \phi = 0, \pi$ we would obtain the usual bosons and fermions statistics, but since in general, $e^{in\phi} \neq 1$ for any $n$ (n exchanges never yield the identity), $\phi$ can be anything so any statistic is possible in two dimensions.
Such diverse statistics in two dimensions are called \textit{anyonic} (see section 1.5).\newline

We should also briefly mention that, in a formal mathematical context, the analysis presented in this section categorizes fermions and bosons within the permutation group $\mathbb{P}_N$. {\cite{Cor23}}. Conversely, the subsequent analysis of anyonic excitations classifies them according to the braid group $\mathbb{B}_N$. {\cite{Rao16}}. \newline
The permutation group $\mathbb{P}_N$ is defined as the set of all possible permutations involving $N$ objects. Within this group, the operation of group multiplication (exchange) corresponds to the sequential application of permutations, and the group inverse involves the inverse of the permutation. Here, executing a permutation twice on two objects results in the initial configuration of the system. Consequently, particles exhibiting transformations according to the representations of the permutation group can exclusively fall into the categories of either bosons or fermions. On the other hand, when it comes to objects that are part of the braid group $\mathbb{B}_N$, we can visualise the process of moving (or exchanging) particles as paths in spacetime with time being the vertical axis and space being the horizontal axis (figure 1.14). The particles can circle around each other and form closed paths by coming back to their original positions {\cite{Wil91}}. \newline


\begin{figure}
	\centering
	\includegraphics[width=0.5\linewidth]{"../Thesis/Immagine 2024-02-23 134521"}
	\caption{\textit{ Identical particles exchange position twice as
			they travel in three dimensions (no exchange). Particle a and b move along a closed path in two dimesions (non contractible). From:\cite{Wil91}}.}
	\label{fig:immagine-2024-02-23-134521}
\end{figure}
%----

\newpage
We can now analyze the statistics of $e$-particles in the context of the ground state of the toric code. If we take a simple string operator $\sigma^z|GS\rangle$ , where $|GS\rangle$ denotes the ground state, it creates two $e$-particles in the vertices adjacent to the edge, as shown in figure 1.15. We know that a string operator of the form $\prod_{j=1}^{N} \sigma_j^z$ allows the separation of these two excitations. In fact, as illustrated in figure 1.15, the exchange of these particles becomes possible by applying the string operator in such a way that it moves the excitations around the lattice until they reach configuration shown in figure 1.15(c). However, due to the commutation of all $\sigma^x$ with each other, after the application of a closed loop of $\sigma^x$ (last passage in figure 1.15), there is no overall acquired phase. Consequently, the $e$-excitations exhibit bosonic statistics. {\cite{Rao16, Her20}}. A similar argument can be carried out for the $m$-excitations, establishing them as bosons as well. 


%picture
\begin{figure}
\begin{center}
	\begin{tikzpicture}
		% Draw dashed lines
		\foreach \i in {-2,-1.5,...,2}
		{
			\draw[dashed] (\i,-2) -- (\i,2);
		}
		\foreach \j in {-2,-1.5,...,2}
		{
			\draw[dashed] (-2,\j) -- (2,\j);
		}
		
		
		% Draw solid grid and nodes with circles in the middle of each side
		\draw[step=1cm] (-2,-2) grid (2,2);
		\foreach \i in {-1.5,...,1.5}
		{
			\foreach \j in {-1.5,...,1.5}
			{
				\begin{scope}[transform canvas={xshift=\i cm,yshift=\j cm}]
					\node[right,xshift=0.2cm,yshift=0.4cm] {};
					% Convert \j and \i to integers
					\pgfmathtruncatemacro{\intj}{\j}
					\pgfmathtruncatemacro{\inti}{\i}
					
					% Draw circles at the midpoints of each side
					\ifnum\intj=1
					\draw node[draw,circle,fill=gray] at (0,0.5) {};
					\else
					\draw node[draw,circle,fill=white] at (0,0.5) {};
					\fi
					
					\ifnum\inti=1
					\draw node[draw,circle,fill=gray] at (0.5,0) {};
					\else
					\draw node[draw,circle,fill=white] at (0.5,0) {};
					\fi
					
					\draw node[draw,circle,fill=white] at (0,-0.5) {};
					\draw node[draw,circle,fill=white] at (-0.5,0) {};
				\end{scope}
			}
		}
		
		
		
		
		\foreach \j in {1,...,1}
		{
			
			\draw[red!50, line width=1.5mm] (-1, \j) -- (-0.5, \j);
			%\draw node[label=north:\textbf{ \large $e_1$}] at (-1,\j) {};
			\draw[red!50, line width=1.5mm] (-0.5, \j) -- (0, \j);
			%\draw node[label=north:\textbf{ \large $e_2$}] at (0,\j) {};
			\node[draw, circle, fill=red!50,label=center:\textbf{$\sigma^x$},label=left:\textbf{ \large $e_1$},label=right:\textbf{ \large $e_2$}] at (-0.5,1) {};
			
		}
		
	\end{tikzpicture}
\end{center}

\vspace*{1cm}

\begin{center}
	\begin{tikzpicture}
		% Draw dashed lines
		\foreach \i in {-2,-1.5,...,2}
		{
			\draw[dashed] (\i,-2) -- (\i,2);
		}
		\foreach \j in {-2,-1.5,...,2}
		{
			\draw[dashed] (-2,\j) -- (2,\j);
		}
		
		
		% Draw solid grid and nodes with circles in the middle of each side
		\draw[step=1cm] (-2,-2) grid (2,2);
		\foreach \i in {-1.5,...,1.5}
		{
			\foreach \j in {-1.5,...,1.5}
			{
				\begin{scope}[transform canvas={xshift=\i cm,yshift=\j cm}]
					\node[right,xshift=0.2cm,yshift=0.4cm] {};
					% Convert \j and \i to integers
					\pgfmathtruncatemacro{\intj}{\j}
					\pgfmathtruncatemacro{\inti}{\i}
					
					% Draw circles at the midpoints of each side
					\ifnum\intj=1
					\draw node[draw,circle,fill=gray] at (0,0.5) {};
					\else
					\draw node[draw,circle,fill=white] at (0,0.5) {};
					\fi
					
					\ifnum\inti=1
					\draw node[draw,circle,fill=gray] at (0.5,0) {};
					\else
					\draw node[draw,circle,fill=white] at (0.5,0) {};
					\fi
					
					\draw node[draw,circle,fill=white] at (0,-0.5) {};
					\draw node[draw,circle,fill=white] at (-0.5,0) {};
				\end{scope}
			}
		}
		
		
		
		
		\foreach \j in {1,...,1}
		{
			
			\draw[red!50, line width=1.5mm] (-1, \j) -- (-0.5, \j);
			\draw node[label=center:\textbf{\large $e_2$}] at (0,\j) {};
			%\draw[red!50, line width=1.5mm] (-0.5, \j) -- (0, \j);
			
			\node[draw, circle, fill=red!50,label=center:\textbf{$\sigma^x$}] at (-0.5,1) {};
			
		}
		
		\foreach \i in {-1,...,-1}
		{
			
			\draw[red!50, line width=1.5mm] (\i,1) -- (\i,-1);
			%\draw node[label=north:\textbf{e1}] at (\i,0) {};
			%\draw[blue!50, line width=1.5mm] (\i,0) -- (\i,0);
			
			\node[draw, circle, fill=red!50,,label=center:\textbf{$\sigma^x$}] at (-1,0.5) {};
			\node[draw, circle, fill=red!50,,label=center:\textbf{$\sigma^x$}] at (-1,-0.5) {};
			
		}
		
		\foreach \j in {-1,...,-1}
		{
			
			\draw[red!50, line width=1.5mm] (-1, \j) -- (1, \j);
			
			
			\node[draw, circle, fill=red!50,label=center:\textbf{$\sigma^x$}] at (0.5,-1) {};
			\node[draw, circle, fill=red!50,label=center:\textbf{$\sigma^x$}] at (1,-0.5) {};
			\node[draw, circle, fill=red!50,label=center:\textbf{$\sigma^x$}] at (-0.5,-1) {};
			
		}
		
		\foreach \i in {1,...,1}
		{
			
			\draw[red!50, line width=1.5mm] (\i,-0.4) -- (\i,0.4);
			\draw node[label=north:\textbf{\large $e_1$}] at (\i,0.5) {};
			%\draw[blue!50, line width=1.5mm] (\i,0) -- (\i,0);
			\draw[red!50, line width=1.5mm] (\i,-1) -- (\i,-0.6);
			
			\node[draw, circle, fill=red!50,,label=center:\textbf{$\sigma^x$}] at (1,0.5) {};
			
		}
		
		
		
	\end{tikzpicture}
\end{center}

\vspace*{1cm}

\begin{center}
	\begin{tikzpicture}
		% Draw dashed lines
		\foreach \i in {-2,-1.5,...,2}
		{
			\draw[dashed] (\i,-2) -- (\i,2);
		}
		\foreach \j in {-2,-1.5,...,2}
		{
			\draw[dashed] (-2,\j) -- (2,\j);
		}
		
		
		% Draw solid grid and nodes with circles in the middle of each side
		\draw[step=1cm] (-2,-2) grid (2,2);
		\foreach \i in {-1.5,...,1.5}
		{
			\foreach \j in {-1.5,...,1.5}
			{
				\begin{scope}[transform canvas={xshift=\i cm,yshift=\j cm}]
					\node[right,xshift=0.2cm,yshift=0.4cm] {};
					% Convert \j and \i to integers
					\pgfmathtruncatemacro{\intj}{\j}
					\pgfmathtruncatemacro{\inti}{\i}
					
					% Draw circles at the midpoints of each side
					\ifnum\intj=1
					\draw node[draw,circle,fill=gray] at (0,0.5) {};
					\else
					\draw node[draw,circle,fill=white] at (0,0.5) {};
					\fi
					
					\ifnum\inti=1
					\draw node[draw,circle,fill=gray] at (0.5,0) {};
					\else
					\draw node[draw,circle,fill=white] at (0.5,0) {};
					\fi
					
					\draw node[draw,circle,fill=white] at (0,-0.5) {};
					\draw node[draw,circle,fill=white] at (-0.5,0) {};
				\end{scope}
			}
		}
		
		
		
		
		\foreach \j in {1,...,1}
		{
			
			\draw[red!50, line width=1.5mm] (0.5, \j) -- (1, \j);
			\draw node[label=center:\textbf{\large $e_1$}] at (0,\j) {};
			\draw node[draw, circle, fill=red!50,label=center:\textbf{$\sigma^x$}] at (0.5,\j) {};
			%\draw node[label=center:\textbf{e2}] at (0,\j) {};
			%\draw[red!50, line width=1.5mm] (-0.5, \j) -- (0, \j);
			
			%\node[draw, circle, fill=red!50,label=center:\textbf{$\sigma^x$}] at (-0.5,1) {};
			
		}
		
		\foreach \i in {-1,...,-1}
		{
			
			\draw[red!50, line width=1.5mm] (\i,-1) -- (\i,0.5);
			\draw node[label=center:\textbf{\large $e_2$}] at (\i,1) {};
			
			\node[draw, circle, fill=red!50,label=center:\textbf{$\sigma^x$}] at (-1,0.5) {};
			\node[draw, circle, fill=red!50,label=center:\textbf{$\sigma^x$}] at (-1,-0.5) {};
			
		}
		
		\foreach \j in {-1,...,-1}
		{
			
			\draw[red!50, line width=1.5mm] (-1, \j) -- (1, \j);
			
			
			\node[draw, circle, fill=red!50,label=center:\textbf{$\sigma^x$}] at (0.5,-1) {};
			\node[draw, circle, fill=red!50,label=center:\textbf{$\sigma^x$}] at (1,-0.5) {};
			\node[draw, circle, fill=red!50,label=center:\textbf{$\sigma^x$}] at (-0.5,-1) {};
			
		}
		
		\foreach \i in {1,...,1}
		{
			
			\draw[red!50, line width=1.5mm] (\i,-0.4) -- (\i,0.4);
			\draw[red!50, line width=1.5mm] (\i,-1) -- (\i,-0.6);
			\draw[red!50, line width=1.5mm] (\i,0.6) -- (\i,1);
			
			\node[draw, circle, fill=red!50,,label=center:\textbf{$\sigma^x$}] at (1,0.5) {};
			
		}
		
		
		
	\end{tikzpicture}
\end{center}

\vspace*{1cm}

\begin{center}
	\begin{tikzpicture}
		% Draw dashed lines
		\foreach \i in {-2,-1.5,...,2}
		{
			\draw[dashed] (\i,-2) -- (\i,2);
		}
		\foreach \j in {-2,-1.5,...,2}
		{
			\draw[dashed] (-2,\j) -- (2,\j);
		}
		
		
		% Draw solid grid and nodes with circles in the middle of each side
		\draw[step=1cm] (-2,-2) grid (2,2);
		\foreach \i in {-1.5,...,1.5}
		{
			\foreach \j in {-1.5,...,1.5}
			{
				\begin{scope}[transform canvas={xshift=\i cm,yshift=\j cm}]
					\node[right,xshift=0.2cm,yshift=0.4cm] {};
					% Convert \j and \i to integers
					\pgfmathtruncatemacro{\intj}{\j}
					\pgfmathtruncatemacro{\inti}{\i}
					
					% Draw circles at the midpoints of each side
					\ifnum\intj=1
					\draw node[draw,circle,fill=gray] at (0,0.5) {};
					\else
					\draw node[draw,circle,fill=white] at (0,0.5) {};
					\fi
					
					\ifnum\inti=1
					\draw node[draw,circle,fill=gray] at (0.5,0) {};
					\else
					\draw node[draw,circle,fill=white] at (0.5,0) {};
					\fi
					
					\draw node[draw,circle,fill=white] at (0,-0.5) {};
					\draw node[draw,circle,fill=white] at (-0.5,0) {};
				\end{scope}
			}
		}
		
		
		
		
		\foreach \j in {1,...,1}
		{
			
			\draw[red!50, line width=1.5mm] (-1, \j) -- (-0.5, \j);
			%\draw node[label=north:\textbf{ \large $e_1$}] at (-1,\j) {};
			\draw[red!50, line width=1.5mm] (-0.5, \j) -- (0, \j);
			%\draw node[label=north:\textbf{ \large $e_2$}] at (0,\j) {};
			\node[draw, circle, fill=red!50,label=center:\textbf{$\sigma^x$},label=left:\textbf{ \large $e_2$},label=right:\textbf{ \large $e_1$}] at (-0.5,1) {};
			
		}
		
	\end{tikzpicture}
\end{center}


\caption{In the first picture we show the creation of a pair of $e$-excitations on the ground state by means of $\sigma^x$ operator. In the following pictures we move around the excitations using a string operator until we reach the configuration in the third picture. In the last passage we apply a closed loop of $\sigma^x$ in order to go back to the first kind of configuartion. \textit{\cite{Rao16}}. }
\label{fig:base}
\end{figure}

%--------------------------

























\newpage
\section{Anyonic excitations}
\label{sec:AE}

Let's examine the mutual statistics between $e$ and $m$ particles by means of an example. Initially, we generate pairs of excitations at distinct chosen sites by applying two $S^x$ and $S?z$ operators as shown in figure 1.16. Afterward, we separate the excitations using string operators and we move an $m$-particle around an $e$-particle, as depicted in figure 1.17. Notice that there is a site where $\sigma^z$ must pass beyond a $\sigma^x$ spin, resulting in an anticommutation and thus introducing a minus sign. {\cite{Kit02}}.
The closed loop can be eliminated, leading to the following transformation:

\begin{center}
	$\sigma_i^z\sigma_j^x|GS\rangle = - \sigma_i^x\sigma^z|GS\rangle$
\end{center}

if we denote the exchange operator as {\it the braiding operator} $R$ then, according to what we said in section 1.4.1, the wave function relation should be the following:

\begin{center}
	$R^2\psi(e,m) = - \psi(e,m)$
\end{center}

since we are going all the way around the particle, we assume to have two exchanges in 2D. Then we simply find out that:

\begin{center}
	$R^2 = - 1 \rightarrow ( e^{-\frac{i\pi}{2}})^2 = - 1  \rightarrow R = e^{-\frac{i\pi}{2}} = \pm i $
\end{center}

this results falls out of the usual bosonic/fermionic statistics, thus we can say that the low-energy excitations of the ground state exhibit anyonic statistics (recalling the fact that we can obtain \textit{any} statistics). {\cite{Rao16, Kit02, Pre04}}.\newline


%picture
\begin{figure}
	\begin{center}
		\begin{tikzpicture}
			% Draw dashed lines
			\foreach \i in {-3,-2.5,...,3}
			{
				\draw[dashed] (\i,-3) -- (\i,3);
			}
			\foreach \j in {-3,-2.5,...,3}
			{
				\draw[dashed] (-3,\j) -- (3,\j);
			}
			
			
			% Draw solid grid and nodes with circles in the middle of each side
			\draw[step=1cm] (-3,-3) grid (3,3);
			\foreach \i in {-2.5,...,2.5}
			{
				\foreach \j in {-2.5,...,2.5}
				{
					\begin{scope}[transform canvas={xshift=\i cm,yshift=\j cm}]
						\node[right,xshift=0.2cm,yshift=0.4cm] {};
						% Convert \j and \i to integers
						\pgfmathtruncatemacro{\intj}{\j}
						\pgfmathtruncatemacro{\inti}{\i}
						
						% Draw circles at the midpoints of each side
						\ifnum\intj=2
						\draw node[draw,circle,fill=gray] at (0,0.5) {};
						\else
						\draw node[draw,circle,fill=white] at (0,0.5) {};
						\fi
						
						\ifnum\inti=2
						\draw node[draw,circle,fill=gray] at (0.5,0) {};
						\else
						\draw node[draw,circle,fill=white] at (0.5,0) {};
						\fi
						
						\draw node[draw,circle,fill=white] at (0,-0.5) {};
						\draw node[draw,circle,fill=white] at (-0.5,0) {};
					\end{scope}
				}
			}
			
			
			
			
			\foreach \j in {1,...,1}
			{
				
				\draw[red!50, line width=1.5mm] (-2, \j) -- (1, \j);
				%\draw node[] at (0,\j) {};
				\draw node[draw, circle, fill=red!50,label=center:\textbf{$\sigma^x$},label=right:\textbf{\large $e$}] at (0.5,\j) {};
				\draw node[draw, circle, fill=red!50,label=center:\textbf{$\sigma^x$}] at (-0.5,\j) {};
				\draw node[draw, circle, fill=red!50,label=center:\textbf{$\sigma^x$},label=left:\textbf{\large $e$}] at (-1.5,\j) {};
				
			}
			
			
			\foreach \j in {-0.5,...,-0.5}
			{
				
				\draw[blue!50, line width=1.5mm] (-2.5, \j) -- (0.5, \j);
				
				
				\node[draw, circle, fill=blue!50,label=center:\textbf{$\sigma^z$},label=left:\textbf{\large $m$}] at (-2,\j) {};
				\node[draw, circle, fill=blue!50,label=center:\textbf{$\sigma^z$},label=right:\textbf{\large $m$}] at (0,\j) {};
				\node[draw, circle, fill=blue!50,label=center:\textbf{$\sigma^z$}] at (-1,\j) {};
				
			}
			
			
		\end{tikzpicture}
	\end{center}
	
	
	\caption{Creation of the $e$-excitations and $m$-excitations.}
	\label{fig:anyons1}
\end{figure}

\begin{figure}
\begin{center}
	\begin{tikzpicture}
		% Draw dashed lines
		\foreach \i in {-3,-2.5,...,3}
		{
			\draw[dashed] (\i,-3) -- (\i,3);
		}
		\foreach \j in {-3,-2.5,...,3}
		{
			\draw[dashed] (-3,\j) -- (3,\j);
		}
		
		
		% Draw solid grid and nodes with circles in the middle of each side
		\draw[step=1cm] (-3,-3) grid (3,3);
		\foreach \i in {-2.5,...,2.5}
		{
			\foreach \j in {-2.5,...,2.5}
			{
				\begin{scope}[transform canvas={xshift=\i cm,yshift=\j cm}]
					\node[right,xshift=0.2cm,yshift=0.4cm] {};
					% Convert \j and \i to integers
					\pgfmathtruncatemacro{\intj}{\j}
					\pgfmathtruncatemacro{\inti}{\i}
					
					% Draw circles at the midpoints of each side
					\ifnum\intj=2
					\draw node[draw,circle,fill=gray] at (0,0.5) {};
					\else
					\draw node[draw,circle,fill=white] at (0,0.5) {};
					\fi
					
					\ifnum\inti=2
					\draw node[draw,circle,fill=gray] at (0.5,0) {};
					\else
					\draw node[draw,circle,fill=white] at (0.5,0) {};
					\fi
					
					\draw node[draw,circle,fill=white] at (0,-0.5) {};
					\draw node[draw,circle,fill=white] at (-0.5,0) {};
				\end{scope}
			}
		}
		
		
		
		
		\foreach \j in {1,...,1}
		{
			
			\draw[red!50, line width=1.5mm] (-2, \j) -- (1, \j);
			%\draw node[] at (0,\j) {};
			\draw node[draw, circle, fill=red!50,label=center:\textbf{$\sigma^x$}] at (0.5,\j) {};
			\draw node[draw, circle, fill=red!50,label=center:\textbf{$\sigma^x$}] at (-0.5,\j) {};
			\draw node[draw, circle, fill=red!50,label=center:\textbf{$\sigma^x$},label=left:\textbf{\large $e$}] at (-1.5,\j) {};
			
		}
		
		\foreach \i in {-1,...,-1}
		{
			
			\draw[red!50, line width=1.5mm] (\i,-1) -- (\i,1);
			\draw node[label=center:\textbf{\large $e$}] at (\i,1) {};
			
			\node[draw, circle, fill=red!50,label=center:\textbf{$\sigma^x$}] at (-1,0.5) {};
			\node[draw, circle, fill=red!50,label=center:\textbf{$\sigma^x$}] at (-1,-0.5) {};
			
		}
		
		\foreach \j in {-1,...,-1}
		{
			
			\draw[red!50, line width=1.5mm] (-1, \j) -- (1, \j);
			
			
			\node[draw, circle, fill=red!50,label=center:\textbf{$\sigma^x$}] at (0.5,-1) {};
			\node[draw, circle, fill=red!50,label=center:\textbf{$\sigma^x$}] at (1,-0.5) {};
			\node[draw, circle, fill=red!50,label=center:\textbf{$\sigma^x$}] at (-0.5,-1) {};
			
		}
		
		\foreach \i in {1,...,1}
		{
			
			\draw[red!50, line width=1.5mm] (\i,-0.4) -- (\i,0.4);
			\draw[red!50, line width=1.5mm] (\i,-1) -- (\i,-0.6);
			\draw[red!50, line width=1.5mm] (\i,0.6) -- (\i,1);
			
			\node[draw, circle, fill=red!50,,label=center:\textbf{$\sigma^x$}] at (1,0.5) {};
			
		}
		
		\foreach \j in {-0.5,...,-0.5}
		{
			
			\draw[blue!50, line width=1.5mm] (-2.5, \j) -- (0.5, \j);
			
			
			\node[draw, circle, fill=blue!50,label=center:\textbf{$\sigma^z$},label=left:\textbf{\large $m$}] at (-2,\j) {};
			\node[draw, circle, fill=blue!50,label=center:\textbf{$\sigma^z$},label=right:\textbf{\large $m$}] at (0,\j) {};
			\node[draw, circle, fill=purple!70,label=center:\textbf{$\pm$}] at (-1,\j) {};
			
		}
		
		
	\end{tikzpicture}
\end{center}


\caption{Creation of an anyonic excitation by braiding of $e$ and $m$ pairs of particles. Here we loop the $e$-excitation around an $m$-excitation. The red loop can be removed with the result that the GS acquires a phase.}
\label{fig:anyons}
\end{figure}



%-------------------------------









\chapter{Computation on the toric code}
\label{ch:chapter_two}

\section{Classical error correction}
\label{sec:CER}

In classical computing the most elementary unit of information is the {\it bit}. A bit stores binary information, which can either assume value 1 or 0. When we transmit information in the classical world, we immagine to make strings of bits travel through a channel similar to the one depicted below in figure 2.1. \newline
During the transmission a bit can change its value either going from 1 to 0 or viceversa, this event is called {\it bit-flip} and it tells us that an error has occured in the transmission. A bit-flip can occur independently on each bit with a certain probability that we will denote as $p << 1$. In such a way we expect each bit to be corrupted after $O(\frac{1}{p})$. \newline
The noisy channel described in picture 2.1, is one of the easiest error models also known as the $\textit{Binary Symmetric Channel}$. An error model, in the context of information theory and communication, is a mathematical or conceptual representation of how errors can occur during the transmission of information. It describes the types of errors that can happen, the probabilities associated with each type of error, and the conditions under which these errors occur. \newline

%image
\begin{figure}
	\centering
	\includegraphics[width=0.7\linewidth]{../Thesis/324193_2_En_1_Fig3_HTML}
	\caption{{\it Binary symmetric channel. From. \cite{Cha06}}}
	\label{fig:3241932en1fig3html}
\end{figure}



In order to protect information what we do is introducing redundancy, which helps us detecting errors and correcting them. Introducing redundacy means duplicating bits of information before sending them through a channel. We say that we encode information into {\it strings of bits}, for example: 0 is encoded into 00 and 1 into 11. This is useful to detect errors because in this way, whenever a bit changes we know that an error has occured, since 10 and 01 pairs should never occur. Formally we have introduced a {\it parity}: the strings 00 and 11 have {\it even} parity while 10 and 01 have {\it odd} parity. Whenever we catch a string with odd parity, we have an error.	\newline
Though, in order to be able to also correct errors, the redundancy above is not enough since we wouldn't be able to define a criterion to correct the corrupted information. Thus, we introduce a third redundant bit; in this way, when we decode information at the destination, we can apply what is called {\it majority voting}. Majority voting decodes into ones those triplets that have most bits set to 1 and decodes as zeros the ones that have most bits set to 0. For example, if we receive 010 we will decode it into 0. \newline
We can still define a parity rule; in this case, we look at two parities: the parity
of the first two bits, and the parity of the second two bits.  For example 000 and 111 have parity 00 while all the other combinations have at least one bit set to 1.  When we check for parity we perform what is called {\it parity-check} and we call the obtained values error {\it syndromes}. If we know the latter, we know what error has occured and how to correct it:

%\begin{center}
\begin{enumerate}
	\item 00 : corresponds to codeword 000 or 111;
	\item 01 : corresponds to codeword 001 or 110;
	\item 10 : corresponds to codeword 100 or 011;
	\item 11 : corresponds to codeword 010 or 101.
\end{enumerate}
%\end{center}

The above redundant code is actually an example of what is called a {\it Linear code}.
We can in fact reframe all of the above in terms of linear algebra.\newline

To do so we define what is called the generator matrix  \textbf{G} as given in \cite{Cha06}, which is the one that we use to introduce redundancy. \textbf{G} encodes the original information into {\it codewords}. For example, if we have a single bit to be encoded in a three bit codeword we can do the following: 

\begin{center}
	\textbf{G} = $[1 \ 1 \ 1]^{T}$  
\end{center}
\begin{center}
	a = $[i]$, where i = 0,1 
\end{center}
\begin{center}
	$[1 \ 1 \ 1]^{T}$ $[0]$ = $[0 \ 0 \ 0]^{T}$
\end{center}

A linear code encoding k-bit messages into an m-bit codespace \textbf{C} is specified by an m $\times$ k matrix {\bf G} whose entries are all elements of $\mathbb{Z}_2$. The generator matrix is used to encode information before sending it through a channel but we have also said that a realistic channel is noisy so it can modify the codewords we send. Thus, we also need to define an operator to specify how an error occurs and how to detect it. We will call such operators $\textbf{N}_j$ $\in \textbf{N}$ where $j$ denotes the element that will be modified by $\textbf{N}_j$ in the original codeword and $\in \textbf{N}$  represents the set of all possible error operators that could happen on the codeword. 
Overall, the codeword is changed by the noise operator into the correspondig codeword $c' =c + n_j$ where $\textbf{n}_j$ is an m $\times$ 1 vector having a 1 at the j-th row and zeros elsewhere; notice that, since we work with bits, addition is always performed modulo 2. \newline

Having defined all of the above we can come to parity-check performed after transmission. Again we define a parity check matrix $\textbf{H}$ having size $m \min k$ $\times$ n.
To check parity we can simply perform multiplication of {\bf H} by the received codeword, the result will be: 

\begin{center}
	\textbf{H}c = 0  
\end{center}

for any valid codeword. Instead, for any corrupted codeword the computation yields:

\begin{center}
	$\textbf{H}c'$ = $\textbf{H} ( c + n_j )$ = $0 + \textbf{H} n_j $  = $\textbf{H}n_j $ 
\end{center}

which is the error syndrome. \newline
The error syndrome serves as a crucial indicator of errors within a coding and decoding system. In the absence of errors or when only a single error occurs, the error syndrome is represented as 0. This indicates a clean transmission. Conversely, in the case of a single error, the error syndrome is denoted as {\bf H}$n_j$, providing information about the specific bit j within the codeword where the error has taken place. Consequently, the corrupted codeword can be successfully decoded by correcting the jth bit. \newline
However, when two errors $n_i$ and $n_j$ are present, the error syndrome becomes {\bf H}($n_i$ + $n_j$). In this case error correction becomes more difficult because there could be a type of error $n_q = n_i + n_j$ that makes error detection ambiguous, as we could correct the q-th bit instead of the j-th bit. Overall, it is possible to have multiple error combinations leading to the same syndrome.\newline
To prevent ambiguity, it is essential to verify that any additional error nq does not satisfy the condition $n_q = n_i + n_j$. If there are no such instances where $n_q$ equals $ n_i + n_j$, the decoding process can proceed with confidence. However, if such instances are detected ambiguity arises. One possible approach is to reject the codeword, prompting a request for retransmission from the sender. Alternatively, refining the encoding/decoding scheme may be considered. \newline

Because of the ambiguity that can arise in error correction it is essential to discuss the detectability and correctability of errors. \newline
We have mentioned how a codeword can be rejected if corrupted and how a receiver can ask to resend information. In the context of safeguarding the code against a set of noise operators, the efficacy of the encoding/decoding scheme is evaluated on its ability to either maintain the integrity of information or detect errors caused by each noise operator. Therefore, we can assert that a code successfully detects a noise operator $\textbf{N}$ if, for every codeword c within the code, the outcome is either $\textbf{N}$c = c or if $\textbf{N}$c is not a member of the code $\textbf{C}$.

\begin{theorem}
	$\textbf{N}$ is detectable by a code if and only if for each $c_m \neq c_n$ in the code, $\textbf{N}c_n \neq c_m $.
\end{theorem} 

Error correction involves an active process where the decoder not only identifies errors but also rectifies them. 
In the context of a code $\textbf{C}$ and a set of error operators $\textbf{N}$, the primary goal is to establish the presence of a decoding procedure capable of correcting errors introduced by $\textbf{N}$. 
Assuming the existence of distinct codewords $c_m$ and $c_n$ within the code, and for specific indices i and j in $\textbf{N}$, let $c_q = \textbf{N}_ic_m = \textbf{N}_jc_n$. 
This expression indicates that applying the error operator $N_i$ to the codeword cm and the error operator $\textbf{N}_j$ to the codeword $c_n$ results in the same state $c_q$. In other words, after these specific errors occur, both codewords end up in the identical state represented by $c_q$. \newline
When an unidentified error in $N$ occurs, resulting in the state cq, the challenge lies in determining whether the original codeword was $c_m$ or $c_n$. 
The error is unidentified meaning that it occurs in the set of error operators $\textbf{N}$, but we don't know which specific error operator caused it.  Because the same state $c_q$ can be obtained from applying different error operators to distinct codewords, the challenge arises when trying to determine whether the original codeword was $c_m$ or $c_n$. In simpler terms, given the state $c_q$, we cannot conclusively say which codeword was initially sent due to the ambiguity in identifying the specific error that occurred. 

\begin{theorem}
	$\textbf{N}$ is correctable if and only if for all $c_m \neq c_n$ in the code and for all i, j, it is true that $\textbf{N}_ic_m$ $\neq$ $\textbf{N}_jc_n$. 
\end{theorem} 

In the context of linear codes, we gain insights into the two essential steps of error recovery: error detection and error correction. For further details see \cite{Cha06}.
When it comes to code design, matrix representations plays a crucial role in studying encoding/decoding procedures, offering an effective representation also to be used in the quantum setting.


%-------------------------------------------





\newpage
\section{Quantum error correction (QEC)}
\label{sec:QEC}



%--------------------------------------------


\newpage
\section{QEC in the toric code}
\label{sec:TC}













\chapter{Conclusions and future developments}
\label{ch:conclusions}%
A final chapter containing the main conclusions of your research/study
and possible future developments of your work have to be inserted in this chapter.

%-------------------------------------------------------------------------
%	BIBLIOGRAPHY
%-------------------------------------------------------------------------

\addtocontents{toc}{\vspace{2em}} % Add a gap in the Contents, for aesthetics
%\bibliography{Thesis_bibliography} % The references information are stored in the file named "Thesis_bibliography.bib"
\begin{thebibliography}{99} % Replace '99' with the widest label in your bibliography
	\bibitem[Cor23]{Cor23} M. Correggi, \textit{Aspetti Matematici della Meccanica Quantistica}, 2023.
	\bibitem[Her20]{Her20} P. Herringer, "The Toric Code", {\it RP}, 2020.
	\bibitem[Kit02]{Kit02} A. Yu. Kitaev, "Fault-tolearnt quantum computation by anyons", {\it Annals of Physics}, 2-30, 2002.
	\bibitem[Rao16]{Rao16} S. Rao, "Introduction to abelian and non-abelian anyons", {\it Harish-Chandra Research Institute}, 1-20, {2016}.
	\bibitem[Wil91]{Wil91} F. Wilczek, "Anyons", {\it Scientific American}, 58-65, 1991.
	\bibitem[Bro14]{Bro14} D. Browne, {\it Topological Codes and Computation}, 2014.
	\bibitem[Odd20]{Odd20} L. Oddis, {\it PhD thesis: Two-Anyon Schrödinger Operators}, 2019-2020.
	\bibitem[Pre04]{Pre04} J. Preskill, {\it Lecture Notes for Physics 219:
	 Quantum Computation}, 2019.
	\bibitem[Cha06]{Cha06} Hsun-Hsien Chang, "An Introduction to Error-Correcting Codes: From Classical to Quantum", {\it arXiv: quant-ph/0602157}, 2006.
	\bibitem[Nie06]{Nie06} M. A. Nielsen, I. L. Chuang, {\it Quantum Computation and Quantum Information}, Cambridge University Press, 2010.
\end{thebibliography}


%-------------------------------------------------------------------------
%	APPENDICES
%-------------------------------------------------------------------------

\cleardoublepage
\addtocontents{toc}{\vspace{2em}} % Add a gap in the Contents, for aesthetics


% LIST OF FIGURES
\listoffigures



% ACKNOWLEDGEMENTS
\chapter*{Acknowledgements}
Here you might want to acknowledge someone.

\cleardoublepage

\end{document}
