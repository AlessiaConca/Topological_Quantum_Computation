\documentclass[margin=10pt]{standalone}
\usepackage{ifthen}
\usepackage[rgb]{xcolor}
\usepackage{tikz}
\usetikzlibrary{cd, arrows, matrix, intersections, math, calc}

\begin{document}
\tikzmath{%
  real \ry, \rz, \longit, \latit, \tox, \toy, \toz;
  real \newxx, \newxy, \newyx, \newyy, \newzx, \newzy;  
  integer \Ny, \Nz, \prevj, \prevk;
  % \j moves around Oy and \k moves around Oz.
  % They must describe full circles of radii \ry and \rz respectively.
  \ry = 4;
  \rz = 1.5;
  \longit = 24;
  \latit = 35;
  \tox = sin(\longit)*cos(\latit);
  \toy = sin(\latit);
  \toz = cos(\longit)*cos(\latit);
  \newxx = cos(\longit); \newxy = -sin(\longit)*sin(\latit);
  \newyy = cos(\latit);
  \newzx = -sin(\longit); \newzy = -cos(\longit)*sin(\latit);
  \Nz = 36;
  \Ny = 84;
  \ktmp = \Nz-1; 
  \jtmp = \Ny-1;
  function isSeen(\j, \k) {
    let \px = cos(360*(\k/\Nz))*cos(360*(\j/\Ny));
    let \py = -sin(360*(\k/\Nz));
    let \pz = cos(360*(\k/\Nz))*sin(360*(\j/\Ny));
    let \res = \px*\tox + \py*\toy + \pz*\toz;
    if \res>0 then {return 1;} else {return 0;};
  };
}
\begin{tikzpicture}[every node/.style={scale=.8},
  x={(\newxx cm, \newxy cm)},
  y={(0 cm, \newyy cm)},
  z={(\newzx cm, \newzy cm)},
  evaluate={%
    int \j, \k;
    for \j in {0, 1, ..., \Ny}{%   \Ny = 84
      for \k in {0, 1, ..., \Nz}{%  \Nz = 36
        \test{\j,\k} = isSeen(\j, \k);
      };
    };
  }]

  % coordinate system $Oxyz$; first layer
  % must be drawn in two steps (there are 2 objects in the final figure)
  \draw[green!50!black]
  (0, 0, 0) -- (\ry, 0, 0)
  % (0, 0, 0) -- (0, \ry+\rz, 0)
  (0, 0, 0) -- (0, 0, \ry);

  % points (P-\j-\k)
  % The minus sign for the y component is due to the fact that
  % the points (for a vertical circle) are to be considered 
  % clockwise starting with 3 o'clock.  Of course, it depends on the
  % observer's position, but in case this position is in the first
  % quadrant, this is the good order.
  \foreach \j in {0, ..., \Ny}{%
    \foreach \k in {0, ..., \Nz}{%
      \path
      ( {( \ry+\rz*cos(360*(\k/\Nz)) )*cos(360*(\j/\Ny))},
      {-\rz*sin(360*(\k/\Nz))},
      {( \ry+\rz*cos(360*(\k/\Nz)) )*sin(360*(\j/\Ny))} )
      coordinate (P-\j-\k);
    }
  }

  % "squares"---the mesh
  % first j then k; in this way the upper "latitude bands" are drawn
  % at the end and the torus appears correctly.
  \foreach \k [remember=\k as \prevk (initially 0)] in {1, ..., \Nz}{%
    \foreach \j [remember=\j as \prevj (initially 0)] in {1, ..., \Ny}{%
      \ifthenelse{\test{\j,\k}=1}{
        \draw[blue!50, very thin, fill=blue!15]
        (P-\j-\prevk) -- (P-\prevj-\prevk)
        -- (P-\prevj-\k) --(P-\j-\k) -- cycle;
      }{}
    }
  }

  % cube inside the torus with one face on the torus defined by
  % latitude and longitude cycles
  \begin{scope}[evaluate={%
      for \j in {0, 1, 2}{ \a{\j} = int(\Ny/4+3+\j); };
      for \k in {0, 1, 2, 3}{ \b{\k} = int(\Nz-3+\k); };
    }]
    % face of the "cube"
    \filldraw[blue!25] (P-\a{0}-\b{0})
    \foreach \k in {1, 2, 3}{-- (P-\a{0}-\b{\k})}
    -- (P-\a{1}-\b{3}) -- (P-\a{2}-\b{3})
    \foreach \k in {2, 1, 0}{-- (P-\a{2}-\b{\k})}
    -- (P-\a{1}-\b{0}) -- cycle;    

    % the "cube"'s four other vertices
    \foreach \j in {0, 2}{%
      \foreach \k in {0, 3}{%
        \path
        ( {( \ry+.5*\rz*cos(360*(\b{\k}/\Nz)) )*cos(360*(\a{\j}/\Ny))},
        {-.5*\rz*sin(360*(\b{\k}/\Nz))},
        {( \ry+.5*\rz*cos(360*(\b{\k}/\Nz)) )*sin(360*(\a{\j}/\Ny))} )
        coordinate (Q-\j-\k);
      }
    }
    % faces of the cube inside the torus
    \filldraw[blue!80, very thin]
    (P-\a{0}-\b{0}) -- (Q-0-0) -- (Q-0-3) -- (P-\a{0}-\b{3}) -- cycle;
    \filldraw[B!50, very thin]
    (P-\a{0}-\b{0}) -- (Q-0-0) -- (Q-2-0) -- (P-\a{2}-\b{0}) -- cycle;

    % longitude cycles
    \foreach \j in {0, 2}{%
      \foreach \k [remember=\k as \prevk (initially 0)] in {1, ..., \Nz}{
        \ifthenelse{\test{\a{\j},\k}=1}{
          \draw[red] (P-\a{\j}-\prevk) -- (P-\a{\j}-\k);
        }{}
      }
    }
    % latitude cycles
    \foreach \k in {0, 3}{%
      \foreach \j [remember=\j as \prevj (initially 0)] in {1, ..., \Ny}{%
        \ifthenelse{\test{\j,\b{\k}}=1}{
          \draw[red] (P-\prevj-\b{\k}) -- (P-\j-\b{\k});
        }{}
      }
    }
  \end{scope}
  
  % coordinate system $Oxyz$; second layer
  \draw[green!50!black, -{Latex[length=5pt, width=5pt]}]
  (\ry+\rz, 0, 0) -- (8, 0, 0) node[right] {$x$};
  \draw[green!50!black, -{Latex[length=5pt, width=5pt]}]
  (0, 0, 0) -- (0, 6, 0) node[above] {$y$};
  \draw[green!50!black, -{Latex[length=5pt, width=5pt]}]
  (0, 0, \ry+\rz) -- (0, 0, 8) node[below left] {$z$};
\end{tikzpicture} 
\end{document}
