\documentclass[12pt]{report}

\usepackage[a4paper, margin=1in]{geometry}

\usepackage[english]{babel}
\usepackage{amsmath} % Add this in the preamble
\usepackage{amssymb}
\usepackage{enumitem}

\title{Tuo Titolo di Laurea}
\author{Tuo Nome}
\date{\today} % O la data desiderata
\newcommand{\university}{Nome dell'Università}
\newcommand{\faculty}{Nome della Facoltà}


\begin{document}
	\maketitle
	
	\begin{abstract}
		Il tuo abstract qui.
	\end{abstract}
	
	\tableofcontents
	
	\pagestyle{headings}
	
	
	\newpage
	\section*{Selected Notation}
	\addcontentsline{toc}{section}{Selected Notation}
	
	\begin{itemize}[label=\textbullet]
		\item we denote an Hermitian operator as $A^H$ instead of $A^*$ in order to distinguish such notation with the one of the conjugate matrix $A^*$.
		
		\item 
		
		\item 
		
	\end{itemize}
	
	
	\chapter{Toric codes}
	% Altre sezioni
	
	
	\begin{minipage}{1\textwidth}
		
		\textbf{Definition 1.1.} (Torus) A torus is a bidimensional square lattice with periodic boundary conditions.\newline 
		
		Let G be a graph with vertices V and edges E, where V is a set of points representing lattice sites, and E is a set of connections between these points. The lattice is bidimensional, meaning each lattice site has two neighbors in the horizontal and vertical directions.\newline
		
		The square lattice can be represented as a regular grid of lattice sites, typically arranged in rows and columns. Mathematically, it can be defined as a pair (V, E) such that:\newline
		
		1. V is a set of points $(x, y)$, where $x$ and $y$ are integers representing the coordinates of lattice sites, and $x, y \in \mathbb{Z} $.\newline
		
		2. E is a set of edges connecting neighboring lattice sites. If $(x1, y1)$ and $(x2, y2)$ are two lattice sites, there is an edge $(x1, y1) - (x2, y2)$ in E if and only if $|x1 - x2| = 1$ and $|y1 - y2| = 0$ or $|x1 - x2| = 0$ and $|y1 - y2| = 1$. This reflects the square lattice's grid structure.\newline
		
		The periodic boundary conditions mean that the lattice wraps around in both the horizontal and vertical directions. This can be expressed as follows:\newline
		
		For any lattice site $(x, y)$, there are periodic boundary conditions such that if $x$ is at the edge of the lattice in the horizontal direction, $(x + 1, y)$ is identified with $(x - Lx + 1, y)$, where $Lx$ is the size of the lattice in the $x$-$direction$, and similarly for the vertical direction. This identification allows for the lattice to be considered as a torus, and lattice sites at the boundaries are connected to their counterparts on the opposite edge.\newline
		
		
		\textbf{Definition 1.2.} (Spin-$\frac{1}{2}$ particle) A fermion with spin $S=\frac{(2k-1)}{2}$ for $k \in \mathbb{N}$ equal to 1.\newline
		
		\textbf{Definition 1.4.} (Vertex and Plaquette operators) Given a vertex v and a plaquette p of a torus we can define the following vertex and plaquette operators as tensor products over Pauli operators acting on individual spins indicated by the indices $j \in star(v)$ and $j \in bdy(p)$  \newline 
		
		\begin{center}
			$ Av = \prod_{j \in star(v)} Z_j $ \newline
			
			$ Bp = \prod_{j \in bdy(v)} X_j $.\newline
		\end{center}
		
		
		
		
		
		
    \end{minipage}
        
    
    \begin{minipage}{1\textwidth}
    	
    	\textbf{Definition 1.3.} (Toric Code Hamiltonian) Given v, p, application sites of,  respectively, vertex and plaquette operators, and given spin-$\frac{1}{2}$ particles located on each edge of a torus, we can define the following Hamiltonian for the system\newline
    	
    	\begin{center}
    		
    		$H = -\sum_{v} 
    		Av - \sum_{p} Bp $.\newline
    		
    	\end{center}
    	
    	Recall the following definitions from linear algebra \newline
    	
    	\textbf{Definition 1.5.} (Linear map) Let $V$, $V'$ be the vector spaces over the field K. A linear mapping $F$; $V \rightarrow V'$ is a mapping that satisfies the following two properties \newline
    	
    	1. For any elements $u,v \in V$ we have $F(u + v)=F(u)+F(v)$\newline
    	
    	2. For all c in $K$ and $v \in V$ we have $F(cv)=cF(v)$\newline
    	
    	
    	\textbf{Definition 1.6.} (Hermitian operator) Let $V$ be  a finite dimensional vector space over $\mathbb{C}$, with a fixed positive definite hermitian product defined as $<v,w>$ for $v,w \in V$. Let $A$: $V \rightarrow V$ be a linear map. An operator is called Hermitian (or self-adjoint) if $A^H=A$. This means that for all $u,v \in V$ we have\newline
    	
    	\begin{center}
    		$<Av,w> = <v,Aw>$.
    	\end{center}
    	
    	In particular, it is important for the following appplications to specify that a square matrix $A$ of complex numbers is called hermitian if $(A^*)^T = A$, i.e. if the conjugate-transpose of $A$ is equal to $A$ itself. Note also that for real matrices it is sufficient to compute only the transpose of the matrix to verify hermiticity. \newline
    	
    	\textbf{Definition 1.7.} (Unitary operator) Let $V$ be  a finite dimensional vector space over $\mathbb{C}$, with a positive definite hermitian product. Let $A$: $V \rightarrow V$ be a linear map. We define A to be complex unitary if \newline
    	
    	\begin{center}
    		$<Av,Aw> = <v,w>$.
    	\end{center}
    	
    	for all $v,w \in V$. We can define a complex matrix $A$ to be unitary if $ (A^*)^T=A^{-1}$. We note that it is possible to define an operator to be real unitary, with the only difference that $(A)^T = A^{-1}$. Then, we can also define that a real matrix is unitary if $(A)^T = A^{-1}$ or, equivalently, if $(A)^T A=I$.\newline
    	
    \end{minipage}  	
	
	\begin{minipage}{1\textwidth}
		
		
	    
	    \textbf{Proposition 1.1.} (Properties of $Av$ and $Bp$ operators) $Av$ and $Bp$ operators are Hermitian and involutory, therefore they have eigenvalues $\pm 1$.
	    \newline
	
     	\textit{Proof}\newline
     	
     	We know that the operator $ Av = \prod_{j \in star(v)} Z_j $ and that $ Bp = \prod_{j \in bdy(v)} X_j $ .\newline
     	
     	Firstly, recall the form of the $\sigma_x$ and $\sigma_z$ matrices representing the Pauli gates $X$ and $Z$:
	
	
	    
	    \[
	    \text{Z = $\sigma_z$} =
	    \begin{bmatrix}
	    	1 & 0 \\
	    	0 & -1
	    \end{bmatrix}
	    \]
	    
	 
	    \[
	    \text{X = $\sigma_x$} =
	    \begin{bmatrix}
	    	0 & 1 \\
	    	1 & 0
	    \end{bmatrix}
	    \]
	    
	    
	    Prove that they are Hermitian:\newline
	    
	    \[
	    \text{$( \sigma_z )^{H}$} = 
	    \begin{bmatrix}
	    	1 & 0 \\
	    	0 & -1
	    \end{bmatrix} ^H =
	    \begin{bmatrix}
	    	1 & 0 \\
	    	0 & -1
	    \end{bmatrix}^T =
	    \begin{bmatrix}
	    	1 & 0 \\
	    	0 & -1
	    \end{bmatrix}
	    = \text{$ \sigma_z $}
	    \]
	    
	    
	    \[
	    \text{$( \sigma_x )^{H}$} = 
	    \begin{bmatrix}
	    	0 & 1 \\
	    	1 & 0
	    \end{bmatrix} ^H =
	    \begin{bmatrix}
	    	0 & 1 \\
	    	1 & 0
	    \end{bmatrix}^T =
	    \begin{bmatrix}
	    	0 & 1 \\
	    	1 & 0
	    \end{bmatrix}
	    = \text{$ \sigma_x $}
	    \]\newline
	    
	    Note that Pauli matrices are real matrices, i.e. $A^H=A^T$.
	    Then, knowing that $[\sigma_i,\sigma_j]=0 for i=j$ we can write		\newline
	    
	    \begin{center}
	    $(Av)^{H} = (\sigma_{x_1} \sigma_{x_2} \sigma_{x_3} \sigma_{x_4})^{H} = \sigma_{x_1}^H \sigma_{x_2}^H \sigma_{x_3}^H \sigma_{x_4}^H = \sigma_{x_1} \sigma_{x_2} \sigma_{x_3} \sigma_{x_4} = Av$ \newline
	    
	    $(Bp)^{H} = (\sigma_{z_1} \sigma_{z_2} \sigma_{z_3} \sigma_{z_4})^{H} = \sigma_{z_1}^H \sigma_{z_2}^H \sigma_{z_3}^H \sigma_{z_4}^H = \sigma_{z_1} \sigma_{z_2} \sigma_{z_3} \sigma_{z_4} = Bp$\newline
	    \end{center}
	    
	     Prove the involutory propertyy:\newline
	    
	    \[
	    \text{$( \sigma_z )^{2}$} = 
	    \begin{bmatrix}
	    	1 & 0 \\
	    	0 & -1
	    \end{bmatrix} *
	    \begin{bmatrix}
	    	1 & 0 \\
	    	0 & -1
	    \end{bmatrix} =
	    \begin{bmatrix}
	    	1 & 0 \\
	    	0 & 1
	    \end{bmatrix}
	    = \text{$I$}
	    \]
	    
	    
	    \[
	    \text{$( \sigma_x )^{2}$} = 
	    \begin{bmatrix}
	    	0 & 1 \\
	    	1 & 0
	    \end{bmatrix} *
	    \begin{bmatrix}
	    	0 & 1 \\
	    	1 & 0
	    \end{bmatrix} =
	    \begin{bmatrix}
	    	1 & 0 \\
	    	0 & 1
	    \end{bmatrix}
	    = \text{$I$}
	    \]\newline
	    
	    
	     Then, again we can write\newline
	    
	    \begin{center}
	    $(Av)^{2} = (\sigma_{x_1} \sigma_{x_2} \sigma_{x_3} \sigma_{x_4})^{2} = \sigma_{x_1}^2 \sigma_{x_2}^2 \sigma_{x_3}^2 \sigma_{x_4}^2 = I$ \newline
	    
	    $(Bp)^{2} = (\sigma_{z_1} \sigma_{z_2} \sigma_{z_3} \sigma_{z_4})^{2} = \sigma_{z_1}^2 \sigma_{z_2}^2 \sigma_{z_3}^2 \sigma_{z_4}^2 = I$\newline
     	\end{center}
	    
    \end{minipage}    
    
	\begin{minipage}{1\textwidth}
	    
	    
	    As regards the second part of the proposition, we prove that Hermitian operators have real eigenvalues: \newline
	    
	    Write the expression for the eigenvalues $Av |\xi> = \lambda |\xi>$ and take as hypothesis that $|\xi> \neq 0$. Then, by means of the scalar product\newline
	    
     	\begin{center}
	    $<\xi|Av|\xi> = \lambda <\xi |\xi>$\newline
	    
	    $\lambda = \frac {<\xi|Av|\xi>}{<\xi |\xi>}$ = $\frac {<\xi|Av|\xi>}{||\xi||^2}$ = $\frac {<\xi|Av^H|\xi>}{||\xi||^2}$ = $\frac {<\xi|Av|\xi>^*}{||\xi||^2}$ = $\lambda^*$\newline
	    \end{center}
	
	    Note that we applied antidistributivity: \newline
	    
	    \begin{center}
	    $(<\xi|Av^H|\xi>)^H$ = $|\xi>^H (Av^H)^H <\xi|^H$ = $<\xi|Av|\xi>^*$. \newline
	    \end{center}
	
	    Using hermiticity with the fact that $(Av)^2=I$ we can derive the unitarity of $Av$ and state that $Av Av^H = Av^H Av = (Av)^2 = I$, i.e. for a unitary operator $U$ we have $U^H=U^{-1}$. \newline
	    
	    One property of unitary operators states that their eigenvalues have modulus equal to one:\newline 
	    
	    
	    By taking as hypothesis $|\xi> \neq 0$, we write the expression $U |\xi> = \lambda |\xi>$ and its self-adjoint $<\xi| U^H = \lambda^* <\xi|$. Then, knowing that $U^H=U^{-1}$, by means of the scalar product we obtain \newline
	    
	    \begin{center}
	    $<\xi|U U^H|\xi> = \lambda \lambda^* <\xi |\xi>$\newline
	    
	    $<\xi|I|\xi> = \lambda \lambda^* <\xi |\xi>$\newline
	    
	    $<\xi|\xi> = \lambda \lambda^* <\xi |\xi>$\newline
	    \end{center}
	    
	    Finally, since we already know that Hermitian operators have real eigenvalues we can write $|\lambda|^2 = 1$.\newline
	    
	    Putting together the fact that $Av$ has real eigenvalues with unitarity we obtain that the only two remaining possibilities for the eigenvalues of $Av$ are $\pm 1$. \newline
	    
	    The same reasoning can be carried out for the $Bp$ operator.
	    
	    \hfill $\square$
	    
	    
	\end{minipage}
	
	\begin{minipage}{1\textwidth}
		
		
		\textbf{Proposition 1.2.} (Spectrum of $Av$ and $Bp$ operators) The spectrum of $Av$ and $Bp$ operators is $\{-1,+1\}$. \newline
		
	    
    	\textbf{Proposition 1.3.} (Commutation of $Av$ and $Bp$ operators) The operator Av commutes with the operator Bp for an even number of edges.
    	\newline
    	
    	\textbf{Proposition 1.4.} (Anticommutation of $Av$ and $Bp$ operators) The operator $Av$ anticommutes with the operator $Bp$ for an odd number of edges.
    	\newline
    	
    	\textit{Proof}\newline 
    	%Av commutes with itself.\newline
    	%Bp commutes with itself.\newline
    	%Av commutes with Bp for an even number of edges.\newline
    	
    	Fix the origin of the coordinate system in the bottom left corner of the lattice as indicated in the picture below:\newline
    	
    	%picture
    	
    	then, we can define the two vectors representing the site of application of the vertex and plaquette operator, respectively over the lattice L and dual lattice L'
    	
    	\begin{center}
    		$\vec{v}$= $n\hat{e_1} + m\hat{e_2}$, where $n,m \in \mathbb{Z}$ \newline
    		
    		$\vec{p}$= $(n + \frac{1}{2}) \hat{e_1} + (m + \frac{1}{2}) \hat{e_2}$, where $n,m \in \mathbb{Z}$\newline
    	\end{center}
    	
    	Rewrite the operators as follows:\newline
    	
    	\begin{center}
    		
    		$A_{\vec{v}} = \sigma^z_{\vec{v}+\frac{1}{2}\hat{e_1}} \sigma^z_{\vec{v}+\frac{1}{2}\hat{e_2}} \sigma^z_{\vec{v}-\frac{1}{2}\hat{e_1}} \sigma^z_{\vec{v}-\frac{1}{2}\hat{e_2}}$ \newline
    		
    		$B_{\vec{p}} = \sigma^x_{\vec{p}+\frac{1}{2}\hat{e_1}} \sigma^x_{\vec{p}+\frac{1}{2}\hat{e_2}} \sigma^x_{\vec{p}-\frac{1}{2}\hat{e_1}} \sigma^x_{\vec{p}-\frac{1}{2}\hat{e_2}}$\newline
    	
    	\end{center}
    	
    	In order to simplify the calculations we rewrite $Bp$ on the lattice L by rewriting the indeces in terms of vector $\vec{v}$
    	
    	\begin{center}
    		$(n\hat{e_1} + m\hat{e_2}) + \frac{1}{2}\hat{e_2}= n\hat{e_1} + (m+\frac{1}{2}\hat{e_2})$\newline
    		
    		$(n\hat{e_1} + m\hat{e_2}) + \frac{1}{2}\hat{e_1}= (n+ \frac{1}{2})\hat{e_1} + m\hat{e_2}$\newline
    		
    		$(n\hat{e_1} + m\hat{e_2}) + \frac{1}{2}\hat{e_1}+\hat{e_2}= (n+ \frac{1}{2})\hat{e_1} + (m + 1)\hat{e_2}$\newline
    		
    		$(n\hat{e_1} + m\hat{e_2}) + \frac{1}{2}\hat{e_2}+\hat{e_1}= (n+ 1)\hat{e_1} + (m + \frac{1}{2})\hat{e_2}$\newline
    	\end{center}
    	
    	Then the $B_{\vec{p}}$ operator becomes:
    	
    	\begin{center}
    		
    		$B_{\vec{v}} = \sigma^x_{n\hat{e_1} + (m+\frac{1}{2}\hat{e_2})} \sigma^x_{(n+ \frac{1}{2})\hat{e_1} + m\hat{e_2}} \sigma^x_{(n+ \frac{1}{2})\hat{e_1} + (m + 1)\hat{e_2}} \sigma^x_{(n+ 1)\hat{e_1} + (m + \frac{1}{2})\hat{e_2}}$\newline
    		
    	\end{center}
    	
    	and the Hamiltonian can be written by grouping the indices:\newline
    	
    	\begin{center}
    	
    	$H = - \sum_{m,n \in \mathbb{Z}} \{ 
    	\sigma^z_{(n+\frac{1}{2})\hat{e_1} + m\hat{e_2}} \sigma^z_{n\hat{e_1}+(m+\frac{1}{2})\hat{e_2}} \sigma^z_{(n-\frac{1}{2})\hat{e_1} + m\hat{e_2}} \sigma^z_{n\hat{e_1}+(m-\frac{1}{2})\hat{e_2}} +
    	\sigma^x_{n\hat{e_1} + (m+\frac{1}{2}\hat{e_2})} \sigma^x_{(n+ \frac{1}{2})\hat{e_1} + m\hat{e_2}} \sigma^x_{(n+ \frac{1}{2})\hat{e_1} + (m + 1)\hat{e_2}} \sigma^x_{(n+ 1)\hat{e_1} + (m + \frac{1}{2})\hat{e_2}} \} $
    	
       \end{center}
    	
	
    \end{minipage}
	
	\begin{minipage}{1\textwidth}
		
		Now calcuate the commutator $[A_{\vec{v}},B_{\vec{v}}] = A_{\vec{v}}B_{\vec{v}} - B_{\vec{v}}A_{\vec{v}}$ by focusiing on the first term:\newline
		
		\begin{center}
			
			$ A_{\vec{v}}B_{\vec{v}} =
			\sigma^z_{(n+\frac{1}{2})\hat{e_1} + m\hat{e_2}} \sigma^z_{n\hat{e_1}+(m+\frac{1}{2})\hat{e_2}} \sigma^z_{(n-\frac{1}{2})\hat{e_1} + m\hat{e_2}} \sigma^z_{n\hat{e_1}+(m-\frac{1}{2})\hat{e_2}} $ *
		
			$\sigma^x_{n\hat{e_1} + (m+\frac{1}{2}\hat{e_2})} \sigma^x_{(n+ \frac{1}{2})\hat{e_1} + m\hat{e_2}} \sigma^x_{(n+ \frac{1}{2})\hat{e_1} + (m + 1)\hat{e_2}} \sigma^x_{(n+ 1)\hat{e_1} + (m + \frac{1}{2})\hat{e_2}}$\newline
			
		\end{center}
		
		
		Matrices do not commute but for Pauli matrices we have the following commutation relationship :\newline
		
		
		\begin{center}
			$\sigma^x_{\vec{v}}\sigma^z_{\vec{v}'} = \sigma^z_{\vec{v}'} \sigma^x_{\vec{v}} + 2* \sigma^x_{\vec{v}}\sigma^z_{\vec{v}'} \delta_{\vec{v} \vec{v}'}$\newline
	    \end{center}
	    
		
		where $\hspace{1cm} \delta_{\vec{v} \vec{v}'} =$
		$\begin{cases}
			1, \hspace{1cm} if \hspace{1cm}  \vec{v} = \vec{v}'\\
			0, \hspace{1cm} if \hspace{1cm} \vec{v} \neq \vec{v}'
		\end{cases}$\newline
		
		which states that 	$\sigma^x_{\vec{v}}\sigma^z_{\vec{v}'}$ commutate for $\vec{v} \neq \vec{v}'$   but anticommutate for $\vec{v} = \vec{v}'$. This is known from the anticommutation relationship of Pauli matrices $\sigma^x \sigma^z = - \sigma^x \sigma^z$. Thus, for an even numer of overlapping edges, in our case 2 or 4, the commutator becomes:\newline
	
	    \begin{center}
	    	
	    	$[A_{\vec{v}},B_{\vec{v}}] = 2 *
	    	\sigma^x_{n\hat{e_1} + (m+\frac{1}{2}\hat{e_2})} \sigma^x_{(n+ \frac{1}{2})\hat{e_1} + m\hat{e_2}} \sigma^x_{(n+ \frac{1}{2})\hat{e_1} + (m + 1)\hat{e_2}} \sigma^x_{(n+ 1)\hat{e_1} + (m + \frac{1}{2})\hat{e_2}}*
	    	\sigma^z_{(n+\frac{1}{2})\hat{e_1} + m\hat{e_2}} \sigma^z_{n\hat{e_1}+(m+\frac{1}{2})\hat{e_2}} \sigma^z_{(n-\frac{1}{2})\hat{e_1} + m\hat{e_2}} \sigma^z_{n\hat{e_1}+(m-\frac{1}{2})\hat{e_2}} $ 	\newline 
	    
	    	
	    \end{center}
	    Instead, for an odd number of edges it becomes null $[A_{\vec{v}},B_{\vec{v}}]=0$.\newline
	    
	    This calculations conclude that $A_{\vec{v}},B_{\vec{p}}$ commute for an even numer of edges but anticommute for an odd number of edges.\newline
		
		\hfill $\square$\newline
		
		
	\end{minipage}
	
	
	
	
	\begin{minipage}{1\textwidth}
		
		\textbf{Proposition 1.5.} (Ground state(s) of the Hamiltonian) The ground states of the Hamiltonian are the simultaneous $+1$ eigenstates of all the $Av$ and $Bp$ operators. \newline
		
		\textit{Proof}\newline 
		
		Recall the form of the Hamiltonianin, having N lattice sites, the following way \newline
		
		\begin{center}
			
			$H = -\sum_{i=1}^{N}
			Av_i - \sum_{j=1}^{N} Bp_j $.\newline
			
		\end{center}
		
		we already know that Hermitian matrices are simultaneously diagonalizable, therefore we can determine the possible eigenvalues for each of the $Av$ and $Bp$ operators. From Proposition 1.2. we know that such eigenvalues can assume two values $\{-1,+1\}$ due to the specific properties of our operators.\newline
		
		To determine the Ground state(s) of the Hamiltonian we have to determine the minimum energy of the system. Therefore we compute all the possible combinations of the eigenvalues associated to the vertex and palquette operators to obtain the following spectrum of values\newline
		
		\begin{center}
		$\sigma( H) =$
		$\begin{cases}
			2N, \hspace{1cm} if \ all\ Av,Bp \ have \ \lambda_{H}= -1\\
			2N-1,\\
			.\\
			intermediate \ energies,\\
			.\\
			-2N+1\\
			-2N, \hspace{1cm} if \ all \ Av,Bp \ have \ \lambda_{H}= +1
		\end{cases}$
		\newline
     	\end{center}
		
		Taking the minimum of this spectrum means $\sigma(H)=-2N$, thus considering only eigenstates for the Hamiltonian associated to $+1$ eigenvalues.\newline
		
		Those will form a basis for the Ground State manifold of the system.\newline
		
		\hfill $\square$\newline
		
		
		Now we focus on determining the form of such ground state(s).\newline
		
	\end{minipage}
	
	
	\begin{minipage}{1\textwidth}
		
		In order to determine which are the admissible configurations that we can use to form the Ground State(s), we treat the $Av$ and $Bp$ operators as constraint equations over the Torus, i.e. the lattice and dual lattice.\newline
		
		All of our configurations will need to respect the following equations:\newline
		
		\begin{center}
		 (1)	$Av|\psi>$ = $+1|\psi>$
		\end{center}
		
		\begin{center}
		 (2)	$Bp|\psi>$ = $+1|\psi>$
		\end{center}
		
		This means that configuration $|\psi>$ needs to be be an eigenvector for both $Av$ and $Bp$ operators. \newline
		
		In order to satisfy equation (1), we look for loop configurations such that if we apply $Av$ to that state, the result would still yield a positive eigenvalue.\newline
		
		Graphically, we identify the loop configurations through strings of 'occupied' edges (here shaded in black), each identified by $|1>$. Then,if we apply $Av$ at one of the open ends of such strings, we have two possibilities: leaving the string of occupied edges open or closing the string over the $Av$ operator (here highlighted in blue).\newline
		
		%drawings
		
		In the first case, what we do is computing $\sigma^{z} |1>$, which implies obtaining $-1|1>$ as a result, thus violating contraint (1). The second case is a natural consequence following the first result. Thus, in order to respect contraint (1) we are interested only in closed loops. notice that such loops will always have an even length and will always involve a maximum of two extremities of the $Av$ operator.\newline
		
		In a more formal notation what we are stating is that:\newline
		
		\begin{center}
			$\prod_{i=1}^{4} \sigma_{i}^{z} |\psi> = +1 |\psi>$. \newline
		\end{center}
		
		Considering only closed loops, we identify different configurations having such characteristic. The below illustrations show some examples of them:\newline
		
		%drawings
		
		Each of these loops is an eigenstate for $Av$, since no matter how I locally apply $Av$, I will always preserve the sign of the state. \newline
	
		
		
		
	\end{minipage}
	
	
	
	\begin{minipage}{1\textwidth}
		
		Then we focus on constraint (2). We want to apply the plaquette operator $Bp$ only to eigenstates of $Av$, which we have determined above. The illustration below shows how to do it with one of the loops above:\newline
		
		%drawings
	
		
		Notice that, after the transformation, we do always end up in a valid eigenstate of $Av$ but the new state $|\psi'>$ is not an eigenstate of $Bp$ by itself.\newline
		
		Furthermore, any new configuration that we obtain through the application of the plaquette operator to an eigenstate of $Av$ simply yields one of the possible permutations of the edges of the initial state, provided that the topological characteristics of the loop are preserved.\newline
		
		This means that if we firstly partition the eigenstates of $Av$ in the following fours classes: \newline
		
		- class 0 : contains all closed loops and thus the vacuum state,
		since all of the closed loops can be continuously deformed into a null state; \newline
		
		- class 1 : contains loops that wind all the way around the horizontal dimension of the torus and their permutations;\newline
		
		- class 2 : contains loops that wind all the way around the vertical dimension of the toru and their permutations;\newline
		
		- class 3 : contains loops that wind all the way around both dimension of the Torus and their permutations; notice that the vertical loop must be taken on the dual lattice to have a valid configuration.\newline
		
		Then, applying a plaquette operator to any of the eigenstates belonging to one of the classes above must yield an eigenstate that lies in the same class. This is formally expressed by stating that the class is invariant under the action of the operator $Bp$. \newline
		
		\textbf{Proposition 1.6.} (Invariance under $Bp$) The four classes of eigenstates of $Av$ are invariant under the action of the $Bp$ operator. \newline
		
		We can prove that the above definition holds by means of counterexamples.\newline
		 
		If we take the loop illustrated below, we would be brought to believe that there indeed exist a way to apply the operator $Bp$ such that we exit class zero and land in class 1. \newline
		
		%drawings
		
		In order to show the impossibility of the above action, we define two topological indeces to label the four categories of eigenstaes, which can assume values in $\mathbb{Z}_2=\{\overline{0},\overline{1} \}$. These elements respectively represent the 'sets' of even and odd numbers. In our context such numerosities identify the number of vertical and horizontal loops trapassing the dimensions of the Torus. \newline
		
		The above indecs can also be expressed as follows: 
		
		\begin{center}
			$n_x= (number \ of \ vertical \ intersections)mod2 = 
			\begin{cases} 
				0mod2 \\
			    1mod2  
			\end{cases}$ 
			$n_y= (number \ of \ horizontal \ intersections)mod2 =\begin{cases} 
				0mod2 \\
				1mod2  
			\end{cases}$ 
		\end{center}
		
		
	
	\end{minipage}
	
	
	
	
	
	\begin{minipage}{1 \textwidth}
		
		in order to create a correspondence with the actual definition of the Torus over $(\mathbb{Z}$ x $\mathbb{Z})$. 
		We only change the representatives of the elements $\{\overline{0},\overline{1} \}$ in the intermediate step. \newline
		
		So, in total our classes are labelled as : $(\overline{0},\overline{0} )$, $(\overline{0},\overline{1} )$, $(\overline{1},\overline{0})$, $(\overline{1},\overline{1})$.\newline
		
		Notice that the intersections are meant to be computed by fixing two circles: one horizontal circle passing through the spins (the princiapla lattice) and one vertical circle passing in the middle of the lattice plaquettes, i.e. taken on the dual lattice.\newline
		 
		For example if we take the below configuration: \newline
		
		%drawing
		
		We have 1 vertical intersection and 1 horizontal intersection, therefore we are in the class three indexed by $n_x=1mod2$ and $n_y=1mod2$, which is labelled as $(\overline{1},\overline{1})$. This would have been true for any odd number of vertical and horizintal intersections, since they all fall in the set of numbers given by $1mod2$.\newline
		
		Going back to the initial example, this would mean that we land not in class 1, but in class 0, as we have an even number of horizontal intersections and an even number of vertical intersections, i.e.   $(\overline{0}, \overline{0})$.\newline
		
		Thus, all the classes are $Bp-invariant$ due to the topological characteristics of the Torus.\newline
		
		More formally we can now write that, given a set of eigenstates of $Av$ named $|\psi_1>,...,|\psi_i>,...,|\psi_n>$ all belonging to the same class and forming state $|\psi>$. Given that we know that $Bp|\psi_i>=+|\psi_j>$, then if we apply $Bp$ to the normalized state $|\psi>$ then we get:
		
		\begin{center}
			$|\Xi>= Bp \frac{1}{\sqrt{n}} \sum_{j=1}^{n} |\psi_i> = \frac{1}{\sqrt{n}} \sum_{j=1}^{n} Bp |\psi_i> = \frac{1}{\sqrt{n}} \sum_{j=1}^{n} |\psi_J> =|\Xi>$
		\end{center}
		
		because what $Bp$ does is only permuting the $|\psi_i>$, thus we get back our initial state. \newline
		
		Which leads to the following Proposition:\newline
		
		\textbf{Proposition 1.7.}(Eigenstates of $Bp$) Eigenstates of $Av$ are not eigenstates of $Bp$ by themselves but completely symmetric superspositions of any of them.\newline
		
		From Proposition 1.6 and Proposition 1.7, naturally follows the degeneracy (and dimension) of the Ground State:\newline

		\textbf{Proposition 1.8.} (Degeneracy of the Ground State manifold) The degeneracy of the Ground State manifold is $4$. \newline
		
		As the four classes of eigenstates are $Bp-invariant$, then we can construct a valid Ground State through the configurations belonging to one the four classes, as these configurations respect both contraint (1) and (2).\newline
		
	\end{minipage}
















 
	
	\begin{minipage}{1 \textwidth}
		\textbf{Exchange statistic in 3D }\newline
		
		Let us first consider the exchange statistics of two particles in three dimensions.  We can describe a system of two particles moving from point $(r_1(t_1),r_2(t_1))$ to point $(r'_1(t_2),r'_2(t_2))$ through the following path integral formulation 
		
		\begin{center}
			$A$ = $\sum_{paths} e^{iS}$
		\end{center}
		
		where S represents the integral of the Lagrangian $L$ over time, which provides a measure of the total "action" along a particular trajectory or path taken by the system. 
		
		\begin{center}
			$S = \int_{t_1}^{t_2}dt \textit{L} [ r_1(t),r_2(t) ] $
		\end{center}
		
		In quantum mechanics, the path integral formulation involves considering all possible paths between the initial and final states and assigning a phase factor $e^{iS}$ to each.\newline
		
		Then, the probability amplitude for a system to evolve from one point to another is given by the sum (or integral) over all paths of these phase factors. Though, in order to obtain the actual probability of the system evolving between the two points we should compute:
		
		\begin{center}
			$P= |A|^2 $
		\end{center}
		
		Overall, we have:\newline
		
		-Probability Amplitude $A$: It is a complex number that encodes both the magnitude and phase information associated with a quantum process. The probability amplitude is used in quantum mechanics, particularly in the context of the path integral formulation, to describe the evolution of a system and calculate probabilities.\newline
		
		-Probability ($P$): It is a real number representing the likelihood of a particular outcome or event. In quantum mechanics, the probability is obtained by taking the squared magnitude of the probability amplitude. Mathematically, this ensures that the probability is a non-negative real value. So, P represents how likely it is that our system of two particles will evolve from a point at t1 to a point at t2.\newline
		
		Since we are talking about a system of indistinguishable particles, we will only have two classes of paths in three dimensions, the direct paths and paths with exchanges as pictured below.\newline
		
		%picture
		
	    This is because, as we can see above, the final configuration at time $t_2$ will always be the same wheter we end up having $(r_1(t_2),r_2(t_2))$ or $(r_2(t_2),r_1(t_2))$, indeed due to indistinguishability.\newline
		
		Though, notice that even though two paths may lead to the same final configuration they can exhibit different behaviors when it comes to considering the relative coordinates. By defining the center of mass $R= \frac{r_1+r_2}{2}$ , it becomes evident that the center of mass motion remains consistent for both paths, while the relative coordinate $r=r_2-r_1$ undergoes distinct changes.\newline
		
		To visualize this in configuration space, consider maintaining the magnitude of the relative coordinate $|r|$ fixed and non-zero, implying that the particles do not intersect. \newline 
		
	
		
	\end{minipage}
	
	\begin{minipage}{1 \textwidth}
		
		In such a scenario, paths are constrained to move along the surface of the sphere due to the fixed magnitude of the relative coordinate. Closed paths on the sphere emerge when particles return to their original positions (no exchange) or reach the antipodal point (exchange). If particles are exchanged a second time, $r$ completes a full revolution around the sphere. \newline
		% in three-dimensional space, a vector with a fixed magnitude can be represented as a point on the surface of a sphere.
		
		Eliminating coincident points ensures that the wave-function remains non-singular and well-defined across all points in configuration space, particularly on the surface of the sphere. Consequently, the phase acquired by the wave-function remains well-defined and invariant under continuous deformations of the path.\newline
		
		Examining the possible phases of the wave-function along three distinct paths: A (no exchange), B (single exchange), and C (two exchanges), as shown in the figure below, we discover distinct characteristics: \newline
		
		-Path A: being a closed loop without exchange, it can be continuously shrunk to a point, implying that the wave-function cannot acquire any phase other than unity.\newline 
		
		-Path B: involving a single exchange, it connects two fixed points on the sphere, making it impossible to shrink to a single point. Consequently, this exchange introduces a non-trivial phase in the wave-function. \newline
		
		-Path C: forming a closed loop with two exchanges, it can be continuously shrunk to a point by envisioning the path as a physical string looped around a sphere. As a result, path C does not acquire any additional phase.\newline
		
		Overall, we only have two classes of paths, those that do not involve any exchange as in class A and C, and those that are characterised by an exchange as in class B.\newline
		Then, if we let $\eta$ represent the phase acquired by the wave function under a single exchange. Since two exchanges are equivalent to no exchange $\eta^2=1$, then it follows that $\eta = \pm 1 $. Therefore, the only possible statistics in three dimensions are Fermi statistics or Bose statistics.\newline
		
		In terms of the path integral formulation that we have provided above, what we will get is thus:
		
		\begin{center}
			$A[r_1(t_1),r_2(t_1) \rightarrow r'_1(t_2),r'_2(t_2)]$ = $\sum_{direct \ paths} e^{iS}$ + $\sum_{exchange \ paths} e^{iS}$
		\end{center}
		
		In terms of the path integral, we can also introduce a phase
		between the two classes of paths and write: 
		
		\begin{center}
			$A[r_1(t_1),r_2(t_1) \rightarrow r'_1(t_2),r'_2(t_2)]$ = $\sum_{direct \ paths} e^{iS}$ + $e^{i \phi} \sum_{exchange \ paths} e^{iS}$
		\end{center}
		
		Since exchanging the particles twice leads again to a direct path, we will have $e^{2i \phi} = 1$. Solving this equation, implies that $\phi$ can only be 0 or $\pi$ giving rise to bosons and fermions. \newline	
		
		-$\phi = 0$: This case leads to the constructive interference of the exchange paths. The resulting particles are known as bosons, and their behavior is characterized by this specific phase condition.\newline
		
		-$\phi = \pi$: This case results in destructive interference between the exchange paths. The particles corresponding to this scenario are referred to as fermions, and their behavior is governed by this particular phase condition.\newline
		
		
		
	\end{minipage}
	
	\begin{minipage}{1 \textwidth}
		\textbf{Exchange statistic in 2D }\newline
		
		In two dimensions the topology of the space configuartion changes. That is, by proceeding in the same way as we did in three dimensions, thus by fixing the magnitude of the relative coordinate we end up moving on a circle.\newline
		
		Though, in this case several possible paths are possible, particularly closed paths.
		If we look at the three classes of paths A,B,C we see that :\newline
		
		-Path A: can be shrunk into a point as it simply moves on the circle, as exemplified in the picture below. \newline
		
		-Path B: cannot be contracted into a point as the endpoints remain fixed. \newline
		
		-Path C: also this kind of paths cannot be shrunk into a point since they wind all  the way around the circle. In three dimensions, the path that forms under two exchanges can be shrunk to a point. This is not possible if the motion is restricted to a plane.\newline
		
		%picture
		
		
		Let $\eta$ represent the phase associated with a single exchange, $\eta^2$
		denote the phase under two exchanges, and $\eta^3$ signify the phase under three exchanges, and so forth. The crucial observation is that, given that the wave-function's modulus remains constant during exchanges, we can express $\eta$ as a phase factor $e^{i\theta}$. \newline
		
		In terms of path integrals what we get as probability amplitude is: \newline
		
		\begin{center}
			$A$ = $\sum_{direct \ paths} e^{iS}$ + $e^{i \phi} \sum_{ one \ exchange} e^{iS}$  + $e^{2i \phi} \sum_{ two \ exchanges} e^{iS}$  + ...
		\end{center}
		
		where for $ \phi = 0, \pi$ we would obtain the usual bosons and fermions statistics, but since in general, $e^{in\phi} \neq 1$ for any $n$ (n exchanges never yield the identity), $\phi$ can be anything so any statistic is possible in two dimensions.\newline
		
		This elucidates the flexibility in obtaining diverse statistics in two dimensions, which are called 'anyonic'.\newline
		
	\end{minipage}
	
	\begin{minipage}{1 \textwidth}
		
		\textbf{Anyons obey Braid Group Statistics}\newline
		
	\end{minipage}
	
	\begin{minipage}{1 \textwidth}
		
		
		
	\end{minipage}
	
	\begin{minipage}{1 \textwidth}
		
		
		
	\end{minipage}
	
	\begin{minipage}{1 \textwidth}
		
		\textbf{Toric Code : excitations of the Ground State }\newline
		\textbf{Statistics and Braiding properties of the excitations}\newline
		
	\end{minipage}
	
	
	\chapter{Fault-tolerant computation }
	% Altre sezioni
	\begin{minipage}{1 \textwidth}
		
		
		
	\end{minipage}
	
	\begin{thebibliography}{99} % "99" è solo un esempio del numero massimo di voci che ti aspetti nella tua bibliografia
		\bibitem{ref1} A. Kitaev, Ann. Phys. (N. Y). 303, 2 (2003),
		arXiv:9707021 [quant-ph].
		
		%\bibitem{ref2}D. Arovas, J. R. Schrieffer, and F. Wilczek,
		%Phys. Rev. Lett. 53, 722 (1984).
		
		%\bibitem{ref3}H. Bartolomei, M. Kumar, R. Bisognin,
		%A. Marguerite, J.-M. Berroir, E. Bocquillon, B. Pla¸cais, A. Cavanna, Q. Dong,
		%U. Gennser, Y. Jin, and G. F`eve, Science
		%(80-. ). 368, 173 (2020), arXiv:2006.13157.
		
		\bibitem{ref4} D. Browne, (2014). "Topological Codes and Computation,".
		
		\bibitem{ref5} S. B. Bravyi and A. Y. Kitaev, (1998).
		arXiv:9811052 [quant-ph].
		
		\bibitem{ref6}  Rao, S., (2017). Introduction to abelian and non-abelian anyons. In: Bhattacharjee, S., Mj, M., Bandyopadhyay, A. (eds) Topology and Condensed Matter Physics. Texts and Readings in Physical Sciences, vol 19. Springer, Singapore. %https://doi.org/10.1007/978-981-10-6841-6_16
		
		%\bibitem{ref6}R. Raussendorf, J. Harrington, and
		%K. Goyal, New J. Phys. 9, 199 (2007),
		%arXiv:0703143 [quant-ph].
		
		\bibitem{ref7} L. Oddis, Ph.D. Thesis in Mathematics: "Two-Anyon Schrodinger Operators".
		
		\bibitem{ref} A. Hatcher, Algebraic Topology, Cambridge University Press, 2002.
		
		\bibitem{ref8} Nielsen, M. A., Chuang, I. L. (2011). Quantum Computation and Quantum Information: 10th Anniversary Edition. Cambridge University Press.
		
		\bibitem{ref9} Preskill, J. (1998). Fault-tolerant quantum computation. An Introduction to quantum computation and information (pp. 213-269).
		
		\bibitem{ref} A. Pondini, (2020) Quantum error correction e toric code.
		
		\bibitem{ref10} Daley, A.J., Bloch, I., Kokail, C. et al. Practical quantum advantage in quantum simulation.
		
		\bibitem{ref11} Herman, D., Googin, C., Liu, X. et al. Quantum computing for finance. Nat Rev Phys 5, 450–465 (2023).
		
		\bibitem{ref12} V.E. Elfving, B.W. Broer, M. Webber, J. Gavartin, M.D. Halls, K. P. Lorton, A. Bochevarov, arXiv:2009.12472 [quant-ph].
		
		\bibitem{ref12} S. Lang, Linear Algebra, Springer (pp. 191-205).
		
		\bibitem{ref12} Wilczek, F. (1991). Anyons. Scientific American, 264(5), 58–65. http://www.jstor.org/stable/24936902
  
	\end{thebibliography}
	

\end{document}
