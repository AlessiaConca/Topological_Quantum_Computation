
\documentclass[12pt]{report}

\usepackage[a4paper, margin=1in]{geometry}

\usepackage[english]{babel}
\usepackage{amsmath} % Add this in the preamble
\usepackage{amssymb}
\usepackage{enumitem}

\title{Tuo Titolo di Laurea}
\author{Tuo Nome}
\date{\today} % O la data desiderata
\newcommand{\university}{Nome dell'Università}
\newcommand{\faculty}{Nome della Facoltà}


\begin{document}


	\begin{minipage}{1\textwidth}
		
	The ground state of the toric code, represents the system's lowest energy configuration. What makes this ground state particularly intriguing is its highly degenerate nature, meaning that there are numerous distinct configurations that all share the same minimum energy. 
	
	The ground state exhibit topological order, characterized by non-local correlations between spins. This means that the model is robust  even when the system undergoes local perturbations.  In fact, quantum entanglement between spins is not confined to neighboring pairs, instead, it extends over long distances and involves collective behaviours of spins that cannot be understood by considering them individually.
	
	Not only, as we will see, these features are intimately tied to the underlying topology of the lattice structure. This topological protection ensures the robustness of quantum information against local perturbations, giving the toric code the potential to be feasibly applied in quantum information processing.\newline
	
	We start by explicitly defining the ground state by making use of the properties of the vertex and plaquette operators that we have derived in the previous section.\newline
	
	\textbf{Proposition 1.5.} (Ground state of the toric code Hamiltonian) The ground states of the toric code Hamiltonian are the simultaneous $+1$ eigenstates of all the $Av$ and $Bp$ operators. \newline
	
	\textit{Proof.}\newline 
	Recall the form of the Hamiltonianin, having N lattice sites, in the following way \newline
	
	\begin{center}
		
		$H = -\sum_{i=1}^{N}
		Av_i - \sum_{j=1}^{N} Bp_j $.\newline
		
	\end{center}
	
	we already know that Hermitian matrices are simultaneously diagonalizable, moreover $Av$ and $Bp$ operators commute for an even number of edges, thus there exists a common base of eigenvectors with their respective eigenvalues. We also know from proposition 1.2 that such eigenvalues can assume two values $\{-1,+1\}$ due to the specific properties of our operators.\newline
	
	To determine the Ground state of the Hamiltonian we have to determine the minimum energy of the system. Therefore we compute all the possible combinations of the eigenvalues associated to the vertex and palquette operators to obtain the following spectrum of values\newline
	
	\begin{center}
		$\sigma( H) =$
		$\begin{cases}
			2N, \hspace{1cm} if \ all\ Av,Bp \ have \ \lambda_{H}= -1\\
			2N-1,\\
			.\\
			intermediate \ energies,\\
			.\\
			-2N+1\\
			-2N, \hspace{1cm} if \ all \ Av,Bp \ have \ \lambda_{H}= +1
		\end{cases}$
		\newline
	\end{center}
	
	Taking the minimum of this spectrum means $\sigma(H)=-2N$, thus considering only eigenstates for the Hamiltonian associated to $+1$ eigenvalues.\newline
	Those are the ones that will form a basis for the Ground State manifold of the system.\newline
	
	\hfill $\square$\newline
	
	
	Now we focus on determining the form of such ground state(s).\newline
	
\end{minipage}


\begin{minipage}{1\textwidth}
	
	In order to determine which are the admissible configurations that we can use to form the ground state, we treat the $Av$ and $Bp$ operators as constraint equations over the Torus, i.e. the lattice and dual lattice.\newline
	
	All of our configurations will need to respect the following equations:\newline
	
	\begin{center}
		(1)	$Av|\psi\rangle$ = $+1|\psi\rangle$
	\end{center}
	
	\begin{center}
		(2)	$Bp|\psi\rangle$ = $+1|\psi\rangle$
	\end{center}
	
	This means that configuration $|\psi\rangle$ needs to be be an eigenvector for both $Av$ and $Bp$ operators. \newline
	
	In order to satisfy equation (1), we look for loop configurations such that if we apply $Av$ to that state, the result would still yield a positive eigenvalue.\newline
	
	Graphically, we identify the loop configurations through strings of 'occupied' edges (here shaded in black), each identified by $|1\rangle$. Then,if we apply $Av$ at one of the open ends of such strings, we have two possibilities: leaving the string of occupied edges open or closing the string over the $Av$ operator (here highlighted in blue).\newline
	
	%drawings closed/open loop with Av
	
	In the first case, what we do is computing $\sigma^{z} |1\rangle$, which implies obtaining $-1|1\rangle$ as a result, thus violating contraint (1). The second case is a natural consequence following the first result. Thus, in order to respect contraint (1) we are interested only in closed loops. notice that such loops will always have an even length and will always involve a maximum of two extremities of the $Av$ operator.\newline
	
	In a more formal notation what we are stating is that:\newline
	
	\begin{center}
		$\prod_{i=1}^{4} \sigma_{i}^{z} |\psi\rangle = +1 |\psi\rangle$. \newline
	\end{center}
	
	Considering only closed loops, we identify different configurations having such characteristic. The below illustrations show some examples of them:\newline
	
	%drawings 4 types of loops
	
	Each of these loops is an eigenstate for $Av$, since no matter how I locally apply $Av$, I will always preserve the sign of the state. \newline
	
	
	
	
\end{minipage}



\begin{minipage}{1\textwidth}
	
	Then we focus on constraint (2). We want to apply the plaquette operator $Bp$ only to eigenstates of $Av$, which we have determined above. The illustration below shows how to do it with one of the loops above:\newline
	
	%drawings Bp applied to loops
	
	
	Notice that, after the transformation, we do always end up in a valid eigenstate of $Av$ but the new state $|\psi'>$ is not an eigenstate of $Bp$ by itself.\newline
	
	Furthermore, any new configuration that we obtain through the application of the plaquette operator to an eigenstate of $Av$ simply yields one of the possible permutations of the edges of the initial state, provided that the topological characteristics of the loop are preserved.\newline
	
	This means that if we firstly partition the eigenstates of $Av$ in the following fours classes: \newline
	
	- class 0 : contains all closed loops and thus the vacuum state,
	since all of the closed loops can be continuously deformed into a null state; \newline
	
	- class 1 : contains loops that wind all the way around the horizontal dimension of the torus and their permutations;\newline
	
	- class 2 : contains loops that wind all the way around the vertical dimension of the toru and their permutations;\newline
	
	- class 3 : contains loops that wind all the way around both dimension of the Torus and their permutations; notice that the vertical loop must be taken on the dual lattice to have a valid configuration.\newline
	
	Then, applying a plaquette operator to any of the eigenstates belonging to one of the classes above must yield an eigenstate that lies in the same class. This is formally expressed by stating that the class is invariant under the action of the operator $Bp$. \newline
	
	\textbf{Proposition 1.6.} (Invariance under $Bp$) The four classes of eigenstates of $Av$ are invariant under the action of the $Bp$ operator. \newline
	
	We can prove that the above definition holds by means of counterexamples.\newline
	
	If we take the loop illustrated below, we would be brought to believe that there indeed exist a way to apply the operator $Bp$ such that we exit class zero and land in class 1. \newline
	
	%drawing loop chiuso ma unito da Bp
	
	In order to show the impossibility of the above action, we define two topological indeces to label the four categories of eigenstaes, which can assume values in $\mathbb{Z}_2=\{\overline{0},\overline{1} \}$. These elements respectively represent the 'sets' of even and odd numbers. In our context such numerosities identify the number of vertical and horizontal loops trapassing the dimensions of the Torus. \newline
	
	The above indecs can also be expressed as follows: 
	
	\begin{center}
		$n_x= (number \ of \ vertical \ intersections)mod2 = 
		\begin{cases} 
			0mod2 \\
			1mod2  
		\end{cases}$ 
		$n_y= (number \ of \ horizontal \ intersections)mod2 =\begin{cases} 
			0mod2 \\
			1mod2  
		\end{cases}$ 
	\end{center}
	
	
	
\end{minipage}





\begin{minipage}{1 \textwidth}
	
	in order to create a correspondence with the actual definition of the Torus over $(\mathbb{Z}$ x $\mathbb{Z})$. 
	We only change the representatives of the elements $\{\overline{0},\overline{1} \}$ in the intermediate step. \newline
	
	So, in total our classes are labelled as : $(\overline{0},\overline{0} )$, $(\overline{0},\overline{1} )$, $(\overline{1},\overline{0})$, $(\overline{1},\overline{1})$.\newline
	
	Notice that the intersections are meant to be computed by fixing two circles: one horizontal circle passing through the spins (the princiapla lattice) and one vertical circle passing in the middle of the lattice plaquettes, i.e. taken on the dual lattice.\newline
	
	For example if we take the below configuration: \newline
	
	%drawing 1 vertical 1 hor
	
	We have 1 vertical intersection and 1 horizontal intersection, therefore we are in the class three indexed by $n_x=1mod2$ and $n_y=1mod2$, which is labelled as $(\overline{1},\overline{1})$. This would have been true for any odd number of vertical and horizintal intersections, since they all fall in the set of numbers given by $1mod2$.\newline
	
	Going back to the initial example, this would mean that we land not in class 1, but in class 0, as we have an even number of horizontal intersections and an even number of vertical intersections, i.e.   $(\overline{0}, \overline{0})$.\newline
	
	Thus, all the classes are $Bp-invariant$ due to the topological characteristics of the Torus.\newline
	
	More formally we can now write that, given a set of eigenstates of $Av$ named $|\psi_1\rangle,...,|\psi_i\rangle,...,|\psi_n\rangle$ all belonging to the same class and forming state $|\psi\rangle$. Given that we know that $Bp|\psi_i\rangle=+|\psi_j\rangle$, then if we apply $Bp$ to the normalized state $|\psi>$ then we get:
	
	\begin{center}
		$|\Xi\rangle= Bp \frac{1}{\sqrt{n}} \sum_{j=1}^{n} |\psi_i\rangle = \frac{1}{\sqrt{n}} \sum_{j=1}^{n} Bp |\psi_i\rangle = \frac{1}{\sqrt{n}} \sum_{j=1}^{n} |\psi_J\rangle =|\Xi\rangle$
	\end{center}
	
	because what $Bp$ does is only permuting the $|\psi_i\rangle$, thus we get back our initial state. \newline
	
	Which leads to the following Proposition:\newline
	
	\textbf{Proposition 1.7.}(Eigenstates of $Bp$) Eigenstates of $Av$ are not eigenstates of $Bp$ by themselves but completely symmetric superspositions of any of them.\newline
	
	From Proposition 1.6 and Proposition 1.7, naturally follows the degeneracy (and dimension) of the Ground State:\newline
	
	\textbf{Proposition 1.8.} (Degeneracy of the Ground State manifold) The degeneracy of the Ground State manifold is $4$. \newline
	
	As the four classes of eigenstates are $Bp$-$invariant$, then we can construct a valid Ground State through the configurations belonging to one the four classes, as these configurations respect both contraint (1) and (2).\newline
	
\end{minipage}



\end{document}
